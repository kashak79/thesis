\chapter{State of the art}

We examined a broad range of different topics with social media as central subject. It was an evolution starting from straightforward Twitter-related subjects to a full-fledged problem assignment many people struggle with.

This chapter exists of two main parts. The first part (\ref{technologies}) is a summary of the different technologies examined in regard with the problem statements together with a short explanation. The second part (\ref{opinionmining}, \ref{twitterinfluencers} and \ref{expertfinding}) states the different examined topics. For each topic there might be several references to technologies described in the first which could be useful in solving the problem. %Feel free to start reading the second section while skipping back to the first when more information is wanted.


\section{Technologies}
\label{technologies}

This is a temporary section, containing a list of keywords and technologies we use in our thesis. Each of these is explained (or should be), but they should be merged into the previous text.

%\subsection{What do we need}

%%%%%%%%%%%%%%%%%%%%%%%%%%%%%%%%%%%%%%%%%%%%%%%%%%%%%%%%%%

%% Several components we still have to explain or discuss inside the text %%

\subsection{Web scraping}
\label{scraping}

Web scraping (also called Web harvesting or Web data extraction) is a computer software technique of extracting information from websites. Usually, such software programs simulate human exploration of the Web by implementing low-level Hypertext Transfer Protocol (HTTP). Web scraping focuses on the transformation of unstructured data on the Web, typically in HTML format, into structured data that can be stored and analyzed in a central local database or spreadsheet. Web scraping is also related to Web automation, which simulates human Web browsing using computer software \footnote{http://en.wikipedia.org/wiki/Web scraping}.

We will use scraping in several of our plugins to collect information about authors and publications as we don't have a large database with this information and constructing one ourselves would take too much time and resources. The downside is that it is typically a slow process as the content of each page has to be downloaded and parsed by the computer.

%\subsection{API}
%\label{api}

%% TODO : verzetten! Deze opsomming van sources zou beter vermeld worden bij plugins, deze zijn niet echt nodig in dit hoofdstuk

%\subsection{DBLP}
%
%DBLP is produced by the Computer Science department of the University of Trier and was initially focused on \textit{DataBase systems and Logic Programming}, but has gradually expanded toward being an confidential server providing bibliographic information on major computer science journals and proceedings, indexing more than one million articles.
%DBLP allows to search by author name, giving back a list of publications and other bibliographic information.
%
%DBLP does not provide us with an API, so we will use web scraping in order to extract all the necessairy information.
%
%%Link naar de volledige uitleg van de werking van de plugin voor meer informatie hieromtrent
%
%\subsection{LinkedIn}
%
%LinkedIn is a business-related social networking site launched in May 2003, mainly used for professional networking by more than 100 million registered users \footnote{http://en.wikipedia.org/wiki/LinkedIn}. Each user has a profile which may contain the following information: current affiliation and title, past working experiences, education, specialties, location, connections with other users...
%
%Using the LinkedIn API we can search for users by entering a name. This will give back list of profiles we can browse and whose information we can extract and use in our framework. As the service is based on a \textit{gated-access approach}, which means you need to be at least a second level connection of the profile you are looking at to see their connections, we can not use this to connect authors to eachother. However, the information provided by the profile which is publicly available does allow us to get a better insight into the subject the person is interested in and what affiliation he is and was connected to.
%
%%Link naar de volledige uitleg van de werking van de plugin voor meer informatie hieromtrent
%
%\subsection{Google Scholar}
%
%Google Scholar is a freely accessible web search engine that indexes the full text of scholarly literature accross an array of publishing formats and disciplines. It allows to search publications based on subject keywords, partial publication titles and author names. It's ranking algorithm uses a combination of factors, but puts mainly high weight to citation count and the words included in a document's title.
%
%Google Scholar is not yet available to the Google AJAX API and Google does not allow automatic crawling or scraping of its services. Thus, it can not be used as a publication reference in our framework.
%
%\subsection{Microsoft Academic Search}
%
%Microsoft Academic Search is a free search engine for academic papers and resources principally in the field of computer science, developed by Microsoft Research Asia, Beijing. The database consists of the bibliographic information (metadata) for academic papers published in journals, conferences proceedings and the citations between them. Objects are ranked according to two factors: their relevance to the query, which is computed by its attributes; and their global importance, calculated by its relationships with other objects \footnote{http://en.wikipedia.org/wiki/Microsoft Academic Search}.
%
%It is a direct competitor of Google Scholar and it allows us to scrape the information of the site, making it useful to extract publications written by a specific author, his fields of interest, the amount of times he was cited and his co-authors for our framework.

\subsection{OWL}
\label{owl}

% TODO explain

\subsection{FOAF}
\label{foaf}

FOAF is a descriptive vocabulary expressed using the Resource Description Framework (RDF) and the Web Ontology Language (OWL). Computers may use these FOAF profiles to find, for example, all people living in Europe, or to list all people both you and a friend of yours know. This is accomplished by defining relationships between people. Each profile has a unique identifier (such as the person's e-mail addresses, a Jabber ID, or a URI of the homepage or weblog of the person), which is used when defining these relationships \footnote{http://en.wikipedia.org/wiki/FOAF (software)}.

% Meer uitleggen wat wij hier mee zijn en hoe dit gebruikt kan worden.

\subsection{Stemming}
\label{stemming}

In linguistic morphology, stemming is the process for reducing inflected (or sometimes derived) words to their stem, base or root form � generally a written word form. The stem need not be identical to the morphological root of the word; it is usually sufficient that related words map to the same stem, even if this stem is not in itself a valid root. 

Algorithms for stemming have been studied in computer science since 1968. The Porter stemming algorithm \cite{porter1980algorithm} dates back to 1980, but is very widely used and the de-facto standard for the English language.

Closely related to stemming is lemmatisation, the algorithmic process of determining the lemma for a given word. The difference is that a stemmer operates on a single word, while a lemmatiser also has knowledge of the context. Stemmers typically run faster and the reduced accuracy may not matter for some applications.

\subsection{Clustering}
\label{clustering}

Cluster analysis or clustering is the assignment of a set of observations into subsets (called clusters) so that observations in the same cluster are similar in some sense. Clustering is a method of unsupervised learning, and a common technique for statistical data analysis used in many fields, including machine learning, data mining, pattern recognition, image analysis, information retrieval, and bioinformatics \footnote{http://en.wikipedia.org/wiki/Cluster analysis}.

There are a lot of different graph clustering algorithms. Algorithms based on spectral clustering \cite{kannan2000clusterings}, multilevel graph partitioning schemes like METIS \cite{karypis1999fast} and the MLKM algorithm \cite{dhillon2005fast}. 

More recently there has been a new direction to graph clustering, modeling the minimum-cut, maximum-flow problem of the underlying graph \cite{flake2004graph}. The algorithm we will use is the dynamic version described in \cite{saha2006dynamic}. It is a dynamic algorithm based on the minimum-cut tree problem. It produces a high quality of clusters without having to look at the entire graph but only a subset of nodes, severely reducing the time of the algorithm.


\subsection{Machine learning}
\label{machinelearning}

Machine learning, part of artificial intelligence, focusses on the development of algorithm that allow computer programs to learn from experience. This can be obtained by constructing rules and constraints, starting from some training data.

There are many different approaches with different computational performances. With regard to \ref{opinionmining}, the most interesting is probably support vector machines which uses a set of related supervised learning methods for classification and regression. Clustering is also a form of machine learning and has already been explained in more detail in \ref{clustering}.

\subsection{Forward Chaining}
\label{forwardchaining}

% TODO explain

\subsection{String matching}
\label{stringmatching}

% TODO explain

\subsection{Part-Of-Speech Tagger}
\label{postagger}

%http://nlp.stanford.edu/software/tagger.shtml

A Part-Of-Speech Tagger, or POS Tagger, is a piece of software that reads text in some language and assigns parts of speech to each word. With parts of speech we mean nouns, verbs, adjectives, etc. Generally computational applications will use more fine-grained POS tags like 'noun-plural'. 

The POS Tagger we refer to is the Stanford Part-Of-Speech Tagger. This is a Java implementation of the log-linear part-of-speech taggers described in the papers \cite{toutanova2000enriching} and \cite{toutanova2003feature}. The software can be downloaded from \url{http://nlp.stanford.edu/software/tagger.shtml}.

\subsection{Graph representation}
\label{graphs}

% TODO explain

%\subsection{Gremlin}

%\subsection{JUNG}

%\subsection{Neo4j}

\section{Opinion mining}
\label{opinionmining}

\subsection{Introduction}
\label{general - opinion mining}
Opinion mining is part of the general area of \textit{sentiment analysis, opinion extraction or opinion mining and feature-based opinion summarization} from the user-generated content or user-generated media on the Web. The applications are manifold with the most important ones in the area of in business intelligence. 

%Large companies receive thousands of pieces of feedback on a daily basis, both direct as indirect. Examples are online customer reviews, customer feedback, survey responses, social media messages, blogposts and comments. Human processing of such text volumes is prohibitively expensive and close to impossible. The only alternative is automatic extraction of relevant information. Ideally one would like to be able to quickly and cheaply customize a system to provide reasonably accurate sentiment classification for a domain, a brand or a specific product.

\subsection{Current situation}

%Sentiment360, lots of papers and onderzoek, really to much to name and to add something of importance.

Opinion mining has been a hot topic the last 10 years due to the rise of social media such as blogs and social networks. Businesses are looking to automate the process of following up on the conversation about their company image and products and opinion mining can help them take steps toward accomplishing this \cite{wright2009mining}.

One step towards this is accomplished in research. Several universities around the world have research teams focussing on the dynamics of sentiment. There is research focused on creating sentiment summaries to capture an author's opinion about a subject based on a publication \cite{beineke2003exploration}, predicting the semantic orientation of adjectectives and combinations of adjectives \cite{hatzivassiloglou1997predicting}, identifying the sources of the opinions rather than the actual sentiment itself \cite{choi2005identifying}...

An interesting ongoing project is \cite{liu2004sentiment} at the University of Illinois, Chicago, backed by Google and Microsoft Corporation. It is a project focussing on three areas: 

\begin{itemize}
	\item Mining direct (or regular) opinions. Example: after taking the drug, I got stomach pain.
	\item Mining comparative opinions. Example: Coke tastes better than Pepsi.
	\item Review and opinion spam analysis and detection. An example is detecting of fake reviews.
\end{itemize}

It is particularly interesting as others can follow the status of and updates on the project. There are also references to the opinion lexicon and data sets they used to evaluate their results allowing third parties to compare their own work using the same input.

\subsection{Improvements and additions}

It could be interesting to work out previous research experiments with a focus on Twitter. The short and typically cryptic messages exchanged by users on this service form a new challenge where existing solutions may fall short as users often use shorthand codes, acronyms and abbreviations which are not part of the official language. A new approach or combination of existing solutions could be tested and compared to existing results.

There are many possible approaches. We could focus on the use of a log-linear regression model to predict the semantic orientation of adjectives and conjunctions \cite{hatzivassiloglou1997predicting}. Another possibility is a holistic lexicon-based approach which exploits external evidences and linguistic conventions of natural language expressions allowing to handle opinion words that are context dependant \cite{ding2008holistic}. In \cite{mei2007topic} a specifically designed Hidden Markov Model structure is utilized to track opinions in weblogs. In \cite{beineke2003exploration} a Naive Bayes and regularized logistic regression models are composed in order to capture opinions from online reviews.

Most of the approaches use some sort of classification algorithm or machine learning technique \ref{machinelearning} in order to correctly predict opinions in a new context. In order to only consider interesting words, we would use a part-of-speech tagger \ref{postagger}. This allows us to strip all the words that do not influence an opinion. On the remaining words, we would use a Stemming algorithm or a Lemmatizer \ref{stemming}.

% TODO waarom hebben we dit niet gekozen ? 


\section{Twitter influencers}
\label{twitterinfluencers}

\subsection{Introduction}
As discussed in \ref{general - opinion mining} about opinion mining, people talk about products, both positive as negative. They have the ability to influence the buying behavior of others who respect their opinion about a certain area. Identifying these influencers can be of great value, for instance in the advertising industry.

There are two main aspects, reach and trust. A person's reach determines the number of people who listen when he has something of value to say. Trust depicts the value people give a certain person's opinion. Both aspects can vary a lot when comparing different topics for the same person.

\subsection{Current situation}

There are papers discussing algorithms to find the ideal subset of individuals which will trigger a large cascade of conversions and papers researching the more general economic issues regarding influencers.

Also regarding the second aspect, trust, there are a lot of well documented scientific results. \cite{ziegler2005propagation} describes propagation models which can be used to present trust and distrust in social networks. Just as in \cite{golbeck2006inferring} an algorithm for calculating a trust metric is presented based on the EigenTrust algorithm\cite{kamvar2003eigentrust}.

The amount of applications focusing on calculating social media influence, as their core business or a useful option, is growing rapidly.

Klout is a well-known online tool focusing on ranking Twitter profiles (and recently also Facebook profiles) using over 35 variables. These rankings are, though fun, not particularly useful as finding influencers based on topic is very limited, however there are interesting third-party applications providing this feature.

PeerIndex and Traackr focus more on identifying topics. Just like Klout, PeerIndex focuses on Twitter profiles and parameters captured from Twitter to assign a certain score to each profile. Traackr on the other hand uses data from multiple online sources connected to the users such as their blog, YouTube channel, LinkedIn and Twitter profile and more. Traackr combines the results and provides a three-way score (reach, relevance and resonance) based on a certain topic together with detailed contact information.

\subsection{Improvements and additions}

%We kunnen hier nog zo veel meer over vertellen, dit is eigelijk de voornaamste voorloper van ons echt onderwerp, dus hier kunnen we reeds heel wat belangrijke dingen aankaarten die we uiteindelijk ook echt zullen gebruiken. 
%Belangrijke dingen die nog aangekaart moeten worden: Semantic Web (Web of Data), DBPedia, RDF, OWL, FOAF, Google Social Graph, location based?, 

We want to create a platform that identifies key influencers on Twitter and that allows users to search influencers using topic keywords. The platform should also maintain up-to-date user profiles.

The main features are based on the Semantic Web. It should allow the framework to understand certain semantics and make connections between information. The usage of the Web Ontology Language (OWL) \ref{owl} for representing the knowledge we have on one influencer together with the Resource Description Framework (RDF) \ref{rdf}. To describe the relations with other influencers and their activities we would us Friend of a Friend (FOAF) \ref{foaf}, which uses RDF to do so.

To find the main subjects of a person we would parse their tweets using a POS tagger \ref{postagger} to find the important words and search what they talk about the most. We would have to make connections between subjects (as people often talk about some feature of a product or a specific product without referencing the general topic). This means we would have to use some kind of machine learning algorithm \ref{machinelearning} such as a clustering algorithm \ref{clustering} to correctly make connections between tweets and subjects. We would also use some external datasets such as Freebase to help us in this quest.

%\section{Event detection}
%
%\subsection{General problem statement}
%
%
%\subsection{Current situation}
%
%\subsection{Improvements and additions}


\section{Expert Finding}
\label{expertfinding}


\subsection{Introduction}

Finding experts is useful for seeking consultants, collaborators or speakers. It is also of great value within the academic world as it provides a source of information to supplement or complement papers and theses. Many researchers and reporters lose a lot of time doing this manually as the amount of sources is ever growing: documents, email, databases, conferences, scientific papers and so on. The topic is luckily seeing an increase in attention in recent years.

%Expert finding is a difficult task requiring a multitude of different steps in order to find results bearing certain value. There are databases containing records of people matched with their area of expertise, which can be queried for a nominal fee, mostly used by reporters. The data has been gathered manually over time. To receive a more up-to-date or specific result, one will have to send a separate request which will come with a higher price and take its time.\\
%Another option is to do it manually. There are several steps which can be undertaken and they can be executed in fashionable order, repeated as often as necessary. One can search for conference sites about the topic and look up the names of the speakers. Current and past working experience can possibly be found on LinkedIn. Publications and co-authors can be retrieved from online databases with dedicated research publications or from more general databases such as Google Scholar. The authors and papers can be interlinked in order to see who works with who and who is most often cited or referenced. Influence can be measured on social media.\\
%It is clearly a lengthy process.

\subsection{Current situation}

% Say something about using OWL, FOAF, RDF, triplestores ... Generally saying more about the 2 month we needed to actually get to our problem ...

\cite{balog2006finding} proposed four simple binary association methods to find expertise information from emails. \cite{liu2005finding} investigated the expertise of users and experts by combining information retrieval techniques. Both these solutions are insufficient for topic-based expert finding as their datasets (emails and online communities) are too limited, they focus too much on previous encounters and lack context.

\cite{mattox1999enterprise} retrieves experts based on the amount of documents persons have for a given topic. As input a keyword phrase is used in order to find relevant documents. The results are however unsatisfactory caused by its slow response time and incorrect relationship between persons and documents. \cite{zhu2006open} gets better results using an algorithm based on a PageRank for document authority, a co-occurrence model for authors and multiple levels of associations between experts and topics. It succeeds to map variants of experts' names on the same author, but fails to identify different authors with the same name.

\cite{jung2007finding} proposes OntoFrame. It is an information service platforum using Semantic Web technologies and is based on an extensive ontology of 16 classes using RDF triples. Identity resolution and full text analysis forms the basis of their expert-finding method. The framework looks promising, however the prototype does not function  \url{http://ontoframe.kr/2008/2008_new/main.jsp}.

\subsection{Improvements and additions}

%Iets vermelden over de connectie met het TWIRL project?

We want to continue the research proposed in \cite{balog2006finding}, but less focused towards the full text analysis of documents. We want to create a platform that extracts and unifies the required information from a variety of online sources and subsequently builds a repository of user profiles.

%We looked into the usage of a triplestore in order to use OWL \ref{owl} ... or sth like that ?

We will need to write a web scraper \ref{scraping} in order te retrieve information from online sources that do not provide a public API. This information will have to be stored and this storage should allow us to make and change connections between different entities and adding and combining them graciously. We choose to use the TinkerPop Graph, using the graph databases \ref{graphs}.

In order to differentiate between authors with the same name and find similarities between publications that are written by the same author, we will use a clustering algorithm \ref{clustering}. The nodes will be names we found in publications or on profiles and clusters will represent one author, mapping one or more names to the same author. This allows us to define rules which calculate the probability that names match the same author and using this new information in the clustering algorithm will lead to reformations in the graph space.

For the clustering algorithm, we focussed on the algorithm proposed in \cite{saha2006dynamic}. This is an efficient, dynamic algorithm which can handle the flow of new information in the form of new nodes and new edges connecting these nodes. For the required minimum-cut tree algorithm, we based ourselves on the sequential Gusfield's algorithm explained in \cite{rodrigues2011mpi}. A very detailed elaboration of everything related to this clustering algorithm can be found in \ref{clustering}.

We want to find experts about certain subjects. This means we have to find keywords related to publications. We will discuss different possibilities, but one is parsing the most important words from the title. We will do this by using the Stanford POS tagger \ref{postagger}.

%Te bespreken: neo4j, Tinkerpop, Gremlin, Stanford POS tagger, lemmatizer, doc split (ruby gem), resque (ruby library - https://github.com/defunkt/resque), max flow :: min-cut
