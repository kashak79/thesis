\section{Clustering algorithm}

\subsection{Introduction}

The core of our framework is the algorithm responsible for deciding what nodes match to the same author, based on similarities between these nodes. These similarities are the output generated by the rules implemented in the Disambiguator. Using these similarities the algorithm can create new information which could be new input for the Disambiguator. It is clear that we need a dynamic approach as there is a constant flow of new information.

We need an efficient algorithm for clustering and it has to be able to handle a steady stream of new information while maintaining an optimal solution. These clusters contain instances which map to the same author and will be used to determine the field of expertise for each author later on. The quality of this algorithm will make or break the quality of the output of our framework.

For the algorithm we based ourselves largely on the algorithm described in \cite{saha2006dynamic}. An efficient dynamic algorithm is described for clustering graphs for handling insertion and deletion of edges, maintaining high quality clusters as defined by the quality requirement given in \cite{flake2004graph}. The algorithm makes heavy use of the minimum-cut tree. We implemented Gusfield's algorithm, based on \cite{rodrigues2011mpi}, and will explain this first.

The main feature of this dynamic algorithm is that it only builds part of the minimum-cut tree as and when necessary. The tree is computed over a subset of nodes, limited to a number of clusters. In our graph there might be an immense number of authors, and thus clusters. However, the amount of authors involved with one change is limited. By just computing the clusters of this coarsened graph we can obtain the clusters of the original graph. These two properties are the reason the algorithm is efficient while maintaining an identical cluster quality as the static version described in \ref{staticcutclustering}. Formal proof for this quality can be found in \cite{saha2006dynamic}.

\subsection{Minimum-cut tree algorithm}
\label{minimumcuttree}

\begin{algorithm}
\caption{Sequential Gusfield's Algorithm}
\label{mincutgusfield}
\begin{algorithmic}
\STATE \textbf{Input:} $G = (V,E,w)$ 
\STATE \textbf{Output:} $T = (V,E,f)$, where T is a cut tree of G
\STATE $V(T) \leftarrow V(G); E(T) \leftarrow \emptyset$
\FOR{$tree_i, flow_i, 1 \leq i \leq N$}
\STATE $tree_i \leftarrow 1; flow_i \leftarrow 0$
\ENDFOR
\STATE // $n - 1$ maximum flow iterations
\FOR{$s \leftarrow 2 $to$ N$}
\STATE $flow_s \leftarrow $MaxFlow$(s, tree_s)$
\STATE // adjust the $tree$ with Cut($s,tree_s$)
\STATE // c1 contains s and connected nodes, c2 contains $tree_s$ and connected nodes
	\FOR{$t \leftarrow 1 $ to $ N$}
		\IF{$t == s \vee t == tree_s$}
			\STATE next
		\ELSIF{$t \in c1 \wedge s \in c2$}
			\STATE $tree_t \leftarrow s$
		\ELSIF{$t \in c2 \wedge s \in c1$}
			\STATE $tree_t \leftarrow tree_s$
		\ENDIF
	\ENDFOR
\ENDFOR
\STATE // Generate T
\FOR{$s \leftarrow 1 $ to $ N$}
\STATE $E(T) \leftarrow E(T) \cup {s, tree_s}$
\STATE $f({s,tree_s}) \leftarrow flow_s$
\ENDFOR
\RETURN T
\end{algorithmic}
\end{algorithm}

% Perhaps this first paragraph should be moved to the SOTA

We briefly explain what is understood under a minimum cut tree, as clarified in \cite{saha2006dynamic}. 

Let $G = (V,E,w)$ denote a weighted undirected graph with $n = |V|$ nodes or vertices and $m = |E|$ links or edges. Each edge $e = (u, v), u,v \in V$ has an associated weight $w(u,v) > 0$. Let $s$ and $t$ be two nodes in $G(V,E)$, the source and destination. The minimum-cut of G with respect to $s$ and $t$ is a partition of $V$ which we will call $S$ and $V/S$. These partitions should be such that $s \in S, t \in V/S$ and the total weight of the edges linking nodes between the two partitions is minimum. The sum of these weights is called the cut-value and is denoted as $c(S,V/S)$. 

The minimum cut tree is a tree on $V$ such that inspecting the path between $s$ and $t$ in the tree, the minimum-cut of $G$ with respect of $s$ and $t$ can be obtained. Removal of the minimum weight edge in the path yields the two partitions and the weight of the corresponding edge gives the cut-value.

We implemented a sequential version of Gusfield's algorithm which calculates the minimum cut tree of any given graph. The pseudocode is given by \ref{mincutgusfield}. In the pseudocode we use numbers to point to nodes or vertices. These numbers can be choosen randomly.

The algorithm consists of $n-1$ iterations of a Maximum Flow algorithm and for every iteration a different vertex is chosen as source. The destination vertex is determined by previous iterations and is saved in the tree. Initially all vertices of the output tree point to node 1, but this can be adjusted after each iteration. This adjustment depends on the minimum-cut between the current source and destination. We split all the nodes in two collections, using this minimum-cut. We adjust the parent of each node if it is on another side as its current parent, which is stored in the tree.

We choose an implementation of the Edmonds-Karp algorithm to find the maximum flow and the minimum-cut. This algorithm is an implementation of the Ford-Fulkerson method for computing the maximum flow and is provided to us through the java library JUNG.

\subsection{Cut clustering}

\cite{flake2004graph} defines a static algorithm for clustering based on minimum cut trees. \ref{staticcutclustering} gives the pseudocode of the basic cut clustering algorithm. It adds an artificial sink $t$ to all the vertices of the graph with weight $\alpha > 0$. The minimum cut tree is computed using this new graph and the disjoint components obtained after removing the artificial vertex $t$, are the required clusters.

\begin{algorithm}
\caption{Static Cut Clustering Algorithm of \cite{flake2004graph}}
\label{staticcutclustering}
\begin{algorithmic}
\STATE \textbf{Input:} $G = (V,E,c), \alpha$ 
\STATE \textbf{Output:} Cluster of G
\STATE $V \leftarrow V \cup t$
\FOR{all vertices $v$ in G}
	\STATE Connect $t$ to $v$ with edge of weight $\alpha$
\ENDFOR
\STATE $G'(V',E') \leftarrow$ new graph after connecting t to V
\STATE Calculate the minimum-cut tree $T'$ of $G'$
\STATE Remove $t$ from T
\RETURN All connected components as clusters of G
\end{algorithmic}
\end{algorithm}

In \cite{saha2006dynamic}, they have extended this basic algorithm allowing it to work efficiently on dynamic graphs. They use some important new components which we also will use throughout the explanation of the algorithm. The adjacency matrix $A$ of $G$ is an $n \times n$ matrix in which $A(i,j) = w(i,j)$ if $(i,j) \in E$, else $A(i,j) = 0$. The algorithm also maintains two new variables for every vertex, the In Cluster Weight (ICW) and the Out Cluster Weigh (OCW). If $C_1,C_2,...C_s$ are the clusters of $G(V,E)$ then ICW and OCW are defined as below.

\begin{mydef}
\textbf{In Cluster Weight (ICW)} of a vertex $v \in V$ is defined as the total weight of the edges linking the vertex $v$ to all the vertices which belong to the same cluster as $v$. That is, if $v \in C_i$, $0 \leq i \leq s$ then $ICW(v) = \sum_{u \in C_i}{w(v,u)}$
\end{mydef}

\begin{mydef}
\textbf{Out Cluster Weight (OCW)} of a vertex $v \in V$ is defined as the total weight of the edges linking the vertex $v$ to all the vertices which do not belong to the same cluster as $v$. That is, if $v \in C_i$, $0 \leq i \leq s$ then $OCW(v) = \sum_{u \in C_j, j \neq i}{w(v,u)}$
\end{mydef}

Receiving new information is represented as a new edge between two nodes with a given weight. The weight represents the amount the two nodes are connected, this is calculated by the rules in the disambiguator. There are two different possibilities that have to be treated separatly, named inter- and intra-cluster edge addition.

We assume we have $C = {C_1,C_2...C_s}$ as the clusters of the graph $G(V,E)$ which have been calculated in previous steps. We denote $A$ as the adjacency matrix of $G$.

\paragraph{Intra-cluster edge addition}

Intra-cluster edge addition means that both the nodes of the added edge belong to the same cluster. The result is that the cluster becomes better connected. We only have to update the $ICW$ and the adjacency matrix $A$ while the nodes remain unchanged. \ref{intracluster} shows the pseudocode.

\begin{algorithm}
\caption{Intra-cluster edge addition between nodes $i$ and $j$ with weight $w(i,j)$}
\label{intracluster}
\begin{algorithmic}
\STATE \textbf{Input:} $G(V,E), (i,j), w(i,j)$ 
\STATE \textbf{Output:} Clusters of G
\STATE $A(i,j) \leftarrow A(i,j) + w(i,j)$
\STATE $ICW(i) \leftarrow ICW(i) + w(i,j)$
\STATE $ICW(j) \leftarrow ICW(j) + w(i,j)$
\RETURN $C$
\end{algorithmic}
\end{algorithm}

\paragraph{Inter-cluster edge addition}

Addition of an edge whose end nodes belong to different clusters is more challenging as it increases the connectivity across different clusters. This means the cluster quality of the clusters involved is lowered and as a result reclustering might be necessary when the cluster quality is no longer maintained. 

There are three identifiable cases:

\begin{enumerate}
	\item \textbf{CASE 1} The addition of the edge does not break the clusters involved
	\item \textbf{CASE 2} The addition of the edge causes the clusters to be so well connected that they are merged into one
	\item \textbf{CASE 3} The new edge deteriorates the cluster quality and the nodes in both the clusters have to be reclustered
\end{enumerate}

In order to understand the complete algorithm shown in \ref{intercluster} for inter-cluster edge addition which also uses the minimum-cut tree algorithm discussed in \ref{minimumcuttree}, we have to explain two new processes: merging and contraction of clusters.

\subparagraph{Merging of clusters} occurs in CASE 2 and is described in \ref{merging}. Two clusters $C_u$ and $C_v$ are merged into one new cluster containing the nodes of the two original clusters. This causes the ICW of all the nodes involved to increase and the OCW to decrease as all the nodes are now more connected.

\begin{algorithm}
\caption{Merging of clusters $C_u$ and $C_v$}
\label{merging}
\begin{algorithmic}
\STATE \textbf{Input:} $C_u$ and $C_v$ 
\STATE \textbf{Output:} Merged cluster
\STATE $D \leftarrow C_u \cup C_v$
\FOR{$\forall u \in C_u$}
	\STATE $ICW(u) \leftarrow ICW(u) + \sum_{v \in C_v}{w(u,v)}$
	\STATE $OCW(u) \leftarrow OCW(u) - \sum_{v \in C_v}{w(u,v)}$
\ENDFOR
\FOR{$\forall v \in C_v$}
	\STATE $ICW(v) \leftarrow ICW(v) + \sum_{u \in C_u}{w(v,u)}$
	\STATE $OCW(v) \leftarrow OCW(v) - \sum_{u \in C_u}{w(v,u)}$
\ENDFOR
\RETURN $D$
\end{algorithmic}
\end{algorithm}

\subparagraph{Contraction of clusters} occurs in CASE 3 as part of the reclustering process and is shown in \ref{contracting}. All the nodes outside the set of clusters $S$ are replaced by a single new node $x$. Self loops that are created in this process are removed and parallel edges are replaced by a single edge with weight equal to the sum of the parallel edges. The reason we consider the clusters outside $S$ is because $S$ will generally be small while the other clusters will contain a lot of nodes.

\begin{algorithm}
\caption{Contraction of clusters outside the set of clusters $S$}
\label{contracting}
\begin{algorithmic}
\STATE \textbf{Input:} $G(V,E)$ and set of clusters $S$ 
\STATE \textbf{Output:} Contracted graph $G'(V',E')$
\STATE Add all vertices of $S$ to a new graph $G'$
\STATE $\forall i,j \in V' : A'(i,j) \leftarrow A(i,j)$
\STATE Add a new vertex $x$ to $G'$
\FOR{$\forall i \in \left\{V' - x\right\}$}
	\STATE $A'(i,x) = ICW(i) + OCW(i) - \sum_{j \in \left\{ V' - x \right\} }{A'(i,j)}$
\ENDFOR
\STATE Obtain $E'$ from $A'$
\RETURN $G'(V',E')$
\end{algorithmic}
\end{algorithm}


\begin{algorithm}
\caption{Inter-cluster edge addition between nodes $i$ and $j$ with weight $w(i,j)$}
\label{intercluster}
\begin{algorithmic}
\STATE \textbf{Input:} $G(V,E), (i,j), w(i,j)$ and $\alpha$ 
\STATE \textbf{Output:} Clusters of $G$
\STATE $i \in C_u$ and $j \in C_v$
\IF{$\frac{\sum_{u \in C_u}{OCW(u) + w(i,j)}}{\left|V - C_u\right|} \leq \alpha \wedge \frac{\sum_{v \in C_v}{OCW(v) + w(i,j)}}{\left|V - C_v\right|} \leq \alpha$}
	\STATE // CASE 1
	\STATE $A(i,j) \leftarrow A(i,j) + w(i,j)$
	\STATE $OCW(i) \leftarrow OCW(i) + w(i,j)$
	\STATE $OCW(j) \leftarrow OCW(j) + w(i,j)$
	\RETURN $C$
\ELSIF{$\frac{2 * c(C_u,C_v)}{V} \geq \alpha$}
	\STATE // CASE 2
	\STATE $D \leftarrow MERGE(C_u,C_v)$
	\RETURN $C + D - \left\{C_u,C_v\right\}$
\ELSE
	\STATE // CASE 3
	\STATE $G'(V',E') \leftarrow CONTRACT(G(V,E),\left\{C_u,C_v\right\} )$
	\STATE Connect $t$ to $v, \forall v \in C_u,C_v$ with edge of weight $\alpha$
	\STATE Connect $t$ to $V' - \left\{C_u, C_v\right\}$ with edge of weight $\alpha * \left\| V - C_u - C_v \right\|$
	\STATE G''(V'',E'') is the graph resulting in connecting $t$
	\STATE Calculate minimum-cut tree $T''$ of $G''(V'',E'')$
	\STATE Remove $t$
	\STATE // $\left\{D_1,D_2...D_k\right\}, k > 0$, are the connected components of $T''$ after removing $t$
	\STATE $C \leftarrow \left\{D_1,D_2...D_k,C_1,C_2...C_s\right\} - \left\{ C_u,C_v \right\}$
	\RETURN $C$
\ENDIF
\end{algorithmic}
\end{algorithm}

% TODO : bij merge, moet A, OCW en ICW updaten en bij CASE 3 hetzelfde, dit wordt niet vermeld in de paper en ook nog niet in onze pseudocode