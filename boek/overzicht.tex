%  Overzichtsbladzijde met samenvatting

\newpage

{
\setlength{\baselineskip}{32pt}
\setlength{\parindent}{0pt}
\setlength{\parskip}{18pt}

\begin{center}

\noindent \textbf{\huge
Dynamische discovery en }
\textbf{\huge gedistribueerde co\"{o}rdinatie van een}
\textbf{\huge semantisch beschreven robot swarm}

\setlength{\baselineskip}{12pt}
\setlength{\parindent}{0pt}
\setlength{\parskip}{12pt}

door 

Niels BOUTEN

Promotoren: Prof. Dr. Ir. Filip De Turck, Prof. Dr. Ir. Bart Dhoedt\\
Begeleiders: Anna Hristoskova, Femke Ongenae, Jelle Nelis

Masterproef ingediend tot het behalen van de academische graad van\\
Master in de ingenieurswetenschappen: computerwetenschappen

Academiejaar 2010-2011\\
Faculteit Ingenieurswetenschappen\\
Voorzitter: Prof. Dr. Ir. Dani\"{e}l De Zutter\\
Vakgroep Informatietechnologie\\

\end{center}

\setlength{\baselineskip}{10pt}
\setlength{\parindent}{0pt}
\setlength{\parskip}{10pt}

\renewcommand{\baselinestretch}{1.1} 	% De interlinie afstand wat vergroten.
\small\normalsize                       % Nodig om de baselinestretch goed te krijgen.

\section*{Samenvatting}

In de huidige swarm robotics toepassingen moeten interacties tussen robots worden geprogrammeerd in de applicaties. Het toevoegen van nieuwe robots aan de swarm of interacties met andere netwerkapparaten kan niet zonder aanpassingen aan de applicatie. Om flexibel gebruik te kunnen maken van deze functionaliteit, is er nood aan een manier om deze apparaten dynamisch te ontdekken en te gebruiken. Het gebruik van voorgeprogrammeerde invocaties limiteert echter de flexibiliteit. Dit kan worden opgelost door niet enkel de interfacedetails, maar ook de betekenis en functionaliteit te defini\"{e}ren in de servicebeschrijving.


Een mogelijke oplossing bestaat erin discovery protocollen zoals UPnP te combineren met semantische beschrijvingen van de services in OWL-S. Deze beschrijvingen moeten dan worden ge\"{e}valueerd ten opzichte van de gezochte services en de huidige context. Zo kan voor iedere aanvraag de op dat moment best passende service worden geselecteerd en aangeroepen, zonder dat deze oproep expliciet moet worden voorgeprogrammeerd.


Binnen deze thesis wordt deze aanpak uitgewerkt met als resultaat een framework voor dynamische service discovery met behulp van semantische beschrijvingen. De verschillende eigenschappen van dit framework, zoals automatische discovery, semantische matching, contextafhankelijke selectie en monitoring van services, worden door middel van een proof of concept scenario aangetoond.

\section*{Trefwoorden}

Swarm Robotics, Service Discovery, Semantische beschrijving, Gedistribueerde werking

}

\newpage % strikt noodzakelijk om een header op deze blz. te vermijden
