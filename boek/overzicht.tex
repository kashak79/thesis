%  Overzichtsbladzijde met samenvatting

\newpage

{
\setlength{\baselineskip}{32pt}
\setlength{\parindent}{0pt}
\setlength{\parskip}{18pt}

\begin{center}

\noindent \textbf{\huge
Identifying experts through }
\textbf{\huge a framework for knowledge extraction}
\textbf{\huge from public online sources}

\setlength{\baselineskip}{12pt}
\setlength{\parindent}{0pt}
\setlength{\parskip}{12pt}

door 

Simon Buelens, Mattias Putman

Promotors: Prof.~Dr.~Ir.~Filip~De~Turck,~Elena~Tsiporkova~(Sirris),~Tom~Tourw\'{e}~(Sirris)\\
Scriptiebegeleiders: Anna~Hristoskova,~Tim~Wauters\\

Masterproef ingediend tot het behalen van de academische graad van\\
Master in de ingenieurswetenschappen: computerwetenschappen

Academiejaar 2010-2011\\
Faculteit Ingenieurswetenschappen\\
Voorzitter: Prof. Dr. Ir. Dani\"{e}l De Zutter\\
Vakgroep Informatietechnologie\\

\end{center}

\setlength{\baselineskip}{10pt}
\setlength{\parindent}{0pt}
\setlength{\parskip}{10pt}

\renewcommand{\baselinestretch}{1.1} 	% De interlinie afstand wat vergroten.
\small\normalsize                       % Nodig om de baselinestretch goed te krijgen.

\section*{Samenvatting}

Onderzoekers verliezen veel tijd met de zoektocht naar informatie gerelateerd aan hun onderzoeksdomein. Er bestaan bijna geen diensten die toelaten om aan de hand van trefwoorden een overzicht te verkrijgen met experts voor de opgegeven domeinen. Er is onderzoek naar disambiguatie van auteurs, maar deze worden meestal niet in combinatie gebracht met het opzoeken van experten, maar het indelen van publicaties (alhoewel de twee gerelateerd zijn).

In deze masterproef onderzoeken we de mogelijkheid om een framework op te stellen dat dit toelaat door online informatie op te zoeken, deze informatie in relatie te brengen met de correcte auteurs en gebruikers toe te laten dit framework te gebruiken om hierin te zoeken. De nadruk van het framework ligt op de disambiguatie van auteurs (aan de hand van de aanwezige informatie alle namen zo goed mogelijk connecteren met de juist auteur) aan de hand van een regelgebaseerde aanpak en de uitbreidbaarheid van het framework.

We maken gebruik van een graafgebaseerde representatie van de data en de architectuur is gebaseerd op pipes en filters. Dit laat toe dat het framework uitbreidbaar, schaalbaar en eenvoudig aanpasbaar is. Op het einde volgen de resultaten gebaseerd op een manueel geannoteerde testset. Uiteindelijk gaan we ook de vergelijking aan met de verdeling van auteurs door DBLP.

\section*{Trefwoorden}

auteur disambiguatie, data verwerking, clustering, pipes en filters

}

\newpage % strikt noodzakelijk om een header op deze blz. te vermijden
