%
%  THESISBOEK
%
%  Dit bestand zorgt voor algemene (layout)definities, en groepeert de
%  afzonderlijke LaTeX-files tot een geheel.
%
%  @author Erwin Six, David De Reu, Brecht Vermeulen
%

\documentclass[11pt,a4paper,oneside,notitlepage]{book}
\usepackage[english]{babel}
\usepackage{algorithmic}
\usepackage{algorithm}
\usepackage{amsthm}
\usepackage{hyperref}
\usepackage{array}
%\usepackage[nottoc]{tocbibind} % Bibliografie in ToC; zie tocbibind.dvi

% marges aanpassen
% (opmerking: moet *voor* inclusie van fancyhdr package komen)
\setlength{\hoffset}{-1in}
\setlength{\voffset}{-1in}
\setlength{\topmargin}{2cm}
\setlength{\headheight}{0.5cm}
\setlength{\headsep}{1cm}
\setlength{\oddsidemargin}{3.5cm}
\setlength{\evensidemargin}{3.5cm}
\setlength{\textwidth}{16cm}
\setlength{\textheight}{23.3cm}
\setlength{\footskip}{1.5cm}

\usepackage{fancyhdr}
\usepackage{graphicx}
% \usepackage[colorlinks]{hyperref}
% Het bibliografisch opmaak bestand.
\bibliographystyle{unsrt}
%\bibliographystyle{bibliodutch}
%\bibpunct{[}{]}{,}{n}{,}{,}

\newtheorem{mydef}{Definition}

\pagestyle{fancy}

\renewcommand{\chaptermark}[1]{\markright{\MakeUppercase{#1}}}
\renewcommand{\sectionmark}[1]{\markright{\thesection~#1}}

\newcommand{\headerfmt}[1]{\textsl{\textsf{#1}}}
\newcommand{\headerfmtpage}[1]{\textsf{#1}}

\fancyhf{}
\fancyhead[LE,RO]{\headerfmtpage{\thepage}}
\fancyhead[LO]{\headerfmt{\rightmark}}
\fancyhead[RE]{\headerfmt{\leftmark}}
\renewcommand{\headrulewidth}{0.5pt}
\renewcommand{\footrulewidth}{0pt}

\fancypagestyle{plain}{ % eerste bladzijde van een hoofdstuk
  \fancyhf{}
  \fancyhead[LE,RO]{\headerfmtpage{\thepage}}
  \fancyhead[LO]{\headerfmt{\rightmark}}
  \fancyhead[RE]{\headerfmt{\leftmark}}
  \renewcommand{\headrulewidth}{0.5pt}
  \renewcommand{\footrulewidth}{0pt}
}

% anderhalve interlinie (opm: titelblad gaat uit van 1.5)
\renewcommand{\baselinestretch}{1.5}

% indien LaTeX niet goed splitst, neem je woord hierin op, of evt om splitsen 
% te voorkomen
\hyphenation{ditmagnooitgesplitstworden dit-woord-splitst-hier}

\begin{document}

%!!!!!!!!!!!!!!!!!!!!!!!!!!!!!!!!!!!!!!!!!!!!!!!!!!!!!!!!!!!!!!!!!!!!!!!!!!!!!!!!!!!!!!!!!!!!!!!!!
%!!!!!!!!!!!              onderaan/bovenaan elk blad thesistitel zetten                !!!!!!!!!!!
%!!!!!!!!!!!!!!!!!!!!!!!!!!!!!!!!!!!!!!!!!!!!!!!!!!!!!!!!!!!!!!!!!!!!!!!!!!!!!!!!!!!!!!!!!!!!!!!!!

% overzicht/samenvatting
%%  Overzichtsbladzijde met samenvatting

\newpage

{
\setlength{\baselineskip}{32pt}
\setlength{\parindent}{0pt}
\setlength{\parskip}{18pt}

\begin{center}

\noindent \textbf{\huge
Identifying experts through }
\textbf{\huge a framework for knowledge extraction}
\textbf{\huge from public online sources}

\setlength{\baselineskip}{12pt}
\setlength{\parindent}{0pt}
\setlength{\parskip}{12pt}

door 

Simon Buelens, Mattias Putman

Promotors: Prof.~Dr.~Ir.~Filip~De~Turck,~Elena~Tsiporkova~(Sirris),~Tom~Tourw\'{e}~(Sirris)\\
Scriptiebegeleiders: Anna~Hristoskova,~Tim~Wauters\\

Masterproef ingediend tot het behalen van de academische graad van\\
Master in de ingenieurswetenschappen: computerwetenschappen

Academiejaar 2010-2011\\
Faculteit Ingenieurswetenschappen\\
Voorzitter: Prof. Dr. Ir. Dani\"{e}l De Zutter\\
Vakgroep Informatietechnologie\\

\end{center}

\setlength{\baselineskip}{10pt}
\setlength{\parindent}{0pt}
\setlength{\parskip}{10pt}

\renewcommand{\baselinestretch}{1.1} 	% De interlinie afstand wat vergroten.
\small\normalsize                       % Nodig om de baselinestretch goed te krijgen.

\section*{Samenvatting}

Onderzoekers verliezen veel tijd met de zoektocht naar informatie gerelateerd aan hun onderzoeksdomein. Er bestaan bijna geen diensten die toelaten om aan de hand van trefwoorden een overzicht te verkrijgen met experts voor de opgegeven domeinen. Er is onderzoek naar disambiguatie van auteurs, maar deze worden meestal niet in combinatie gebracht met het opzoeken van experten, maar het indelen van publicaties (alhoewel de twee gerelateerd zijn).

In deze masterproef onderzoeken we de mogelijkheid om een framework op te stellen dat dit toelaat door online informatie op te zoeken, deze informatie in relatie te brengen met de correcte auteurs en gebruikers toe te laten dit framework te gebruiken om hierin te zoeken. De nadruk van het framework ligt op de disambiguatie van auteurs (aan de hand van de aanwezige informatie alle namen zo goed mogelijk connecteren met de juist auteur) aan de hand van een regelgebaseerde aanpak en de uitbreidbaarheid van het framework.

We maken gebruik van een graafgebaseerde representatie van de data en de architectuur is gebaseerd op pipes en filters. Dit laat toe dat het framework uitbreidbaar, schaalbaar en eenvoudig aanpasbaar is. Op het einde volgen de resultaten gebaseerd op een manueel geannoteerde testset. Uiteindelijk gaan we ook de vergelijking aan met de verdeling van auteurs door DBLP.

\section*{Trefwoorden}

auteur disambiguatie, data verwerking, clustering, pipes en filters

}

\newpage % strikt noodzakelijk om een header op deze blz. te vermijden


\pagestyle{fancy}
\frontmatter

% hoofdstukken
\mainmatter

\chapter{Conclusion and Future Work}

%Our system only provides a snapshot of the expertise on a given time, it should be better if we could provide an overview in time.

\section{Conclusion}

In this master's thesis we examined the opportunities using the semantic web and data processing. We found out that the possibilities within expert finding and author disambiguation are challenging and can contribute in solving a real-life problem. At the end of this informative investigation, the objective became clear: creating a framework that is able to disambiguate authors and allows users to find experts for a certain subject, while keeping it extensible for future additions.

% Wat is juist zo naief aan de eerste aanpak en waarop is deze precies gebaseerd, in 1 zin

We started with a naive approach for the framework. We focused on what we knew and were comfortable with rather than the problem we had to solve. It lacked both in performance and scalability, while the usage of a relational database would render it unrealizable. Profound inspection of this first version of the framework, however, learned us that in order to create a good one, it would have to amplify the following five foundations:

\begin{enumerate}
	\item All instances are considered different authors until proven otherwise.
	\item No decision is made permanent.
	\item Any information is considered partial information.
	\item A constantly changing input asks for a constantly changing output.
	\item The stream of informaton is endless.
\end{enumerate}

% Iets over graaf en regels

With these foundations in mind, we first created a theoretical model. This model consists of three layers that integrate structural, informational and algorithmic aspects. It deals with the main difficulties of author disambiguation, while minimizing the problem domain. We also explain how to process the (partial) information in order for similarities to be found. 

Rules drive the entire flow in our framework by converting novel facts into similarities that group instances with authors through clustering. The four rules we defined for our framework are:

\begin{enumerate}
	\item \textit{Community Rule} Exploiting the fact that authors often work together with the same co-author.
	\item \textit{Interest Rule} The subjects of publications of the same author are usually located within the same field of research.
	\item \textit{Email Rule} Authors with the same email address, are most likely the same person.
	\item \textit{Affiliation Rule} Authors are more likely to work at one affiliation at a given time.
\end{enumerate}

We implemented this theoretical model with pipes and filters using Ruby, Java, key-value store Redis, Resque for scaling and the Tinkerpop stack for the graph representation. The usage of pipes and filters allows for modifiability. New pipes can easily be plugged into the flow which enables the convenient addition of extra information sources or new rules while the performance and scalability remain intact.

Clustering is the process that is responsible for grouping of authors based on the calculated similarities, a key component of our framework. We implemented the dynamic clustering algorithm proposed in \cite{saha2006dynamic} which is based on minimum-cut trees. The quality of the clusters is guaranteed to be the same as its static counterpart, while it is a lot faster as it considers much smaller subgraphs. We added an extra case to the proposed algorithm, exploiting the fact that our framework often has a succession of small similarities connecting the same clusters. This extra case permits the most resource-intensive case to be postponed and executed less frequently.

Finding experts using a set of keywords is easily attainable within our framework, though we do not provide a user-friendly interface for inspecting these experts. However, the graph can be directly queried for these keywords and the corresponding authors, ordered by decreasing relevance, can be obtained with a simple traversal.

We tested the framework thoroughly using a manually annotated testset of just over 1000 publications. We showed the impact of each of the rules on the accuracy, clarifying that the combination of all four of them render the best average results, although the increased accuracy from the addition of a rule is minimal in certain cases. We also examined the accuracy using different distributions for the rules and we noticed that the community rule had the biggest impact. Finally, we also compared the accuracy of our framework with the accuracy of DBLP for the tested authors and concluded that our framework transcends DBLP by $14\%$ or $17\%$, depending on how the mean accuracy is calculated.

\section{Future Work}

There are many possible additions to be made to the framework. Adding extra online sources by writing new pipes can help us in the search for more information about each author and give us a better view for disambiguation. Additional, extra rules could be written to translate the newly acquired information to similarities between authors which were previously unknown.

Another improvement is enabling the usage of negative weights for similarities. This could undo previously calculated results, when there is reason to believe two instances are not connected. Related to the addition of using negative weights, would be the possibility to have immutable decisions. Especially if this decision is that two authors can never be the same. A simple reason that comes to mind is that they are both co-authors of the same publication.

Using an actual category tree containing a large overview of categories, could help in calculating more accurate similarities between keywords that are not the same, but are bound to the same area of expertise. This could also enable giving higher weights to keywords that are more specific for a certain subject.

Another addition would be using the abstract or even the actual text of the publication for deciding the category. This could get far better results are publication titles are sometimes vague and open for interpretation. Getting the abstract could be achieved by writing plugins for the most popular digital libriries by using web scraping.

% Rekening houden met negatieve gewichten
% Absolute vaststellingen kunnen maken : deze 2 auteurs zijn ZEKER verschillend en moeten altijd zo blijven
% Opstellen van categorie boom om betere keyword results te krijgen
% Grotere gewichten toekennen voor woorden die meer subject specifiek zijn
% Scalability fixen :o
% Meer bronnen opnemen
% Category bepaling niet enkel op basis van titel, maar ook abstract of zelfs volledige tekst => plugins schrijven voor verschillende paper aanbied sites (zijn er slechts een paar)
% 
\backmatter
\end{document}
