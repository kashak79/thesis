%
%  THESISBOEK
%
%  Dit bestand zorgt voor algemene (layout)definities, en groepeert de
%  afzonderlijke LaTeX-files tot een geheel.
%
%  @author Erwin Six, David De Reu, Brecht Vermeulen
%

\documentclass[11pt,a4paper,oneside,notitlepage]{book}
\usepackage[english]{babel}
\usepackage{algorithmic}
\usepackage{algorithm}
\usepackage{amsthm}
\usepackage{hyperref}
\usepackage{array}
%\usepackage[nottoc]{tocbibind} % Bibliografie in ToC; zie tocbibind.dvi

% marges aanpassen
% (opmerking: moet *voor* inclusie van fancyhdr package komen)
\setlength{\hoffset}{-1in}
\setlength{\voffset}{-1in}
\setlength{\topmargin}{2cm}
\setlength{\headheight}{0.5cm}
\setlength{\headsep}{1cm}
\setlength{\oddsidemargin}{3.5cm}
\setlength{\evensidemargin}{3.5cm}
\setlength{\textwidth}{16cm}
\setlength{\textheight}{23.3cm}
\setlength{\footskip}{1.5cm}

\usepackage{fancyhdr}
\usepackage{graphicx}
% \usepackage[colorlinks]{hyperref}
% Het bibliografisch opmaak bestand.
\bibliographystyle{unsrt}
%\bibliographystyle{bibliodutch}
%\bibpunct{[}{]}{,}{n}{,}{,}

\newtheorem{mydef}{Definition}
\newtheorem{foundation}{Foundation}

\pagestyle{fancy}

\renewcommand{\chaptermark}[1]{\markright{\MakeUppercase{#1}}}
\renewcommand{\sectionmark}[1]{\markright{\thesection~#1}}

\newcommand{\headerfmt}[1]{\textsl{\textsf{#1}}}
\newcommand{\headerfmtpage}[1]{\textsf{#1}}

\fancyhf{}
\fancyhead[LE,RO]{\headerfmtpage{\thepage}}
\fancyhead[LO]{\headerfmt{\rightmark}}
\fancyhead[RE]{\headerfmt{\leftmark}}
\renewcommand{\headrulewidth}{0.5pt}
\renewcommand{\footrulewidth}{0pt}

\fancypagestyle{plain}{ % eerste bladzijde van een hoofdstuk
  \fancyhf{}
  \fancyhead[LE,RO]{\headerfmtpage{\thepage}}
  \fancyhead[LO]{\headerfmt{\rightmark}}
  \fancyhead[RE]{\headerfmt{\leftmark}}
  \renewcommand{\headrulewidth}{0.5pt}
  \renewcommand{\footrulewidth}{0pt}
}

% anderhalve interlinie (opm: titelblad gaat uit van 1.5)
\renewcommand{\baselinestretch}{1.5}

% indien LaTeX niet goed splitst, neem je woord hierin op, of evt om splitsen 
% te voorkomen
\hyphenation{ditmagnooitgesplitstworden dit-woord-splitst-hier}

\begin{document}

%!!!!!!!!!!!!!!!!!!!!!!!!!!!!!!!!!!!!!!!!!!!!!!!!!!!!!!!!!!!!!!!!!!!!!!!!!!!!!!!!!!!!!!!!!!!!!!!!!
%!!!!!!!!!!!              onderaan/bovenaan elk blad thesistitel zetten                !!!!!!!!!!!
%!!!!!!!!!!!!!!!!!!!!!!!!!!!!!!!!!!!!!!!!!!!!!!!!!!!!!!!!!!!!!!!!!!!!!!!!!!!!!!!!!!!!!!!!!!!!!!!!!

% overzicht/samenvatting
%%  Overzichtsbladzijde met samenvatting

\newpage

{
\setlength{\baselineskip}{32pt}
\setlength{\parindent}{0pt}
\setlength{\parskip}{18pt}

\begin{center}

\noindent \textbf{\huge
Identifying experts through }
\textbf{\huge a framework for knowledge extraction}
\textbf{\huge from public online sources}

\setlength{\baselineskip}{12pt}
\setlength{\parindent}{0pt}
\setlength{\parskip}{12pt}

door 

Simon Buelens, Mattias Putman

Promotors: Prof.~Dr.~Ir.~Filip~De~Turck,~Elena~Tsiporkova~(Sirris),~Tom~Tourw\'{e}~(Sirris)\\
Scriptiebegeleiders: Anna~Hristoskova,~Tim~Wauters\\

Masterproef ingediend tot het behalen van de academische graad van\\
Master in de ingenieurswetenschappen: computerwetenschappen

Academiejaar 2010-2011\\
Faculteit Ingenieurswetenschappen\\
Voorzitter: Prof. Dr. Ir. Dani\"{e}l De Zutter\\
Vakgroep Informatietechnologie\\

\end{center}

\setlength{\baselineskip}{10pt}
\setlength{\parindent}{0pt}
\setlength{\parskip}{10pt}

\renewcommand{\baselinestretch}{1.1} 	% De interlinie afstand wat vergroten.
\small\normalsize                       % Nodig om de baselinestretch goed te krijgen.

\section*{Samenvatting}

Onderzoekers verliezen veel tijd met de zoektocht naar informatie gerelateerd aan hun onderzoeksdomein. Er bestaan bijna geen diensten die toelaten om aan de hand van trefwoorden een overzicht te verkrijgen met experts voor de opgegeven domeinen. Er is onderzoek naar disambiguatie van auteurs, maar deze worden meestal niet in combinatie gebracht met het opzoeken van experten, maar het indelen van publicaties (alhoewel de twee gerelateerd zijn).

In deze masterproef onderzoeken we de mogelijkheid om een framework op te stellen dat dit toelaat door online informatie op te zoeken, deze informatie in relatie te brengen met de correcte auteurs en gebruikers toe te laten dit framework te gebruiken om hierin te zoeken. De nadruk van het framework ligt op de disambiguatie van auteurs (aan de hand van de aanwezige informatie alle namen zo goed mogelijk connecteren met de juist auteur) aan de hand van een regelgebaseerde aanpak en de uitbreidbaarheid van het framework.

We maken gebruik van een graafgebaseerde representatie van de data en de architectuur is gebaseerd op pipes en filters. Dit laat toe dat het framework uitbreidbaar, schaalbaar en eenvoudig aanpasbaar is. Op het einde volgen de resultaten gebaseerd op een manueel geannoteerde testset. Uiteindelijk gaan we ook de vergelijking aan met de verdeling van auteurs door DBLP.

\section*{Trefwoorden}

auteur disambiguatie, data verwerking, clustering, pipes en filters

}

\newpage % strikt noodzakelijk om een header op deze blz. te vermijden


\pagestyle{fancy}
\frontmatter

\setlength{\parindent}{0pt}
\setlength{\parskip}{0.5\baselineskip plus 0.5ex minus 0.2ex}
\setlength{\parskip}{1ex plus 0.5ex minus 0.2ex}

% hoofdstukken
\mainmatter

\chapter{Introduction}

Researchers are spending more than half of their total research and development hours on the hunt for information, acccording to the US NSF (National Science Foundation). If we could limit the time necessairy for the search for information, researchers would have more time to spend on their research and development, the main point of focus.

Leading search engines mainly provide keyword-based results in response of a search query. This is both limited in terms of accuracy and efficiency of information comprehension. Researchers still have to bend over backwards in order to find more information about authors, their level of expertise and their connections. A new type of information service is required which is focussed on this problem. It should search the desired information and connect, combine and analyze it in the greater picture of the semantic available information on the internet in order to provide as much value to the user as possible.

% Doelstelling

We want to help in this upcoming research by creating a framework that can retrieve experts for any given subject matter. The end result should allow anyone to query the framework with a set of keywords defining the subject area they want to investigate. The outcome of this query is a list of authors, ranked by decreasing level of expertise defined by the dictated keywords, together with a list of papers, highly touted co-authors and any other information the user might find useful.

% TODO bespreek onze werkmethode op een zodanige manier dat een 'expert' (bv. medestudent) door de beschrijving verstaat waarom we deze bepaalde aanpak gekozen hebben

% TODO bespreek hoe de hoofdstukken in elkaar zitten (overzicht)
\bibliography{collection}
\backmatter
\end{document}
