\chapter{The framework}

The goal is to build a platform that allows a user to query for experts in a particular domain. The
platform extracts and unifies the required information from a variety of online sources and
subsequently builds a repository of user profiles.

We basically want to create a framework that would function as a Google for finding experts given a certain subject.

Describe how we got to the architecture for the framework, show a figure of the (simplified + extensive) architecture and explain the different components.

\section{Features}

\subsection{Information extraction}

We want to create a framework that can search different online sources autonomously and extract information. We 


\subsection{Information linkage}



\subsection{Pitfalls}

Describe the possible difficulties: people with the same name that are a different person, a person who's name is written differently (usage of abbreviations, altering last name due to marriage...), change of expertise due to change of interest/job.

\section{Components / Architecture}

Plugins, IterationManager, CategoryBuilder, Disambiguator, DataMerger

\section{Technologies}

\subsection{MySQL}

Discuss the differences, advantages and disadvantages between a relational datastore, a record-store and a triple-store.

\subsubsection{Relational datastore}

\subsubsection{A record-store}

\subsubsection{Triple-store}

\subsubsection{Conclusion}

We choose to use MySQL, a relational datastore, because we are familiar with it. It allows us to quickly set up our prototype to work and test our framework on.

\subsection{Two databases}

We use two databases, one for development and one for production.

The former is used to collect all information about the possible experts and their work. 

\subsection{Ruby}
