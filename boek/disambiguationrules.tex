\section{Disambiguation rules}

Per regel zouden we een soort van gewicht moeten meegeven die genormaliseerd is over alle regels heen. Hier moeten we nog over nadenken.

\subsection{Prequel - Doc split}

In order to enlarge our disambiguation possibilities, we want access to the actual text of the publications. Using a search engine, like Google Scholar, we can find most of the actual publications in pdf on the internet. By using the Ruby library Doc split \footnote{http://documentcloud.github.com/docsplit/}, we can easily parse the pdf into computer-understandable text. Most of the time it will be enough to just get the first page, giving us access to the authors, their affiliations and location. The first page also almost always has an abstract and a list of keywords.

In the following rules we will make a distinction between those who rely on this extra information and those who don't. The rule who rely on it will be followed by an asterisk *.

\subsection{Place of publication rule*}

This rule defines that is less likely to have multiple persons with the exact same name writing a publication in the same city.

\subsection{Co-author clustering}

Authors often work together with the same people, writing multiple publications together. If we find clusters \footnote{how do we formally define clusters and where do we explain this concept? Should this be in the sota related towards the publication 'Clustering using min cut tree'?} of co-authors which are completely indepent, it enlarges the chance that name we are investigating is related to multiple authors.

We can also use this rule to enlarge the possibility of author names being related by using this rule the other 