\chapter{Introduction}

Researchers are spending a lot of their total research and development hours searching for information. If we could speed up the process of finding the correct information, researchers would have more time to spend on their research and development, the main point of focus.

Leading search engines mainly provide keyword-based results in response of a search query. This is both limited in terms of accuracy and efficiency of information comprehension. Researchers still have to bend over backwards in order to find more information about authors, their level of expertise and their connections. A new type of information service is required which focuses on this problem. It should search the desired information and connect, combine and analyze it in the greater picture of the semantic available information on the Internet in order to provide as much value to the user as possible.

% Doelstelling

\section{Thesis objective and approach}

We want to help in this upcoming research by creating a framework that can retrieve experts for any given subject matter. The end result should allow anyone to query the framework with a set of keywords defining the subject area they want to investigate. The outcome of this query is a list of authors, ranked by decreasing level of expertise defined by the dictated keywords. Each author is accompanied by a profile, containing a list of papers, highly touted co-authors and any other information the user might find useful.

We split the internal functioning of this framework in three main components: retrieving information from various online sources (publications, author profiles or online presentations), analyzing this information and linking it to a specific author (a process we call clustering) and defining the areas of interest of each author and their level of expertise for each of these areas.

For the first component, we limited ourselves to using DBLP \cite{dblp} as online source. This is a service providing bibliographic information on major computer science journals and proceedings. We limit ourselves to this one source as the information is comparable to other listings, while they provide an XML overview for each publication allowing us to parse it with ease. It is still sufficiently challenging, as publications are often attributed to the wrong author. We considered adding author profiles from LinkedIn, but this often complicated it as profiles were often inaccessible, incomplete or outdated.

In the second component, each publication we retrieve from DBLP is saved as a unique instance and each author is initially considered different from any previous ones. Clustering consists of linking author names to distinct authors. This is the key component for this thesis and is what we focused on the most. We have composed various rules based on name similarity, recurring co-authors, publication subjects, affiliation and email address. These rules are used by the framework to calculate similarities between publications and authors. As more information is obtained, the framework dynamically updates the clusters containing publications from the same author.

The last component is responsible for defining the area of interest of an author. In order to accomplish this, a category tree is created. Each publication is then mapped onto this tree and the combination of all the publications form a subtree which is linked an author's interests. To decide the level of expertise, the number of subject references, the number of publication citations and the level of expertise of co-authors should be combined. However, this is beyond the scope of our thesis and we have simplified this by comparing the subject of the publications extracted from the title and the abstract and the number of publications for each subject.

\section{Chapter outline}

In \autoref{sota}, we start with a summary of existing technologies and techniques that we researched. Afterwards, we describe the different thesis topics we examined and how we could extend existing research. 

\autoref{framework} contains a naive attempt to create the framework. Although the features are accurately construed, the proposed subdivisions and internal operations would never suffice for a real application. In \autoref{foundation}, we explain its shortcomings and clarify the five foundational acknowledgments.

Based on the foundations, a theoretical model is unveiled in \autoref{model}. It is based on a three-layer structure and the four rules that calculate the similarities are also handled in this chapter.

The implementation details of the theoretical model can be found in \autoref{newframework}. The technologies we used to construct this implementation and the drafted pipes and filers are explained here.

\autoref{clustering} contains an in-depth explanation of the clustering process, a key component of our framework, responsible for creating and reordering the collections of authors based on the similarities.

In \autoref{results}, we present our evaluation setup and obtained results.

\autoref{conclusion} contains the conclusion of our master's thesis and proposes future improvements.