%
%  THESISBOEK
%
%  Dit bestand zorgt voor algemene (layout)definities, en groepeert de
%  afzonderlijke LaTeX-files tot een geheel.
%
%  @author Erwin Six, David De Reu, Brecht Vermeulen
%

\documentclass[11pt,a4paper,oneside,notitlepage]{book}
\usepackage[english]{babel}
\usepackage{algorithmic}
\usepackage{algorithm}
\usepackage{amsthm}
\usepackage{hyperref}
\usepackage{array}
%\usepackage[nottoc]{tocbibind} % Bibliografie in ToC; zie tocbibind.dvi

% marges aanpassen
% (opmerking: moet *voor* inclusie van fancyhdr package komen)
\setlength{\hoffset}{-1in}
\setlength{\voffset}{-1in}
\setlength{\topmargin}{2cm}
\setlength{\headheight}{0.5cm}
\setlength{\headsep}{1cm}
\setlength{\oddsidemargin}{3.5cm}
\setlength{\evensidemargin}{3.5cm}
\setlength{\textwidth}{16cm}
\setlength{\textheight}{23.3cm}
\setlength{\footskip}{1.5cm}

\usepackage{fancyhdr}
\usepackage{graphicx}
% \usepackage[colorlinks]{hyperref}
% Het bibliografisch opmaak bestand.
\bibliographystyle{unsrt}
%\bibliographystyle{bibliodutch}
%\bibpunct{[}{]}{,}{n}{,}{,}

\newtheorem{mydef}{Definition}
\newtheorem{foundation}{Foundation}

\pagestyle{fancy}

\renewcommand{\chaptermark}[1]{\markright{\MakeUppercase{#1}}}
\renewcommand{\sectionmark}[1]{\markright{\thesection~#1}}

\newcommand{\headerfmt}[1]{\textsl{\textsf{#1}}}
\newcommand{\headerfmtpage}[1]{\textsf{#1}}

\fancyhf{}
\fancyhead[LE,RO]{\headerfmtpage{\thepage}}
\fancyhead[LO]{\headerfmt{\rightmark}}
\fancyhead[RE]{\headerfmt{\leftmark}}
\renewcommand{\headrulewidth}{0.5pt}
\renewcommand{\footrulewidth}{0pt}

\fancypagestyle{plain}{ % eerste bladzijde van een hoofdstuk
  \fancyhf{}
  \fancyhead[LE,RO]{\headerfmtpage{\thepage}}
  \fancyhead[LO]{\headerfmt{\rightmark}}
  \fancyhead[RE]{\headerfmt{\leftmark}}
  \renewcommand{\headrulewidth}{0.5pt}
  \renewcommand{\footrulewidth}{0pt}
}

% anderhalve interlinie (opm: titelblad gaat uit van 1.5)
\renewcommand{\baselinestretch}{1.5}

% indien LaTeX niet goed splitst, neem je woord hierin op, of evt om splitsen 
% te voorkomen
\hyphenation{ditmagnooitgesplitstworden dit-woord-splitst-hier}

\begin{document}

%!!!!!!!!!!!!!!!!!!!!!!!!!!!!!!!!!!!!!!!!!!!!!!!!!!!!!!!!!!!!!!!!!!!!!!!!!!!!!!!!!!!!!!!!!!!!!!!!!
%!!!!!!!!!!!              onderaan/bovenaan elk blad thesistitel zetten                !!!!!!!!!!!
%!!!!!!!!!!!!!!!!!!!!!!!!!!!!!!!!!!!!!!!!!!!!!!!!!!!!!!!!!!!!!!!!!!!!!!!!!!!!!!!!!!!!!!!!!!!!!!!!!

% overzicht/samenvatting
%%  Overzichtsbladzijde met samenvatting

\newpage

{
\setlength{\baselineskip}{32pt}
\setlength{\parindent}{0pt}
\setlength{\parskip}{18pt}

\begin{center}

\noindent \textbf{\huge
Identifying experts through }
\textbf{\huge a framework for knowledge extraction}
\textbf{\huge from public online sources}

\setlength{\baselineskip}{12pt}
\setlength{\parindent}{0pt}
\setlength{\parskip}{12pt}

door 

Simon Buelens, Mattias Putman

Promotors: Prof.~Dr.~Ir.~Filip~De~Turck,~Elena~Tsiporkova~(Sirris),~Tom~Tourw\'{e}~(Sirris)\\
Scriptiebegeleiders: Anna~Hristoskova,~Tim~Wauters\\

Masterproef ingediend tot het behalen van de academische graad van\\
Master in de ingenieurswetenschappen: computerwetenschappen

Academiejaar 2010-2011\\
Faculteit Ingenieurswetenschappen\\
Voorzitter: Prof. Dr. Ir. Dani\"{e}l De Zutter\\
Vakgroep Informatietechnologie\\

\end{center}

\setlength{\baselineskip}{10pt}
\setlength{\parindent}{0pt}
\setlength{\parskip}{10pt}

\renewcommand{\baselinestretch}{1.1} 	% De interlinie afstand wat vergroten.
\small\normalsize                       % Nodig om de baselinestretch goed te krijgen.

\section*{Samenvatting}

Onderzoekers verliezen veel tijd met de zoektocht naar informatie gerelateerd aan hun onderzoeksdomein. Er bestaan bijna geen diensten die toelaten om aan de hand van trefwoorden een overzicht te verkrijgen met experts voor de opgegeven domeinen. Er is onderzoek naar disambiguatie van auteurs, maar deze worden meestal niet in combinatie gebracht met het opzoeken van experten, maar het indelen van publicaties (alhoewel de twee gerelateerd zijn).

In deze masterproef onderzoeken we de mogelijkheid om een framework op te stellen dat dit toelaat door online informatie op te zoeken, deze informatie in relatie te brengen met de correcte auteurs en gebruikers toe te laten dit framework te gebruiken om hierin te zoeken. De nadruk van het framework ligt op de disambiguatie van auteurs (aan de hand van de aanwezige informatie alle namen zo goed mogelijk connecteren met de juist auteur) aan de hand van een regelgebaseerde aanpak en de uitbreidbaarheid van het framework.

We maken gebruik van een graafgebaseerde representatie van de data en de architectuur is gebaseerd op pipes en filters. Dit laat toe dat het framework uitbreidbaar, schaalbaar en eenvoudig aanpasbaar is. Op het einde volgen de resultaten gebaseerd op een manueel geannoteerde testset. Uiteindelijk gaan we ook de vergelijking aan met de verdeling van auteurs door DBLP.

\section*{Trefwoorden}

auteur disambiguatie, data verwerking, clustering, pipes en filters

}

\newpage % strikt noodzakelijk om een header op deze blz. te vermijden


\pagestyle{fancy}
\frontmatter

\setlength{\parindent}{0pt}
\setlength{\parskip}{0.5\baselineskip plus 0.5ex minus 0.2ex}
\setlength{\parskip}{1ex plus 0.5ex minus 0.2ex}

% hoofdstukken
\mainmatter

\chapter{Introduction}

Researchers are spending a lot of their total research and development hours searching for information. If we could speed up the process of finding the correct information, researchers would have more time to spend on their research and development, the main point of focus.

Leading search engines mainly provide keyword-based results in response of a search query. This is both limited in terms of accuracy and efficiency of information comprehension. Researchers still have to bend over backwards in order to find more information about authors, their level of expertise and their connections. A new type of information service is required which focuses on this problem. It should search the desired information and connect, combine and analyze it in the greater picture of the semantic available information on the internet in order to provide as much value to the user as possible.

% Doelstelling

\section{Thesis objective and approach}

We want to help in this upcoming research by creating a framework that can retrieve experts for any given subject matter. The end result should allow anyone to query the framework with a set of keywords defining the subject area they want to investigate. The outcome of this query is a list of authors, ranked by decreasing level of expertise defined by the dictated keywords. Each author is accompanied by a profile, containing a list of papers, highly touted co-authors and any other information the user might find useful.

We split the internal functioning of this framework in three main components: retrieving information from various online sources (publications, author profiles or online presentations), analyzing this information and linking it to a specific author (a proces we call clustering) and defining the areas of interest of each author and their level of expertise for each of these areas.

For the first component, we limited ourselves to using DBLP \cite{dblp} as online source. This is a service providing bibliographic information on major computer science journals and proceedings. We limit ourselves to this one source as the information is comparable to other listings, while they provide an XML overview for each publication allowing us to parse it with ease. It is still sufficiently challenging, as publications are often attributed to the wrong author. We considered adding author profiles from LinkedIn, but this often complicated it as profiles were often inaccessible, incomplete or outdated.

In the second component, each publication we retrieve from DBLP is saved as a unique instance and each author is initially considered different from any previous ones. Clustering consists of linking authornames to distinct authors. This is the key component for this thesis and is what we focused on the most. We have composed various rules based on name similarity, recurring co-authors, publication subjects, affiliation and email address. These rules are used by the framework to calculate similarities between publications and authors. As more information is obtained, the framework dynamically updates the clusters containing publications from the same author.

The last component is responsible for defining the area of interest of an author. In order to accomplish this, a category tree is created. Each publication is then mapped onto this tree and the combination of all the publications form a subtree which is linked an author's interests. To decide the level of expertise, the number of subject references, the number of publication citations and the level of expertise of co-authors should be combined. However, this is beyond the scope of our thesis and we have simplified this by comparing the subject of the publications extracted from the title and the abstract and the number of publications for each subject.

%To tackly this problem, we split it in two main parts. The first consists of retrieving and analyzing information about authors from various online sources. This includes defining what publications map to what authors, a proces we call clustering and which is the key component of this thesis, and retrieving the subjects of each publication. 
%
%The proces of clustering is the main point of interest of this thesis. Linking publications to authors is a lot harder than just comparing names. Authors may have the exact same name, names can be misspelled and names can be shortened or altered. In \autoref{} we define a set of rules combining attributes of authors and relationships between them. In \autoref{} we test combinations of these rules to find the best outcome.
%
\section{Chapter outline}

The first chapter is this introduction and describes the objective of our thesis and how we want to accomplish this.

In \autoref{sota}, we start with a summary of existing technologies and techniques that we researched. Afterwards, we describe the different thesis topics we examined. For each topic, we discuss the existing research and 

% TODO bespreek onze werkmethode op een zodanige manier dat een 'expert' (bv. medestudent) door de beschrijving verstaat waarom we deze bepaalde aanpak gekozen hebben

% TODO bespreek hoe de hoofdstukken in elkaar zitten (overzicht)
\bibliography{collection}
\backmatter
\end{document}
