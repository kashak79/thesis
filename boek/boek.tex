%
%  THESISBOEK
%
%  Dit bestand zorgt voor algemene (layout)definities, en groepeert de
%  afzonderlijke LaTeX-files tot een geheel.
%
%  @author Erwin Six, David De Reu, Brecht Vermeulen
%

\documentclass[11pt,a4paper,oneside,notitlepage]{book}
\usepackage[english]{babel}


\usepackage{algorithmic}
\usepackage{algorithm}
\usepackage{amsthm}
\usepackage{hyperref}
\usepackage{array}

\usepackage{pdfpages}  % pdf pagina's importeren


%\usepackage[nottoc]{tocbibind} % Bibliografie in ToC; zie tocbibind.dvi

% marges aanpassen
% (opmerking: moet *voor* inclusie van fancyhdr package komen)
\setlength{\hoffset}{-1in}
\setlength{\voffset}{-1in}
\setlength{\topmargin}{2cm}
\setlength{\headheight}{0.5cm}
\setlength{\headsep}{1cm}
\setlength{\oddsidemargin}{3.5cm}
\setlength{\evensidemargin}{3.5cm}
\setlength{\textwidth}{16cm}
\setlength{\textheight}{23.3cm}
\setlength{\footskip}{1.5cm}

\usepackage{fancyhdr}
\usepackage{graphicx}
% \usepackage[colorlinks]{hyperref}
% Het bibliografisch opmaak bestand.
\bibliographystyle{unsrt}
%\bibliographystyle{bibliodutch}
%\bibpunct{[}{]}{,}{n}{,}{,}

\newtheorem{mydef}{Definition}
\newtheorem{foundation}{Foundation}

\pagestyle{fancy}

\renewcommand{\chaptermark}[1]{\markright{\MakeUppercase{#1}}}
\renewcommand{\sectionmark}[1]{\markright{\thesection~#1}}

\newcommand{\headerfmt}[1]{\textsl{\textsf{#1}}}
\newcommand{\headerfmtpage}[1]{\textsf{#1}}

\fancyhf{}
\fancyhead[LE,RO]{\headerfmtpage{\thepage}}
\fancyhead[LO]{\headerfmt{\rightmark}}
\fancyhead[RE]{\headerfmt{\leftmark}}
\renewcommand{\headrulewidth}{0.5pt}
\renewcommand{\footrulewidth}{0pt}

\fancypagestyle{plain}{ % eerste bladzijde van een hoofdstuk
  \fancyhf{}
  \fancyhead[LE,RO]{\headerfmtpage{\thepage}}
  \fancyhead[LO]{\headerfmt{\rightmark}}
  \fancyhead[RE]{\headerfmt{\leftmark}}
  \renewcommand{\headrulewidth}{0.5pt}
  \renewcommand{\footrulewidth}{0pt}
}

% anderhalve interlinie (opm: titelblad gaat uit van 1.5)
\renewcommand{\baselinestretch}{1.5}

% indien LaTeX niet goed splitst, neem je woord hierin op, of evt om splitsen 
% te voorkomen
\hyphenation{ditmagnooitgesplitstworden dit-woord-splitst-hier}

\begin{document}

% voorwoord met dankwoord en toelating tot bruikleen (ondertekend)
%%  Voorwoord (dankwoord) en toelating tot bruikleen

\newpage

\noindent \textbf{\huge Dankwoord}

\vspace{1.5cm}

\noindent

%Met het schrijven van dit voorwoord wordt een punt gezet achter een interessante en leerrijke thesis. Eerst en vooral wil ik mijn promotoren, prof. dr. ir. F. De Turck en prof. dr. ir. B. Dhoedt, bedanken om dit onderwerp mogelijk te maken. 
%\\
%
%Natuurlijk mogen mijn begeleiders, A. Hristoskova, F. Ongenae en J. Nelis hier niet ontbreken. Ik dank hen voor hun bijdrage aan dit werk, de nuttige tips en de aangename vergaderingen.
%\\
%
%Verder wil ik ook B. Jooris en F. Depuydt bedanken die mij hebben geholpen met enkele praktische aspecten van deze thesis. Verder wens ik ook Dimitri en Jan te bedanken voor de hulp en leuke samenwerking tijdens de thesis.
%\\
%
%Ook wil ik ook mijn familie, vriendin en vrienden bedanken voor hun steun en geduld gedurende mijn studies.

\addvspace{4cm}

\noindent Simon Buelens en Mattias Putman, augustus 2011\newpage

\noindent \textbf{\huge Toelating tot bruikleen}

\vspace{1.5cm}

\noindent
``De auteur geeft de toelating deze scriptie voor consultatie beschikbaar
te stellen en delen van de scriptie te kopi\"eren voor persoonlijk
gebruik.\\
Elk ander gebruik valt onder de beperkingen van het auteursrecht,
in het bijzonder met betrekking tot de verplichting de bron uitdrukkelijk
te vermelden bij het aanhalen van resultaten uit deze scriptie.''

\addvspace{4cm}

\noindent Simon Buelens en Mattias Putman, augustus 2011\newpage


%!!!!!!!!!!!!!!!!!!!!!!!!!!!!!!!!!!!!!!!!!!!!!!!!!!!!!!!!!!!!!!!!!!!!!!!!!!!!!!!!!!!!!!!!!!!!!!!!!
%!!!!!!!!!!!              onderaan/bovenaan elk blad thesistitel zetten                !!!!!!!!!!!
%!!!!!!!!!!!!!!!!!!!!!!!!!!!!!!!!!!!!!!!!!!!!!!!!!!!!!!!!!!!!!!!!!!!!!!!!!!!!!!!!!!!!!!!!!!!!!!!!!

% overzicht/samenvatting
%%  Overzichtsbladzijde met samenvatting

\newpage

{
\setlength{\baselineskip}{32pt}
\setlength{\parindent}{0pt}
\setlength{\parskip}{18pt}

\begin{center}

\noindent \textbf{\huge
Identifying experts through }
\textbf{\huge a framework for knowledge extraction}
\textbf{\huge from public online sources}

\setlength{\baselineskip}{12pt}
\setlength{\parindent}{0pt}
\setlength{\parskip}{12pt}

door 

Simon Buelens, Mattias Putman

Promotors: Prof.~Dr.~Ir.~Filip~De~Turck,~Elena~Tsiporkova~(Sirris),~Tom~Tourw\'{e}~(Sirris)\\
Scriptiebegeleiders: Anna~Hristoskova,~Tim~Wauters\\

Masterproef ingediend tot het behalen van de academische graad van\\
Master in de ingenieurswetenschappen: computerwetenschappen

Academiejaar 2010-2011\\
Faculteit Ingenieurswetenschappen\\
Voorzitter: Prof. Dr. Ir. Dani\"{e}l De Zutter\\
Vakgroep Informatietechnologie\\

\end{center}

\setlength{\baselineskip}{10pt}
\setlength{\parindent}{0pt}
\setlength{\parskip}{10pt}

\renewcommand{\baselinestretch}{1.1} 	% De interlinie afstand wat vergroten.
\small\normalsize                       % Nodig om de baselinestretch goed te krijgen.

\section*{Samenvatting}

Onderzoekers verliezen veel tijd met de zoektocht naar informatie gerelateerd aan hun onderzoeksdomein. Er bestaan bijna geen diensten die toelaten om aan de hand van trefwoorden een overzicht te verkrijgen met experts voor de opgegeven domeinen. Er is onderzoek naar disambiguatie van auteurs, maar deze worden meestal niet in combinatie gebracht met het opzoeken van experten, maar het indelen van publicaties (alhoewel de twee gerelateerd zijn).

In deze masterproef onderzoeken we de mogelijkheid om een framework op te stellen dat dit toelaat door online informatie op te zoeken, deze informatie in relatie te brengen met de correcte auteurs en gebruikers toe te laten dit framework te gebruiken om hierin te zoeken. De nadruk van het framework ligt op de disambiguatie van auteurs (aan de hand van de aanwezige informatie alle namen zo goed mogelijk connecteren met de juist auteur) aan de hand van een regelgebaseerde aanpak en de uitbreidbaarheid van het framework.

We maken gebruik van een graafgebaseerde representatie van de data en de architectuur is gebaseerd op pipes en filters. Dit laat toe dat het framework uitbreidbaar, schaalbaar en eenvoudig aanpasbaar is. Op het einde volgen de resultaten gebaseerd op een manueel geannoteerde testset. Uiteindelijk gaan we ook de vergelijking aan met de verdeling van auteurs door DBLP.

\section*{Trefwoorden}

auteur disambiguatie, data verwerking, clustering, pipes en filters

}

\newpage % strikt noodzakelijk om een header op deze blz. te vermijden


\frontmatter


% titelblad (voor kaft)
\setlength{\hoffset}{0in}
\setlength{\voffset}{0in}
%\addcontentsline{toc}{chapter}{Extended abstract}
\includepdf[pages=-]{voorblad.pdf}
\setlength{\hoffset}{-1in}
\setlength{\voffset}{-1in}

% lege pagina (!!)
\newpage
\thispagestyle{empty}
\mbox{}

% titelblad (!!)
\setlength{\hoffset}{0in}
\setlength{\voffset}{0in}
%\addcontentsline{toc}{chapter}{Extended abstract}
\includepdf[pages=-]{voorblad.pdf}
\setlength{\hoffset}{-1in}
\setlength{\voffset}{-1in}

%% geen paginanummering tot we aan de inhoudsopgave komen
\pagestyle{empty}
%
%% voorwoord met dankwoord en toelating tot bruikleen (ondertekend)
%  Voorwoord (dankwoord) en toelating tot bruikleen

\newpage

\noindent \textbf{\huge Dankwoord}

\vspace{1.5cm}

\noindent

%Met het schrijven van dit voorwoord wordt een punt gezet achter een interessante en leerrijke thesis. Eerst en vooral wil ik mijn promotoren, prof. dr. ir. F. De Turck en prof. dr. ir. B. Dhoedt, bedanken om dit onderwerp mogelijk te maken. 
%\\
%
%Natuurlijk mogen mijn begeleiders, A. Hristoskova, F. Ongenae en J. Nelis hier niet ontbreken. Ik dank hen voor hun bijdrage aan dit werk, de nuttige tips en de aangename vergaderingen.
%\\
%
%Verder wil ik ook B. Jooris en F. Depuydt bedanken die mij hebben geholpen met enkele praktische aspecten van deze thesis. Verder wens ik ook Dimitri en Jan te bedanken voor de hulp en leuke samenwerking tijdens de thesis.
%\\
%
%Ook wil ik ook mijn familie, vriendin en vrienden bedanken voor hun steun en geduld gedurende mijn studies.

\addvspace{4cm}

\noindent Simon Buelens en Mattias Putman, augustus 2011\newpage

\noindent \textbf{\huge Toelating tot bruikleen}

\vspace{1.5cm}

\noindent
``De auteur geeft de toelating deze scriptie voor consultatie beschikbaar
te stellen en delen van de scriptie te kopi\"eren voor persoonlijk
gebruik.\\
Elk ander gebruik valt onder de beperkingen van het auteursrecht,
in het bijzonder met betrekking tot de verplichting de bron uitdrukkelijk
te vermelden bij het aanhalen van resultaten uit deze scriptie.''

\addvspace{4cm}

\noindent Simon Buelens en Mattias Putman, augustus 2011\newpage


% overzicht
%%  Overzichtsbladzijde met samenvatting

\newpage

{
\setlength{\baselineskip}{32pt}
\setlength{\parindent}{0pt}
\setlength{\parskip}{18pt}

\begin{center}

\noindent \textbf{\huge
Identifying experts through }
\textbf{\huge a framework for knowledge extraction}
\textbf{\huge from public online sources}

\setlength{\baselineskip}{12pt}
\setlength{\parindent}{0pt}
\setlength{\parskip}{12pt}

door 

Simon Buelens, Mattias Putman

Promotors: Prof.~Dr.~Ir.~Filip~De~Turck,~Elena~Tsiporkova~(Sirris),~Tom~Tourw\'{e}~(Sirris)\\
Scriptiebegeleiders: Anna~Hristoskova,~Tim~Wauters\\

Masterproef ingediend tot het behalen van de academische graad van\\
Master in de ingenieurswetenschappen: computerwetenschappen

Academiejaar 2010-2011\\
Faculteit Ingenieurswetenschappen\\
Voorzitter: Prof. Dr. Ir. Dani\"{e}l De Zutter\\
Vakgroep Informatietechnologie\\

\end{center}

\setlength{\baselineskip}{10pt}
\setlength{\parindent}{0pt}
\setlength{\parskip}{10pt}

\renewcommand{\baselinestretch}{1.1} 	% De interlinie afstand wat vergroten.
\small\normalsize                       % Nodig om de baselinestretch goed te krijgen.

\section*{Samenvatting}

Onderzoekers verliezen veel tijd met de zoektocht naar informatie gerelateerd aan hun onderzoeksdomein. Er bestaan bijna geen diensten die toelaten om aan de hand van trefwoorden een overzicht te verkrijgen met experts voor de opgegeven domeinen. Er is onderzoek naar disambiguatie van auteurs, maar deze worden meestal niet in combinatie gebracht met het opzoeken van experten, maar het indelen van publicaties (alhoewel de twee gerelateerd zijn).

In deze masterproef onderzoeken we de mogelijkheid om een framework op te stellen dat dit toelaat door online informatie op te zoeken, deze informatie in relatie te brengen met de correcte auteurs en gebruikers toe te laten dit framework te gebruiken om hierin te zoeken. De nadruk van het framework ligt op de disambiguatie van auteurs (aan de hand van de aanwezige informatie alle namen zo goed mogelijk connecteren met de juist auteur) aan de hand van een regelgebaseerde aanpak en de uitbreidbaarheid van het framework.

We maken gebruik van een graafgebaseerde representatie van de data en de architectuur is gebaseerd op pipes en filters. Dit laat toe dat het framework uitbreidbaar, schaalbaar en eenvoudig aanpasbaar is. Op het einde volgen de resultaten gebaseerd op een manueel geannoteerde testset. Uiteindelijk gaan we ook de vergelijking aan met de verdeling van auteurs door DBLP.

\section*{Trefwoorden}

auteur disambiguatie, data verwerking, clustering, pipes en filters

}

\newpage % strikt noodzakelijk om een header op deze blz. te vermijden


\pagestyle{empty}
\newpage


%%\setlength{\hoffset}{0in}
%%\setlength{\voffset}{0in}
%%%\addcontentsline{toc}{chapter}{Extended abstract}
%%\includepdf[pages=-]{abstract/abstract.pdf}
%%\setlength{\hoffset}{-1in}
%%\setlength{\voffset}{-1in}
%
%%\newpage
%
%
%% inhoudstafel
\renewcommand{\baselinestretch}{1.08} 	% De interlinie afstand wat verkleinen.
\small\normalsize                       % Nodig om de baselinestretch goed te krijgen.
\tableofcontents
\renewcommand{\baselinestretch}{1.2} 	% De interlinie afstand wat vergroten.
\small\normalsize                       % Nodig om de baselinestretch goed te krijgen.


\pagestyle{fancy}

% inhoudstafel

% opmaak voor het eigenlijke boek; onderstaande lijnen
% weglaten als de eerste regel van een nieuwe alinea moet
% inspringen in plaats van extra tussenruimte
\setlength{\parindent}{0pt}
\setlength{\parskip}{1.0\baselineskip plus 0.5ex minus 0.2ex}
\setlength{\parskip}{1.5ex plus 0.5ex minus 0.2ex}

% hoofdstukken
\mainmatter

% hier worden de hoofdstukken ingevoegd (\includes)
\chapter{Introduction}

Researchers are spending more than half of their total research and development hours on the hunt for information, acccording to the US NSF (National Science Foundation). If we could limit the time necessairy for the search for information, researchers would have more time to spend on their research and development, the main point of focus.

Leading search engines mainly provide keyword-based results in response of a search query. This is both limited in terms of accuracy and efficiency of information comprehension. Researchers still have to bend over backwards in order to find more information about authors, their level of expertise and their connections. A new type of information service is required which is focussed on this problem. It should search the desired information and connect, combine and analyze it in the greater picture of the semantic available information on the internet in order to provide as much value to the user as possible.

% Doelstelling

We want to help in this upcoming research by creating a framework that can retrieve experts for any given subject matter. The end result should allow anyone to query the framework with a set of keywords defining the subject area they want to investigate. The outcome of this query is a list of authors, ranked by decreasing level of expertise defined by the dictated keywords, together with a list of papers, highly touted co-authors and any other information the user might find useful.

% TODO bespreek onze werkmethode op een zodanige manier dat een 'expert' (bv. medestudent) door de beschrijving verstaat waarom we deze bepaalde aanpak gekozen hebben

% TODO bespreek hoe de hoofdstukken in elkaar zitten (overzicht)

\chapter{State of the art}

We examined a broad range of different topics with social media as central subject. It was an evolution starting from simple Twitter-related subjects to a full-fledged problem assignment many people struggle with. 
%The journey towards this problem is interesting enough to devote a full thesis to in itself, but we will spare you the details and cover it in this chapter.

\section{Opinion mining}

\subsection{Introduction}
\label{general - opinion mining}
Opinion mining is part of the general area of \textit{sentiment analysis, opinion extraction or opinion mining and feature-based opinion summarization} from the user-generated content or user-generated media on the Web. The applications are manifold with the most important ones in the area of in business intelligence. 

%Large companies receive thousands of pieces of feedback on a daily basis, both direct as indirect. Examples are online customer reviews, customer feedback, survey responses, social media messages, blogposts and comments. Human processing of such text volumes is prohibitively expensive and close to impossible. The only alternative is automatic extraction of relevant information. Ideally one would like to be able to quickly and cheaply customize a system to provide reasonably accurate sentiment classification for a domain, a brand or a specific product.

\subsection{Current situation}

%Sentiment360, lots of papers and onderzoek, really to much to name and to add something of importance.

Opinion mining has been a hot topic the last 10 years due to the rise of social media such as blogs and social networks. Businesses are looking to automate the process of following up on the conversation about their company image and products and opinion mining can help them take steps toward accomplishing this \footnote{Wright, Alex. "Mining the Web for Feelings, Not Facts", New York Times, 2009-08-23. Retrieved on 2010-11-05}.

One step towards this is accomplished in research. Several universities around the world have research teams focussing on the dynamics of sentiment. There is research focused on creating sentiment summaries to capture an author's opinion about a subject based on a publication \footnote{An exploration of sentiment summarization}, predicting the semantic orientation of adjectectives and combinations of adjectives \footnote{Predicting the Semantic Orientation of Adjectives}, identifying the sources of the opinions rather than the actual sentiment itself \footnote{Identifying Sources of Opinions with Conditional Random Fields and Extraction Patterns}...

An interesting ongoing project is \footnote{http://www.cs.uic.edu/~liub/FBS/sentiment-analysis.html} at the University of Illinois, Chicago, backed by Google and Microsoft Corporation. It is a project working in three areas: 

\begin{itemize}
	\item Mining direct (or regular) opinions. Example: after taking the drug, I got stomach pain.
	\item Mining comparative opinions. Example: Coke tastes better than Pepsi.
	\item Review and opinion spam analysis and detection. An example is detecting of fake reviews.
\end{itemize}

It is particularly interesting as others can follow the status of and updates on the project. There are also references to the opinion lexicon and data sets they used to evaluate their results allowing third parties to easily compare their own work using the same input.

\subsection{Improvements and additions}

%Voorloping nog heel erg dunnetjes, Google Prediction API is het enige dat we aanbrengen en dit is meer iets simpel uitvoeren dan echt een uitdagende opdracht. Beter de uitdaging van Twitter aankaarten en hoe we dit concreet wouden aanpakken.

We are looking into the analysis of social media messages, more specifically Twitter messages. We want to use Google's new service, Google Prediction API, which provides pattern-matching and machine learning capabilities. We can compare the results with one or more of the many papers and see if there is any added value in using this service.

We need a lot of training data in order for the Prediction API to learn likely future outcomes. As this service decides what algorithms it uses, there is a lack of possible research to make an interesting thesis.


%Zeer veel onderzoek en bestaande tools, mogelijke uitbreidingen zijn beperkt en toepassingen ook.

%\subsection{Bijlage}
%http://code.google.com/intl/nl-NL/apis/predict/

\section{Twitter influencers}

\subsection{Introduction}
As discussed in \ref{general - opinion mining} about opinion mining, people talk about products, both positive as negative. They have the ability to influence the buying behavior of others who respect their opinion about a certain area. Identifying these influencers can be of great value, for instance in the advertising industry.

There are two main aspects, reach and trust. A person's reach determines the number of people who listen when he has something of value to say. Trust or influence depicts the value people give a certain person's opinion. Both aspects can vary a lot when comparing different topics for the same person.

\subsection{Current situation}

There are papers discussing algorithms to find the ideal subset of individuals which will trigger a large cascade of conversions and papers researching the more general economic issues regarding influencers.

Also regarding the second aspect, trust, there are a lot of well documented scientific results. \footnote{Propagation Models for Trust and Distrust in Social Networks} describes propagation models which can be used to present trust and distrust in social networks. Just as in \footnote{Inferring Trust Relationships in Web-based Social Networks} an algorithm for calculating a trust metric is presented based on the EigenTrust algorithm\footnote{The eigentrust algorithm for reputation management in p2p networks, Kamvar 2003}.

The amount of applications focusing on calculating social media influence, as their core business or a useful option, is growing rapidly. Following is a short summary a few better applications.

Klout is a well-known online tool focusing on ranking Twitter profiles (and recently also Facebook profiles) using over 35 variables. These rankings are, though fun, not particularly useful as finding influencers based on topic is very limited, however there are interesting third-party applications providing this feature.

PeerIndex and Traackr focus more on identifying topics. Just like Klout, PeerIndex focuses on Twitter profiles and parameters captured from Twitter to assign a certain score to each profile. Traackr on the other hand uses data from multiple online sources connected to the users such as their blog, YouTube channel, LinkedIn and Twitter profile and more. Traackr combines the results and provides a three-way score (reach, relevance and resonance) based on a certain topic together with detailed contact information.

\subsection{Improvements and additions}

%We kunnen hier nog zo veel meer over vertellen, dit is eigelijk de voornaamste voorloper van ons echt onderwerp, dus hier kunnen we reeds heel wat belangrijke dingen aankaarten die we uiteindelijk ook echt zullen gebruiken. 
%Belangrijke dingen die nog aangekaart moeten worden: Semantic Web (Web of Data), DBPedia, RDF, OWL, FOAF, Google Social Graph, location based?, 

We want to start from a Twitter profile and find out more about a user using other social media networks, semantic databases and social graphs, comparable to Traackr. 

%\section{Event detection}
%
%\subsection{General problem statement}
%
%
%\subsection{Current situation}
%
%\subsection{Improvements and additions}


\section{Expert Finding}

\subsection{Introduction}

Finding experts is useful for seeking consultants, collaborators or speakers. It is also of great value within the academic world as it provides a source of information to supplement or complement papers and theses. Many researchers and reporters lose a lot of time doing this manually as the amount of sources is ever growing: documents, email, databases, conferences, scientific papers and so on. The topic is luckily seeing an increase in attention in recent years.

%Expert finding is a difficult task requiring a multitude of different steps in order to find results bearing certain value. There are databases containing records of people matched with their area of expertise, which can be queried for a nominal fee, mostly used by reporters. The data has been gathered manually over time. To receive a more up-to-date or specific result, one will have to send a separate request which will come with a higher price and take its time.\\
%Another option is to do it manually. There are several steps which can be undertaken and they can be executed in fashionable order, repeated as often as necessary. One can search for conference sites about the topic and look up the names of the speakers. Current and past working experience can possibly be found on LinkedIn. Publications and co-authors can be retrieved from online databases with dedicated research publications or from more general databases such as Google Scholar. The authors and papers can be interlinked in order to see who works with who and who is most often cited or referenced. Influence can be measured on social media.\\
%It is clearly a lengthy process.

\subsection{Current situation}

\footnote{Balog, K. and Rijke, M.: Finding Experts and Their Details in E-mail Corpora. In Proceedings
of the 15th International Conference on World Wide Web (2006)} proposed four simple binary association methods to find expertise information from emails. \footnote{5} investigated the expertise of users and experts by combining information retrieval techniques. Both these solutions are insufficient for topic-based expert finding as their datasets (emails and online communities) are too limited, they focus too much on previous encounters and lack context.

\footnote{6} retrieves experts based on the amount of documents persons have for a given topic. As input a keyword phrase is used in order to find relevant documents. The results are however unsatisfactory caused by its slow response time and incorrect relationship between persons and documents. \footnote{8} gets better results using an algorithm based on a PageRank for document authority, a co-occurrence model for authors and multiple levels of associations between experts and topics. It succeeds to map variants of experts' names on the same author, but fails to identify different authors with the same name.

\footnote{Finding Topic-centric Identified Experts based on Full Text Analysis} proposes OntoFrame. It is an information service platforum using Semantic Web technologies and is based on an extensive ontology of 16 classes using RDF triples. Identity resolution and full text analysis forms the basis of their expert-finding method. The framework looks promising, however the prototype does not function \footnote{$http://ontoframe.kr/2008/2008_new/main.jsp$}.

\subsection{Improvements and additions}

%Iets vermelden over de connectie met het TWIRL project?

We want to continue the research proposed in \footnote{Finding Topic-centric ...}, but less focused towards the full text analysis of documents. We want to create a platform that extracts and unifies the required information from a variety of online sources and subsequently builds a repository of user profiles.

%Te bespreken: neo4j, Tinkerpop, Gremlin, Stanford POS tagger, lemmatizer, doc split (ruby gem), resque (ruby library - https://github.com/defunkt/resque), max flow :: min-cut

\section{Technologies}

This is a temporary section, containing a list of keywords and technologies we use in our thesis. Each of these is explained (or should be), but they should be merged into the previous text.

%\subsection{What do we need}

%%%%%%%%%%%%%%%%%%%%%%%%%%%%%%%%%%%%%%%%%%%%%%%%%%%%%%%%%%

%% Several components we still have to explain or discuss inside the text %%

\subsection{Web scraping}

Web scraping (also called Web harvesting or Web data extraction) is a computer software technique of extracting information from websites. Usually, such software programs simulate human exploration of the Web by implementing low-level Hypertext Transfer Protocol (HTTP). Web scraping focuses on the transformation of unstructured data on the Web, typically in HTML format, into structured data that can be stored and analyzed in a central local database or spreadsheet. Web scraping is also related to Web automation, which simulates human Web browsing using computer software \footnote{http://en.wikipedia.org/wiki/Web scraping}.

We will use scraping in several of our plugins to collect information about authors and publications as we don't have a large database with this information and constructing one ourselves would take too much time and resources. The downside is that it is typically a slow process as the content of each page has to be downloaded and parsed by the computer.

\subsection{API}

%% TODO : verzetten! Deze opsomming van sources zou beter vermeld worden bij plugins, deze zijn niet echt nodig in dit hoofdstuk

%\subsection{DBLP}
%
%DBLP is produced by the Computer Science department of the University of Trier and was initially focused on \textit{DataBase systems and Logic Programming}, but has gradually expanded toward being an confidential server providing bibliographic information on major computer science journals and proceedings, indexing more than one million articles.
%DBLP allows to search by author name, giving back a list of publications and other bibliographic information.
%
%DBLP does not provide us with an API, so we will use web scraping in order to extract all the necessairy information.
%
%%Link naar de volledige uitleg van de werking van de plugin voor meer informatie hieromtrent
%
%\subsection{LinkedIn}
%
%LinkedIn is a business-related social networking site launched in May 2003, mainly used for professional networking by more than 100 million registered users \footnote{http://en.wikipedia.org/wiki/LinkedIn}. Each user has a profile which may contain the following information: current affiliation and title, past working experiences, education, specialties, location, connections with other users...
%
%Using the LinkedIn API we can search for users by entering a name. This will give back list of profiles we can browse and whose information we can extract and use in our framework. As the service is based on a \textit{gated-access approach}, which means you need to be at least a second level connection of the profile you are looking at to see their connections, we can not use this to connect authors to eachother. However, the information provided by the profile which is publicly available does allow us to get a better insight into the subject the person is interested in and what affiliation he is and was connected to.
%
%%Link naar de volledige uitleg van de werking van de plugin voor meer informatie hieromtrent
%
%\subsection{Google Scholar}
%
%Google Scholar is a freely accessible web search engine that indexes the full text of scholarly literature accross an array of publishing formats and disciplines. It allows to search publications based on subject keywords, partial publication titles and author names. It's ranking algorithm uses a combination of factors, but puts mainly high weight to citation count and the words included in a document's title.
%
%Google Scholar is not yet available to the Google AJAX API and Google does not allow automatic crawling or scraping of its services. Thus, it can not be used as a publication reference in our framework.
%
%\subsection{Microsoft Academic Search}
%
%Microsoft Academic Search is a free search engine for academic papers and resources principally in the field of computer science, developed by Microsoft Research Asia, Beijing. The database consists of the bibliographic information (metadata) for academic papers published in journals, conferences proceedings and the citations between them. Objects are ranked according to two factors: their relevance to the query, which is computed by its attributes; and their global importance, calculated by its relationships with other objects \footnote{http://en.wikipedia.org/wiki/Microsoft Academic Search}.
%
%It is a direct competitor of Google Scholar and it allows us to scrape the information of the site, making it useful to extract publications written by a specific author, his fields of interest, the amount of times he was cited and his co-authors for our framework.

\subsection{OWL}

\subsection{FOAF}

FOAF is a descriptive vocabulary expressed using the Resource Description Framework (RDF) and the Web Ontology Language (OWL). Computers may use these FOAF profiles to find, for example, all people living in Europe, or to list all people both you and a friend of yours know. This is accomplished by defining relationships between people. Each profile has a unique identifier (such as the person's e-mail addresses, a Jabber ID, or a URI of the homepage or weblog of the person), which is used when defining these relationships \footnote{http://en.wikipedia.org/wiki/FOAF (software)}.

% Meer uitleggen wat wij hier mee zijn en hoe dit gebruikt kan worden.

\subsection{Stemming}

In linguistic morphology, stemming is the process for reducing inflected (or sometimes derived) words to their stem, base or root form � generally a written word form. The stem need not be identical to the morphological root of the word; it is usually sufficient that related words map to the same stem, even if this stem is not in itself a valid root. 

Algorithms for stemming have been studied in computer science since 1968. The Porter stemming algorithm \footnote{M.F. Porter, 1980, An algorithm for suffix stripping, Program, 14(3) pp 130-137.} dates back to 1979, but is very widely used and the de-facto standard for the English language.

Closely related to stemming is lemmatisation, the algorithmic process of determining the lemma for a given word. The difference is that a stemmer operates on a single word, while a lemmatiser also has knowledge of the context. Stemmers typically run faster and the reduced accuracy may not matter for some applications.

\subsection{Clustering}

Cluster analysis or clustering is the assignment of a set of observations into subsets (called clusters) so that observations in the same cluster are similar in some sense. Clustering is a method of unsupervised learning, and a common technique for statistical data analysis used in many fields, including machine learning, data mining, pattern recognition, image analysis, information retrieval, and bioinformatics \footnote{http://en.wikipedia.org/wiki/Cluster analysis}.

There are a lot of different graph clustering algorithms. Algorithms based on spectral clustering \footnote{R. Kannan, S. Vempala, and A. Veta. On clusterings-good,
bad and spectral. In FOCS �00:, page 367, 2000.}, multilevel graph partitioning schemes like METIS \footnote{G. Karypis and V. Kumar. A Fast and High Quality Multilevel
Scheme for Partitioning Irregular Graphs. Technical Report
95-035, University of Minnesota, June 1995.} and the MLKM algorithm \footnote{Inderjit S. Dhillon, Yuqiang Guan, and Brian Kulis. A fast
kernel-based multilevel algorithm for graph clustering. In
KDD�05, pages 629�634, 2005.}. 

More recently there has been a new direction to graph clustering, modeling the minimum-cut, maximum-flow problem of the underlying graph \footnote{Inderjit S. Dhillon, Yuqiang Guan, and Brian Kulis. A fast
kernel-based multilevel algorithm for graph clustering. In
KDD�05, pages 629�634, 2005.} \footnote{G. W. Flake, R. E. Tarjan, and K. Tsioutsiouliklis. Graph
clustering and minimum cut trees, Internet Mathematics, 1(3),
355-378, 2004.}. The algorithm we will use is the dynamic version of \footnote{Dynamic Algorithm for Graph Clustering Using Minimum Cut Tree}. It is a dynamic algorithm based on the minimum-cut tree problem. It produces a high quality of clusters without having to look at the entire graph but only a subset of nodes, severely reducing the time of the algorithm.


\subsection{Machine learning}

\subsection{Forward Chaining}

\subsection{String matching}

\subsection{Stanford Part-Of-Speech Tagger}

%http://nlp.stanford.edu/software/tagger.shtml

A Part-Of-Speech Tagger, or POS Tagger, is a piece of software that reads text in some language and assigns parts of speech to each word (and other token), such as noun, verb, adjective, etc., although generally computational applications use more fine-grained POS tags like 'noun-plural'. 

The Stanford POS Tagger is Java implementation of the log-linear part-of-speech taggers described in the papers \footnote{Kristina Toutanova and Christopher D. Manning. 2000. Enriching the Knowledge Sources Used in a Maximum Entropy Part-of-Speech Tagger. In Proceedings of the Joint SIGDAT Conference on Empirical Methods in Natural Language Processing and Very Large Corpora (EMNLP/VLC-2000), pp. 63-70.} and \footnote{Kristina Toutanova, Dan Klein, Christopher Manning, and Yoram Singer. 2003. Feature-Rich Part-of-Speech Tagging with a Cyclic Dependency Network. In Proceedings of HLT-NAACL 2003, pp. 252-259.}.

\subsection{Gremlin}

\subsection{JUNG}

\subsection{Neo4j}

\chapter{The framework}
\label{framework}

\section{Main Scenario}

The goal is to build a platform that allows a user to query for experts in a particular domain. The platform extracts and unifies the required information from a variety of online sources and subsequently builds a repository of user profiles. We basically want to create a framework that would function as a Google for finding experts given a certain subject.

We will use the following steps to build such a platform:

%verduidelijken dat de volgorde van deze stappen niet belangrijk is, ze zijn interchangeable

\begin{itemize}
	%Welke auteur namen ? Waar halen we die vandaan ? Conference sites ?
	\item \textbf{Seed data} We need a list of author names to start with.
	\item \textbf{Information extraction}
		\begin{itemize}
			\item Looking up personal information, for instance profiles from LinkedIn.
			\item Looking up publications, extracting title, co-authors and affiliation. The co-authors can be used as new input for further information extraction.
			\item Categorize publications by subject
		\end{itemize}
	\item \textbf{Disambiguation} Each name is represented as a single entity. In this step we search for similarities between these entities to decide which publications belong to the same author.
\end{itemize}

\section{Features}

Based on the main scenario, we can identify a few key features our framework will need. A short summarization of each follows, explaining the challenges we face with our thesis.

\subsection{Information extraction}

We need to extract information from multiple sources, which we will accomplish by using a plugin system. Each plugin is responsible for one source and will extract specific information which can be used by other plugins or other steps in our framework.

\subsection{Categorization}

In order to decide who is an expert regarding a certain subject, we will have to decide the subject of expertise of each author. A very important part of this will be in deciding the subject of the publications. The challenge is to decide this using as few information as possible, preferably by just inspecting the title, as getting access to the text of the publication itself is a whole lot more time- and resource-consuming.

\subsection{Disambiguation}

The most challenging feature is the disambiguation. There are two reasons. The first is the fact that an author's name is not a unique reference for a person. There might be multiple authors with the same name, which means we have to take this into account when deciding who is an expert. Secondly, a name might be spelled differently or changed throughout time. Examples are abbreviations, an extra family name because of a marriage or simply spelling errors.

\subsection{Dynamic}

There is a big dynamic concept tied to our thesis. People who are experts on a certain subject a decade ago, may not be as important anymore now or may have changed their subject of expertise. The framework should allow new information to be processed at any time and update the expertise of authors on the flow.

\section{Architecture}

Based on the scenario and the described features, we came up with the necessary components our framework needs. \autoref{fig:architectuur} shows a simplistic version of how the architecture will look like. The different components of the framework and their connection toward each other is displayed. In the next sections, the most important components will be explained at large.

\begin{figure}[htb]
	\centering
		\includegraphics[width=0.75\textwidth]{fig/architectuur}
	\caption{First attempt at the architecture based on the necessary components}
	\label{fig:architectuur}
\end{figure}

\subsection{Iteration Manager}

This component is responsible for the coordination of the flow in the entire framework. It knows what iteration our program is at and knows what the next steps are. It calls the necessary components and hands it the correct parameters in order for it to execute its code. This will become clearer when we explain the next components.

\subsection{Data Collector}

The Data Collector receives sources from the Source Generator. Each source describes the job to collect a particular piece of information. It will call each of its plugins with the parameters received from the Iteration Manager, updating the database with the new data the plugins will generate.

\subsection{Source Generator}

At each iteration step, the source generator examines the current data set and decides which paths have to be investigated in terms of sources.

\subsection{Category Builder}
\label{categorybuilder}

The Category Builder's role is deciding the subject(s) of the publications and assigning it to a category which fits in a category tree.

We have two main possibilities to decide the subject of a publication. The first is by only focusing on the publication title, the second is by also making use of the actual text in the publication. The second will yield us with better results, but getting the text and analyzing it will take a lot more time than just analyzing a few words. 

For both ways in deciding the subject, we will make use of the Stanford Part-of-speech Tagger. We use it to select the adjectives and nouns, as these contain the most interesting information regarding to deciding the field of the publication.

When only having the publication title, the words marked by the Stanford POS Tagger will be searched on Wikipedia, by use of the API. We will search for the words separately and make combinations, based on the closeness of the words in their original context.

When we have the original text of the publication, the easiest method is scanning for the keywords. These are often cited after the abstract and give us a very quick enumeration of the subjects discussed in the article. Another possibility is parsing the abstract itself by using the same algorithm explained above, used on the publication title.


\subsection{Disambiguator}
\label{disambiguator}

The disambiguation is an important factor towards the strength of our framework. In the data collection phase, we save each name as a new object in our database, even if that name is already stored. This is necessary to keep the option open that the same name might be connected towards different authors. This also means our database grows fast. 

The disambiguation exists out of a number of rules. These inspect several objects in the database and define the probabilities that names, typically connected to a publication or a profile, are connected to each other and to one author.

\subsection{Data Merger}

The Data Merger is responsible for merging the names together from time to time so they would reference to the same author. This component uses the probabilities calculated by the Disambiguator in order to do so.

\subsection{Data Layer and Database}



\chapter{Problem Insights and Foundation}

In the first framework, many properties of our environment were not taken into account. We did not think enough about the consequences of the characteristics of the disambiguation problem. Therefore, we went back on our steps and formulated five basic foundation acknowledgement. This new foundation has a lot of contradictions with the design of our first framework. We decided to abandon the framework. In the next chapter we build an entire theoretical model based on this foundation. The chapter after that discusses the practical realisation of this model.

\section{Foundation Acknowledgements}
\label{foundation}

Below, we list the five foundation acknowledgement that drive the development of a theoretical model.

\begin{foundation}
\label{foundation:different}
All instances are considered different authors until proven otherwise.
\end{foundation}

First, we want a solution that provides precision. The most existing services do simple grouping on equal names. This causes the service to return all the publications for the person that we are searching, but also a lot of papers that belong elsewhere. In other words, they have a high recall, but a low precision. Reaching a high recall is much less difficult than obtaining a high precision while still having a relatively high recall. To obtain this high precision, we only decide that two authors are actually the same if enough evidence is available. When a piece of information on an author with a certain name is collected (we call this an instance of an author), we always consider this information to be an author on its own. This way, we do not make unsupported decisions.

\begin{foundation}
\label{foundation:decision}
No decision is made permanent.
\end{foundation}

We operate in a highly dynamic environment, subject to constant change. This change is controlled from two sides. The first is the one of information itself. As time goes on, more and more information is produced (authors write new publications) and more of this information becomes available on the web. The second is the one of our framework. Its capabilities will constantly improve, making it possible to collect information that was not collectable before. The decision to which physical author information belongs to can change as new information becomes available. Therefore it is wise to not base a decision only on past information. Future information can provide new motives to make another decision. It has to be possible to change our minds though time.

\begin{foundation}
\label{foundation:partial}
Any information is considered partial information.
\end{foundation}

No information can be considered complete and correct. We will substantiate this rule with an example. Suppose that a publication document is scanned for author names. Two things can go wrong here. Fist, the information in the paper can either be incomplete or plain wrong. The publication could accidentally not mention a writer or his name could be misspelled. Secondly, a case that is not recognized by the framework could result is an incomplete extraction of the information. Assuming that the information would be complete would not leave room to add the missing author later or straighten the spelling mistake.

\begin{foundation}
\label{foundation:incremental}
A constantly changing input asks for a constantly changing output.
\end{foundation}

This foundation is based on the same observations as Foundation \autoref{foundation:decision}. In this highly dynamic environment, there is constant change. We want our output, disambiguated authors, to adapt to this constant change. This means that, ideally, for every change of information in the input, a changed output would be precieved almost instantaneous. To achieve this, changes in decisions must be processed efficiently.

\begin{foundation}
\label{foundation:endless}
The stream of informaton is endless.
\end{foundation}

Last but not least, do not forget that the stream is almost endless. Information that would be relevant to the process of disambiguating authors is immense. Processing information of some sources can also be very intensive (ex. OCR scanning PDF documents). If the framework plans to cope with this endless growing stream, it needs to scale.

\section{Why the first approach needed a complete overhaul}

Now we can check if the measures taken in the first framework suffice to cope with the challenges posed in the foundation acknowledgements. Different aspects of the framework will be treated seperately to determine to what degree they meet the requirements.

\paragraph{The flow of data} A logical flow of data in a system with this much data to process is of primary importance. In the framework, several components do not comunicate directly with each other. All communication is "relayed" through the database. This puts enormous unnecessary load on our database. When we would scale the framework out to several servers, all communication would still pass the database nodes (over the network).

\paragraph{Scalability} We have more than one concern about the scalability of the first framework. First, the more instances in the cluster, the more queries to the database to retreive the information about a task. Secondly, the data merger would put an enormous load on the database from time to time (see next paragraph). Thirdly, a task dependency control component would also relay all its control messages through the database, increasing load in this point.

\paragraph{Performance} At certain intervals, the data merger would take a snapshot of the entire system in order to execute a cluster algorithm on it (grouping publications in physical author clusters). This is a very intensive process. As we deal with a very dynamic environment, it is preferrable that this algorithm would be run timely to keep the output up to date. The drawback with this approach is that "snapshot"-clustering is a very time consuming operation every time it is executed. If we had an incremental clustering algorithm that modifies the clusters when new information is found, it would only be intensive the first time. Afterwards it would just have to make a small modification to the clusters. This kind of approach would be more suitable in this dynamic situation.

\paragraph{The right tool for the job} In the first framework, we chose MySql as our database technology. We used it because it is familiar and widely supported. In practice, the complex queries that needed to be made were very impractical and slow due to the many joins. It occured to us that a graph database would be a better tool for the job. In \cite{vicknair2010comparison}, we find a comparison between Mysql and Neo4j, a leading graph database technology. When querying with a large depth, Neo4j is the clear winner.

\paragraph{Conclusion} Out of these experiences we learn that our new model and framework must meet some new requirements and avoid some pitfalls. In short, we need to improve the communication between the components of the framework, use a graph-database instead of a relational database and replace the data merger with a more dynamic solution. It would also be preferrable if the new framework was more scalable than the first version. We elaborate on this better framework in \autoref{newframework}.

% better title
\chapter{Theoretical Model}
% nog niet goed

First, a theoretical model is defined. This model allows us to illustrate and argue about the information, problems and respective solutions concerning our challenge to disambiguate people around the web. We introduce a three-layer graph model that will integrate structural, informational and algorithmic aspects while respecting the foundation acknowledgements we established earlier (\autoref{foundation}). In the next chapter, we show how this model can be practically realised.

\section{Graph as a model}



% \begin{mydef}
% A \textbf{Fact} is a certainty derived from a piece of information.
% \end{mydef}

% \begin{mydef}
% A \textbf{Discovery} is a fact that proofs the existence of a certain entity (an address, a publication, \ldots)
% \end{mydef}

% \begin{mydef}
% A \textbf{Relation} is a fact that denotes a relation between two entities \ldots
% \end{mydef}

\section{Three Layer Model: Structural Layer}



\subsection{Instances and Authors}

One of the main responsibilities of the structural layer is reflecting a decision through a change in structure. It is clear that this is very important. Deciding to which physical author some information belongs, is the main issue of our thesis. This part focusses on how the model adapts to such a decision. We first explain the first solution that would probably come to mind and the shortcomings of it. Then we provide a solution that deals with these shortcomings.

\paragraph{Merging} Assume that we have a source of information that has two occurences of the same name X. Along each of these occurrences, we find some extra information on the person(s) with this name. At first we do not know if this is just one person or two persons who share the same name. Now suppose that we have an algorithm that claims the latter. Therefore we have made a decision that the two pieces of information depict the same actual person. The idea is now to simply merge these two chunks of information, representing exactly one person. The situation together with its outcome is illustrated in \autoref{fig:clusteringsimple}.

The main problem with this solution is that it does not take all of the foundation acknowledgements into account. Merging two authors is a definitive operation, it can not be undone. We operate in a very dynamic environment, our system is subject to constant change. Therefore it is wise to not base a decision only on past information. Future information can provide new motives to make another decision.

\begin{figure}[htb]
	\centering
		\includegraphics[width=0.6\textwidth]{fig/clusteringsimple}
	\caption{Merging of two authors}
	\label{fig:clusteringsimple}
\end{figure}

\paragraph{Grouping} To solve this problem, we define a new concept: instances. An author is no longer defined by an author node and its property nodes. Instead, an extra level is added: an actual author is now equivalent with a collection of instances. Let us first formally define instances:

\begin{mydef}
An \textbf{Instance} is a collection of (partial) information that describes an author at a particular moment in time, it is a "snapshot" of an author.
\end{mydef}

Our goal is to try and link these chunks of information and thereby making a complete, correct image of an author. We reduced our problem to finding an optimal partioning of instances so that each of the instance-groups represent a unique author. This is a process called clustering. From now on we will use the clustering terminology. This means that an instance-group will be called a cluster and an author will by consequence be equivalent with a cluster. An any moment in time, when new information becomes available, a reclustering can take place. This means that the grouping of instances can change over time. An instance part of an author A can later be part of an author B. This method gives us the flexibility to cope with a changing environment. Applying this method to the situation described earlier leads to the result in \autoref{fig:clusteringinstances}.

\begin{figure}[htb]
	\centering
		\includegraphics[width=0.6\textwidth]{fig/clusteringinstances}
	\caption{Grouping of instances under the same author}
	\label{fig:clusteringinstances}
\end{figure}

\subsection{Problem Domain}

As already said, our goal is to link different instances of an author. Therefore the properties of those instances have to be compared. In the case of the discovery of a new instance, it is infeasible to compare it with all the instances that were already found before. That would make the problem more and more complex over time. This forces us to narrow down the problem space. For a certain instance $I_i$, we want to minimize the compared instances that do not represent the same author as this instance ("real" instances), while still trying to compare all instances that are real ones. This is formulated in equation \autoref{eq:problemspace}.

\begin{equation}
\min |I^\text{compared}_i\backslash I^\text{real}_i|,\ s.t.\ I^\text{real}_i \subseteq I^\text{compared}_i \subseteq I
\label{eq:problemspace}
\end{equation}

Names are still an important property for person identification. If we find two totally different names, we can say with confidence that they are not the same person. We let the case where the person in question would have undergone a namechange slip, this rare incident would lead us too far. Additionaly, people almost always refer to people by using their name. When information is found, we can be almost sure that it will be accompanied by a name. These properties make names an excellent candidate for narrowing down our problem space.

Note that in the example used in the previous section, we sidestepped an extra difficulty. It assumed that our source contained two instances associated with the same name. When people have totally different names, they are not the same. When they have the same name, they could be the same. But what about those names that are almost the same? If the names are not too different, the instances must certainly be compared. The definition of "too different" needs to comply with equation \autoref{eq:problemspace}.

Narrowing down the problem space using names is done using two rules. Two instance will never be compared if any of the following two conditions are satisfied:

\begin{enumerate}
\item Their family names are different.
\item The notation of their names is too different.
\end{enumerate}

The formal definition of a "family name" and the exact meaning of "too different" will be explained in more detail in .... REF. The structural represenation of name notations will be further explained in the next section.

\subsection{Name Notations}

Concerning the structure of name notations in our model we have also walked two different paths. Again, we will first explain the most obvious one. 

\paragraph{Instance-Author} Every instance belongs to a certain author (due to clustering), but it also has a number of authors it could belong to. These authors are called match-authors. If similar instances are found that are part of a match-author, it would be possible for an instance to become part of this author as well (reclustering). The problem with this approach is that an author is completely defined by its instances which change over time. The decision if an author should be a match-author therefore changes along. The problem is illustrated in \autoref{fig:namematchproblem}. Information on J. and Jake Anderson has lead us to the conclusion that they are probably the same person, the instances are clustered. Before, it was considered possible that instance John Anderson belonged to the author J. Anderson. Because of earlier clustering, J. Anderson is now actually Jake Anderson. It comprises information of both J. and Jake Anderson. This means that John Anderson will also be compared to the instance with Jake Anderson. If we would re-match John Anderson, it would never have been considered to be a part of Jake Anderson (different names). This is an inconsistency.

\begin{figure}[htb]
	\centering
		\includegraphics[width=0.6\textwidth]{fig/namematchproblem}
	\caption{Problem with instance-author matching}
	\label{fig:namematchproblem}
\end{figure}

\paragraph{Name-Name} It is better practice to keep authors and names more isolated. This new approach is illustrated in \autoref{fig:namematchsolution}. In this situation, instances associated with the name Jake Anderson will never be compared with instances with name John Anderson.

\begin{figure}[htb]
	\centering
		\includegraphics[width=0.6\textwidth]{fig/namematchsolution}
	\caption{Solution of instance-author matching: pure name matching}
	\label{fig:namematchsolution}
\end{figure}

\subsection{Conclusion and Overview}

We have addressed all the main issues that arose during the design of a structure that can stisfy the efficiency and flexibility needed according to the foundation acknowledgements (REF). We will finish with a final example of the typical structure of the graph as a theoretical model. This overview can be found in \autoref{fig:graphstructureoverview}.

\begin{figure}[htb]
	\centering
		\includegraphics[width=0.75\textwidth]{fig/graphstructureoverview}
	\caption{Overview of the structure layer of the graph model}
	\label{fig:graphstructureoverview}
\end{figure}

\section{Three Layer Model: Information Layer}

% TODO

\section{Three Layer Model: Similarity Layer}

As we have stated several times, the goal is to link instances together to form an author (cluster). Until now, linking was just an action of an algorithm. If an algorithm decides that two instances represent the same author, the algorithm links them together (by actually changing the "instance of" edges in the graph). But what about the links itself? Should these links also represent a "physical" concept in our model? This is a question answered by the similarity layer.

\paragraph{Weighted Similarities} When two instances are compared, we need to compare their properties. Properties could be email addresses, locations, people we work with, \ldots When compared instances have the location "Belgium" in common, but their email addresses are different, it would be doubtful that these two are the same. In the opposite situation (the same email address and a different location) it would be alot more probable for these instances to represent the same person. We can conclude that some properties are more distinctive than others. When the value of two properties is (partially) the same (degree of equality), we speak of a similarity. These similarities are assigned a normalized weight proportional to the distinctiveness of the property and the degree of equality.
% misschien ook nog een formele definitie hier

\paragraph{Stateful Similarities} There is a reasons why similarities have to actually be persisted. According to the foundation acknowledgements [REF] we are in need of a constantly changing output. This means that the system will incrementally build towards a solution. If we only want to calculate as few as necessary similarities when new information is found, all other similarities must already be present. With the similarity layer (plane) enabled, \autoref{fig:graphstructureoverview} could become \autoref{fig:graphstructureoverviewsim} (We left out the family node).

\begin{figure}[htb]
	\centering
		\includegraphics[width=0.75\textwidth]{fig/graphstructureoverviewsim}
	\caption{Overview of the structure layer of the graph model with the similarity layer}
	\label{fig:graphstructureoverviewsim}
\end{figure}

\section{Rules}
\label{rules}

%introductie

%3 scopes: umbrella: matched instances
%- matching-name instances
%- equal-name instances
%- shared-cluster instances

% standard rules do matching-name and equal-name

\begin{mydef}
\label{def:rule}
A \textbf{Rule} is a mechanism that issues a complex query on the graph-model in order to find similarities between instances.
\end{mydef}

\subsection{Community Rule}

Authors often work together with the same people, writing multiple publications together. Instances that are linked (due to co-authoring) to instances belonging to the same author or with a similar name are more similar themselves because of these links. This is a property that will be exploited by the community rule. 

\subsubsection{Three variants}

We can distinguish three situations, some yielding a stronger similarity than others. We list them below ordered by increasing similarity. Before going through each situation, let us sketch the common situation. There exists an instance V with a matched instance W (any scope). Instance Y is a co-instance of instance W and we just discovered that instance V is a co-instance of instance Y (a co-instance is the same as a co-author except it is on the instance level). It depends on the situation in which match scope Y is located with respect to X.

\paragraph{Name Matching} Consider that Y is a matching-name instance of X. This means that V and W have co-instances with similar names. Because of this, V and W are similar with weight $w_m$.

\paragraph{Name Equality} Consider that Y is a equal-name instance of X. This means that V and W both have a co-instance with the same name. We can now say that instance V and W are similar with weight $w_e$.

An example of these two rules combined is illustrated in \autoref{fig:coauthorrulenameeq}. The family nodes are not shown to maintain clarity. Here, the two instances with names "James" represent X and Y in the "Name Equality" case and the V and W in the "Name Matching" case. The "Yu" and "Yu C." instances then represent the complementary instances.

\begin{figure}[htb]
	\centering
		\includegraphics[width=0.75\textwidth]{fig/coauthorrulenameeq}
	\caption{}
	\label{fig:coauthorrulenameeq}
\end{figure}

\paragraph{Shared Cluster} A more complex case is where two instances are in the same cluster (bound to the same author). Assume that instances X and Y have been clusters (belong to the same author). Because it is now proven that X and Y are the same author, we add an extra similarity between V and W. This is justified because V and W now collaborate with the same author, this deservers and extra similarity with weight $w_c$. Note that this similarity can be combined with a "Name Matching" or "Name Equality" similarity. Also note that this rule is triggered by an extra event than the first two rules. All three rules are executed when it is discovered that two instances are co-instances. Here, the rule must be triggered there has been a reclustering as well. This includes the following two cases:

\begin{enumerate}
\item Instances that belonged to the same author are now instances of different authors.
\item Instances that belonged to different authors are now instances of the same author.
\end{enumerate}

In each of the two cases, it is possible for similarities to be added and to be removed. The precise execution of this is explained later. Consider \autoref{fig:coauthorrulenameeq} the starting point for the following example. Assume that the similarity between cluster C and D increases their attraction so that they are merged. Because of this action, the shared cluster rule is triggered. After querying the graph, the rule decides that the two instances with name "James" qualify for a second co-author similarity. The result is shown in \autoref{fig:coauthorrulecluster}. After this, cluster A and B will be merged as well.

\begin{figure}[htb]
	\centering
		\includegraphics[width=0.75\textwidth]{fig/coauthorrulecluster}
	\caption{}
	\label{fig:coauthorrulecluster}
\end{figure}

\paragraph{Combinations} Different combinations of these three rules result in six rules. Variant D, E and F are invertible, yielding another three rules. An overview of the six main combinations is given in \autoref{fig:coauthorrulecases}. We use a simplified view where scopes and relations are directly indicated as an edge between the instances.

\begin{figure}[h!]
	\centering
		\includegraphics[width=0.9\textwidth]{fig/coauthorrulecases}
	\caption{}
	\label{fig:coauthorrulecases}
\end{figure}

\subsubsection{Execution Details}

\paragraph{Adding Similarities} Finding co-author similarities in the graph requires complex querying. To illustrate this, the query for the case in \autoref{fig:coauthorrulenameeq} is explained in more detail below. As already said, the execution of this rule was triggered by the fact that the instance in cluster B published publication B (bolder edge). We will refer to instance "James" in cluster B as vertex v. From vertex v we follow the edges labeled "name". From this name node, every instance with this name can be reached (this is by definition the equal-name scope). We then query the co-instances via the publications and the instances that match with these co-instances via the name-nodes (by definition the matching-name scope). Consider every instance on the query-path, these are exactly the V,W and X,Y pairs. There is one condition that is not yet satisfied: instance X must also be a co-instance of instance V. We only take into account the paths where this condition is satisfied. The remaining pairs can be linked with the appropriate similarities and weights.

\paragraph{Removing Similarities} Sometimes similarities have to be removed. Our model accepts that decisions are not permanent. If the algorithm decides that two instances that were present in the same cluster should be separated, some similarities will be broken. If this is the case, all rules triggered by the merge of these two instances should be re-evaluated and the according similarities should be removed. Then the reclustering operation can take place.

\subsection{Interest Rule}

This is an important rule and it can define a lot of the precision of our whole framework. It states that different authors with the same name are unlikely to work on the same topic or have the same area of expertise. We can achieve this rule by adding keywords to every publication. These keyword could for example be noun phrases extracted from the title of the publication. Matched instances that published a publication containing the same keyword are linked with a similarity with weight $w_k$.

\paragraph{Time-Dependency} A feature to increase the precision of this rule is to deteriorate the weight of the similarities produced by this rule as the publication dates are further removed. For example, two instances with publications published in 2010 and with similar affiliations yield a stronger similarity than if the publications would have been published in 2006 and 2010.

An example of this rule can be found in \autoref{fig:interestrule}. Both publication share the keyword "Security". This causes the associated instances to be coupledby a similarity with weight $w_k$.

\begin{figure}[h!]
	\centering
		\includegraphics[width=0.9\textwidth]{fig/interestrule}
	\caption{}
	\label{fig:interestrule}
\end{figure}

\subsection{Email Rule}

This rule is based on the fact that it is very likely for two instances with the same email address to represent the same author. Time-Dependency is not used in combination with this rule, as emails are normally not interchanged between people. An example is illustrated in \autoref{fig:emailrule}

\begin{figure}[h!]
	\centering
		\includegraphics[width=0.9\textwidth]{fig/emailrule}
	\caption{}
	\label{fig:emailrule}
\end{figure}

\subsection{Affiliation Rule}

The affiliation rule exploits the affiliations of the authors. It is unlikely for an author to be tied towards multiple affiliations at the same time. This rule can benefit of the time-dependency concept. Authors sometimes change from one affiliation to another. Having the same affilation ten years later yields less of a similarity that having the same one right now. Again, an example can be found in \autoref{fig:affiliationrule}.

\begin{figure}[h!]
	\centering
		\includegraphics[width=0.9\textwidth]{fig/affiliationrule}
	\caption{}
	\label{fig:affiliationrule}
\end{figure}

\section{Conclusion}

% dissimilarities

\chapter{Framework: Pipes and Filters}
\label{newframework}

In this chapter, we design a framework based on the theoretical model from previous chapter. We start by establishing a small and general core architecture and continue with how we extended and used this core to build a flexible author-disambiguation tool. Implementation details of the framework and used algorithms and technologies in this framework will be discussed as well. Throughout the design, the requirements set in the foundation \autoref{foundation} and theoretical model were taken into account. At the end of the chapter we will discuss the strengths and weaknesses of the framework.

\section{Small and Simple Core}

We opted for a small and simple core framework. It is based on Pipes and Filters with a few extension. We chose to use this design pattern for the following reasons:

\begin{enumerate}
\item \textbf{Scalability} As said in Foundation \autoref{foundation:endless}, our algorithm needs to scale. Pipes and Filters makes this easy as all component are treated equally, can run independent from others and have a common interface.
\item \textbf{Modifiability} The Foundation (\autoref{foundation}) points to the highly dynamic environment several times. The framework will thus constantly be subject to new developments. These new developments will be in the form of new pipes, which can be plugged into the framework at any place.
\item \textbf{Maintainability} Using Pipes and Filters will reduce the complexity added by constantly modifying the framework. Pipes can always simply be replaced, moved and reconfigured in any way.
\end{enumerate}

\subsection{Architecture}

In figure \autoref{fig:architecturev2}, the architecture for the small Pipes and Filters core is shown. Each pipe has a number of input and output connectors. These connectors can be coupled to connectors of other pipes with a connection. When a pipe has executed it pushed its processed flow into the appropriate connector. This connector then pushes it to the connection that again pushes the flow into the connector of the next pipe. This pipe is then executed on the flow.

\begin{figure}[htp]
	\centering
		\includegraphics[width=0.8\textwidth]{fig/architecturev2}
	\caption{Pipes and Filters based Architecture.}
	\label{fig:architecturev2}
\end{figure}

The three classes Pipe, Connector and Connection are the real core of the architecture. All extensions and software to be created using this core will probably extend Pipe or Connection. Three important extensions are already present in the core architecture: LocalConnection, AsyncConnection and StatefulPipe. These classes will be discussed later on.

\subsection{Concepts, Terminology and Notations}

Together with this architecture we developed a few concepts that we use in combination with the core framework. These concepts allow us to build pipe configurations more easily and give us flexible control over the data flowing to the system. These concepts and associated terminology are explained in the following paragraphs.

\paragraph{Flows and Aspects} The information flowing through the pipes and connections are flows. Flows exist of different aspects. Every pipe is only interested in a few aspects of a flow. Other aspects can travel along too feed pipes later on the path. The choice of which aspect we process or send is completely free. An example of aspects are instance, publication or dependencies (see \autoref{dependencies}).

\paragraph{Connector Identifiers} Connectors need to be identified to allow us to connect a specific input or output to the right connection. The identifiers of the connectors can be any object. This makes configuring the pipes more expressive as identifiers suiting the context of the pipe can be used. For example, the output connector identifiers of the Filter pipe (\autoref{par:filterpipe}) are the booleans "true" and "false". We denote a input-connector with identifier $x$ as $Cn^i_x$ and an output-connector with identifier $y$ as $Cn^o_y$. The output ports of the Filter pipe would thus be denoted as $Cn^o_{true}$ and $Cn^o_{false}$.

\paragraph{Flow enrichment and filtering} In every pipe, we can push out whatever we want. However, a much used situation is where the pipe outputs the input flow with an extra aspect or a modified aspect. We call this flow enrichment. The other way around is also a possibility, any aspect or part of an aspect can be filtered in any pipe. However, this is not used in regular pipes but in a dedicated Filter pipe (\autoref{par:filterpipe}). Enriching and Filtering are denoted with plus and minus signs on the flow. An enriched flow A can be indicated with $A^+$ and a filtered flow A with $A^-$. The aspect that has been filtered or added can be indicated between brackets.

\paragraph{Flow Rate} Some pipes can increase or decrease the flow rate. A Filter for example will decrease the flow rate on both output connections. A Rule on the other hand will probably output multiple similarities for one input. Increasing the flow rate is denoted by adding a star (*) to the flow notation at the output. Decreasing the rate is indicated by a star on an input flow.

The notations we introduced are summarized in \autoref{fig:pipecomponents}.

\begin{figure}[htb]
	\centering
		\includegraphics[width=0.75\textwidth]{fig/pipecomponents}
	\caption{Pipe components, terminology and notations.}
	\label{fig:pipecomponents}
\end{figure}

\subsection{Core Extensions}

In the architecture (\autoref{fig:architecturev2}), we already mentioned three core extensions. These extensions extensively used throughout the program, so we included them in the architecture. Each of these three extensions is explained below.

\paragraph{Local Connections}

The most basic connection between two local pipes is one that just pushes the flow forward without memorizing anything. This connection is the bread and butter of configuring pipe networks in our program.

\paragraph{Asynchronous Connections}

A more complex connection is the asynchronous connection. It allows us to run any pipe in a distributed way. When a connection is made asynchronous, the flow of the pipe will be interpreted as the description of a job. This job will be pushed to a queue in a distributed shared memory store. A predefined set of workers ran on different machines will poll this queue constantly and execute the jobs on the machine they are running. This way, the load pushed to the pipes can be evenly distributed over a set of nodes in a computing cluster. The technology we use for this and extra details can be found in \autoref{resque}.

\paragraph{Stateful Pipes}

For some pipes, it is required to have a non-transient memory, a memory that survives between executions. This allows the pipe to process input differently based on what is present in its long-time memory. As Pipes can be run in a distributed fashion, it is possible for a pipe to be initialized on another physical machine for every execution. If the memory would be local, bound to a machine, the pipe would lose access to it if it was executed on another machine. To solve this, we use a distributed shared memory store (\autoref{redis}).

\subsection{Implementation details and Technologies}

\paragraph{The Use of Ruby and Java}

We implemented the entire framework in Ruby. We chose ruby because of the following reasons:

\begin{enumerate}
\item It is a concise and powerful language without many configuration which allows us to build something useful relatively fast.
\item It has a great community, we benefit from this as we make use of several excellent software packages from the open source ruby community.
\end{enumerate}

Pipes can either be written in Ruby or Java. We have used Ruby for almost all of them, but the clustering Pipe uses Java. This is because we use a library (JUNG) to calculate the maximum flow in a graph.

<<<<<<< HEAD:boek/newframework.tex
\paragraph{A Distrbuted Shared Key-Value Store: Redis}
\label{redis}
=======
\paragraph{A Distributed Shared Key-Value Store: Redis}
>>>>>>> ada8a6047328b60dd37ec99e92660f9205588a3c:boek/newframework.tex

We chose Redis to serve as our distributed key-value store. It provides the functionality we need and has a great client for Ruby. It is also used by Resque (explained in the next paragraph), so choosing this as our distributed store prevents us from having to install two different solutions.

Redis also has a much larger set of commands than just the get and set primitives. It has built-in atomic actions for almost anything. It can also act as a tool to provide synchronization. This is necessary as we plan to run our pipe configuration over multiple machines. It allows us to use bots pessimistic and optimistic locking.

\paragraph{A Distributed Job Processing Framework: Resque}
\label{resque}

Resque is a Redis-backed library for creating background jobs, placing those jobs on multiple queues, and processing them later. Resque is used at Github to process millions of jobs each day. It spawns a pool of workers that poll the queues for jobs. These jobs are then executed in separate processes so that they can be terminated if things get out of hand (ex. memory leaks). It is very easy to set up and provides us with a Web-based user interface to inspect our jobs and workers.

\section{Generally Useful Pipes}

With these small and simple Pipes and Filters core we built some useful pipes along the way. Clearly the filter pipe is important to have in a Pipes and Filters architecture, so we explain this one first. Merging and Splitting flows, both extensively used pipes, are mentioned second. Thirdly we introduce a pipe that allows us to do network operations. Last, we discuss two pipes more specific for our case but as they are used throughout the entire pipe system, we mention them here.

\paragraph{Filter} \label{par:filterpipe} The Filter pipe allows us to filter flow in two ways, as you can also see on \autoref{fig:filter}. First, it is possible to test a condition for each arriving flow. Depending on this condition, the flow is forwarded to either the connector with identifier "true" or identifier "false". On top of that we are able to filter aspects of the flow itself. This Filter pipe gives us the possibility to control the flow in our pipe configuration. The flows should be small and contain only aspects that are necessary.

\begin{figure}[htp]
	\centering
		\includegraphics[width=0.75\textwidth]{fig/filter}
	\caption{Filter pipe.}
	\label{fig:filter}
\end{figure}

\paragraph{Merge and Split} In the case we want to forward the result from one pipe to multiple pipes or from multiple pipes to one, we need a Split or Merge pipe. Split pipes have the default ":in" connector and split their input over $n$ output connectors identified by $1,\ldots,n$. A Merge pipe does the inverse by merging input connections $1,\ldots,n$ into one connector identified by the default ":out".

\begin{figure}[htb]
	\centering
		\includegraphics[width=0.75\textwidth]{fig/mergeandsplit}
	\caption{Merge and Split pipes.}
	\label{fig:mergeandsplit}
\end{figure}

\paragraph{Network} The network pipe has exactly one input and output. It expects an inflow of url's and provides an outflow of the contents of the web-pages found at these url's.

\section{Working Towards a Contextual System}

We now apply these general concepts and designs in the context of our thesis problem. First we agree upon a set of flow types. A flow type marks a flow and is used to distinguish flow from each other. In short, the path of a flow through a pipe configuration will depend on its type.

\subsection{Flow Types} We divide flow into four types. Two of these types carry crawled information from the web. The flow associated with these types are "information flows. The other two accompany "control messages". We indicate the type of flow $A$ as $A_type$.

\begin{enumerate}
\item \textbf{Discovery}: The discovery of a new piece of information. The source can be anything but will be probably the web or a publication document. The most important discoveries are instances and publications.
\item \textbf{Fact}: The finding of a relation between two entities is called a fact. The most important flow with this type the one with the fact that a certain instance published a publication (published fact).
\item \textbf{Similarity} Rules (\autoref{rules}) process these two types and generate similarities. These similarities are again sent through as a flow message.
\item \textbf{Cluster} Similarity messages are intended for clustering pipes, these pipes interpret the similarities and make the appropriate changes in the author clusters.
\end{enumerate}

After establishing these four main message types, we created another two generally usable pipes. Persisting discoveries is done with the Persist Discovery pipe. It provides automatic persisting of sets, automatic indexing and avoidance of duplicate discoveries. The Persist Fact pipe takes care of the same tasks for facts.

\subsection{A Graph Database: Neo4j in Combination With the Tinkerpop Stack}

As already showed, Neo4j performs much better when it comes to querying with a large depth. This is the same as doing joins in MySQL which is slow. That is why we chose Neo4j to serve as our database. On top of Neo4j we use the Tinkerpop stack. This stack creates an abstract layer above several graph databases. It uses a general property graph model to persist and query data in the graph.

We do not use SPARQL to issue our queries. We use Gremlin, part of the Tinkerpop stack. Gremlin also allows us to do pattern matching on the graph but provides us with much more flexibility in terms of controlling what is done while Gremlin traverses the graph.

Communicating with Tinkerpop is done through a HTTP interface, causing a lot of I/O waiting, but as we distribute our work to different workers, the impact of this waiting decreases.

\section{Pipe Configuration}

In this section we will give an overview of the entire pipe configuration of our program. We have divided this configuration into parts, each with their own responsibilities, difficulties and used technologies. We first discuss the information retrieval using parsers. Then we discuss the parts that form the structural layer, followed by the informational layer and the similarity layer.

\subsection{Parsers}

Retrieving information from external sources is a task of the parsers. We currently only use DBLP as an external source. Each of the author pages we want to process is fed to the parser. The outflow exists out of publication discoveries, instances discoveries and "published"-facts. We achieve this by first creating a pipe that searches the contents of the author page for links to XML representations of each of the publications. Each of this links then flows through a Network pipe, again collecting its contents. The according XML representation is then parsed and the extracted flows are pushed.

\subsection{Instance Integration}

When a new instance is discovered, it has to be integrated in the graph. Later on, information can be attached to this instance to feed the rules. The different steps of this integration have been explained in \autoref{structurallayer}. An overview of the pipe configuration is found in \autoref{fig:integrationpipe}.

First, the incoming instance discovery is lead to the "Persist Instance" pipe. This pipe persists the instance in the graph. As every new instance is assigned its very own author (cluster), the cluster is also persisted and the appropriate edges are added. The subsequent two pipes on the path deal with the persisting the family and the name nodes. If a name is already present in the system, the integration of the instance has finished. Otherwise, we must execute a name matching algorithm. This algorithm is explained in the next paragraph.

\begin{figure}[htb]
	\centering
		\includegraphics[width=0.75\textwidth]{fig/integrationpipe}
	\caption{Pipe Configuration for Instance Integration.}
	\label{fig:integrationpipe}
\end{figure}

\paragraph{Name Matching} Name Matching is the process of matching the name of a family with the names already present in the family. The goal is to reduce the problem domain size (\autoref{problemdomain}). When a new name is added, we query all the names in the family and compare them to the name that is being added. We do this using a fuzzy string matching algorithm based on Dynamic Time Warping. Taking properties of names into account lead to some modifications of the algorithm, making it especially suitable for name matching.

\begin{enumerate}
\item We use a token based approach (tokens between whitespace). The tokens themselves are compared using the Jaro Winkler fuzzy string matching algorithm.
\item The algorithm is always started from the viewpoint of the name with the least amount of tokens. An insertion has a cost of zero and a deletion is heavily penalized.
\item In case of a short token (one symbol or one symbol followed by a dot), we do not use a fuzzy approach. If the first letter matches with the first letter of the token it is being compared with, we assign a minimum cost, otherwise we assign a maximum cost.
\end{enumerate}

\subsection{Magic Facts}

As we also make use of email and affiliation rules, we needed to find a way to collect this data. We first attempted to retrieve publication documents via Microsoft Bing. Google scholar is not usable due to its security against programmatic access. The number of publications found by Bing was not enough to feed the rules. Although we created a pipe that can extract email addresses out of PDF documents and assign them to the correct author, we did not use it. We did this partly because information retrieval is not our main focus and not the real challenge. It is definitely possible to collect all this data with more effort. However, we thought it was more important to research the power of this information in the context of disambiguation.

To still be able to use the data, we manually collected email addresses and affiliations out of the publications of the authors in our test set. As this information makes a magical appearance in our system, we called the pipe that provides it "Magic Facts". Magic Facts takes a published fact, giving us the publication and instance aspects, as its input and enriches it with the email and affiliation of the instance for that publication. The pipe is illustrated in \autoref{fig:magicfacts}.

\begin{figure}[htb]
	\centering
		\includegraphics[width=0.75\textwidth]{fig/magicfactspip}
	\caption{Pipe Configuration for Magic Facts.}
	\label{fig:magicfacts}
\end{figure}

\subsection{Publications}

As a part of the informational layer (\autoref{informationallayer}), we need to persist publications in the graph along with extracted keywords. The configuration of this pipe is rather simple, it is shown in \autoref{fig:publicationpipe}. As you can see, we make good use of the PersistDiscovery and PersistFact pipes. The keyword extractor extracts keywords that define the domain of the publication. How we manage to extract this keywords is explained in the following paragraph.

\begin{figure}[htb]
	\centering
		\includegraphics[width=0.75\textwidth]{fig/publicationpipe}
	\caption{Pipe Configuration for Publication Processing.}
	\label{fig:publicationpipe}
\end{figure}

\paragraph{Keyword Extraction} To extract the keywords out of the title of a publication we use a Part of Speech Tagger. This tagger is able to identify the types of the words present in the title. Using Engtagger, a simple PoS tagger for Ruby, we extract nouns and noun phrases. The more nouns in a noun phrase, the more specific this noun phrase is. This property is being exploited in the email rule (\autoref{emailrule}).

\subsection{Rules}

Rules are the essence of the framework, it is what fulfills the goal of this framework: disambiguating authors. All the previous steps in building the framework were in order to manage the execution of these rules and getting the right information to them. A Rule basically inspects the incoming flow to initialize its context en subsequently issues a query on the graph to derive similarities. We refer to \autoref{rules} for an elaborate investigation on the working of these rules.

Most of the queries for these rules are not that difficult to implement with the help of Gremlin. However, the community rule variants require more complex queries. We explain the variant where we deal with an exact name match on the one side and a similar name on the other side. 

We start in the instance V. We need to query all instances that share a name with V. This is done by following "name" to the name node and following it back out so all instances with this name are reached (except node V). The names found here are labeled W. To find the co-instances of each of these nodes we follow the edges labeled "published". These instances are labeled Y. Now, we again follow the "name" labeled edges but this time we follow "matches" in between to get all names that are similar with the names of Y. The instances of these names are labeled X. X is now an large collection of names, but most of them do not satisfy the last condition: Any instance in X must be a co-instance of V. We require this by adding a retain pipe. The id's of the four sets of instances (V,W,Y,X) are returned in a table. Now we can add the appropriate similarities between the instances. The implementation of this query is shown below.

\begin{verbatim}
v.as("V")
  .out("name").in("name")
  .except(v).as("W")
  .out("published").in("published").as("Y")
  .out("name").both("matches").in("name").as("X")
  .retain(v.out("published").in("published").except(v).to_a)
  .table(:id,:id,:id,:id)
\end{verbatim}

All similarities produces by these rules are merged into one stream which is fed to the clustering pipe. This is illustrated in \autoref{fig:clusteringpipe}. An overview of the pipe configuration of the several rules is shown in \autoref{fig:rulespipe}.

\begin{figure}[htb]
	\centering
		\includegraphics[width=1\textwidth]{fig/clusteringpipe}
	\caption{Clustering flow}
	\label{fig:clusteringpipe}
\end{figure}

\begin{figure}[htb]
	\centering
		\includegraphics[width=1\textwidth]{fig/rulespipe}
	\caption{Pipe Configuration of the Rules Network}
	\label{fig:rulespipe}
\end{figure}

\subsection{Concurrent, incremental Clustering}

Clustering is a very important step in building towards a solution. Each new flow of information indirectly leads to a clustering operation. Acquiring new information triggers rules which on their turn yield similarities. These similarities change the balance between clusters. In the worst case, an expensive rebalancing procedure is necessary.

<<<<<<< HEAD:boek/newframework.tex
It is obvious that processing similarities is something that will be executed very frequently (numbers?). The combination of the enormous amount of similarities and their expensive processing requires us to make this process as streamlined and efficient as possible. The clustering algorithm as explained in \autoref{clustering} leads to a first, naive implementation approach. Rethinking the absolute needs of the clustering algorithm then leads to a second approach that benifits greatly of the foundations of our framework.
=======
It is obvious that processing similarities is something that will be executed very frequently (numbers?). The combination of the enormous amount of similarities and their expensive processing requires us to make this process as streamlined and efficient as possible. The clustering algorithm as explained in \autoref{clustering} leads to a first, naive implementation approach. Rethinking the absolute needs of the clustering algorithm then leads to a second approach that benefits greatly of the foundations of our framework.
>>>>>>> ada8a6047328b60dd37ec99e92660f9205588a3c:boek/newframework.tex

\paragraph{In-graph implementation} The most simple solution one can think of is to maintain the ICW and OCW for every vertex in the vertex itself. The adjacency matrix would implicitly be defined by the edges between two nodes in the similarity plane. This technique has two main drawbacks:

\begin{enumerate}
\item A lot of load would be pushed to the database.
\item There would be a need for several concurrency control mechanisms/
\end{enumerate}

If clustering could be executed without the use of the database in an efficient manner, it would be preferable. After all it is in our best interest to take as much load as possible away from the database because it is much more difficult to scale than our pipes and filters architecture. Besides, the similarity plane is not something that should be queried from our end-user application. The users are interested in the result of the clustering, not the way we got there.

\paragraph{As a stateful pipe}

Currently, the clustering algorithm is implemented as a stateful pipe. This means that all values of the adjacency matrix, the OCW's and ICW's  are maintained in our distributed key-value store. The id's of both involved clusters are provided and the graph is not needed anymore. This way, clustering is just a pipe like any other and therefore completely distributable.

In the case of similarities being processed in parallel, we need to pay some extra attention. We do not want that the clustering mechanism inflicts race conditions and, by consequence, inconsistencies on the graph. In case of an intra-cluster similarity being added, we actually do not need locks as we can make use of a technique call optimistic locking. This technique assumes that multiple transactions can complete without affecting each other, and that therefore transactions can proceed without locking the data resources that they affect. Before committing, each transaction verifies that no other transaction has modified its data. Intra-cluster similarities can always be processed in parallel for any two instances. Inter-cluster similarities on the other hand require that no changes are made to the entire cluster.

In case of an intra-cluster addition, we could watch the locks of both involved instances for modifications. If they are not set and do not change during the processing of the similarity, the transaction (intra\_add) succeeds. If the locks were modified the transaction will fail and we will retry to process the similarities. In case of an inter-cluster addition, the entire two clusters will be locked. This flow is shown in Algorithm \autoref{locking}. The cluster method locks all the instances of the clusters  of the provided instances in one transaction.

\begin{algorithm}
\caption{Locking mechanism to control concurrent similarity processing.}
\label{locking}
\begin{algorithmic}
\IF{$cluster(I_1) == cluster(I_2)$}
  \STATE \textbf{label} retry
  \STATE \textbf{watch} ${lock}_{I_1},{lock}_{I_2}$
  \IF{${lock}_{I_1}$ and ${lock}_{I_2}$}
  \STATE \textbf{goto} retry
  \ELSIF{!transaction(intra\_add$(I_1,I_2))$}
    \STATE \textbf{goto} retry
  \ENDIF
\ELSE
  \STATE \textbf{lock} cluster($I_1$),cluster($I_2$)
  \STATE inter\_add$(I_1,I_2))$
\ENDIF
\end{algorithmic}
\end{algorithm}

\subsection{Dependency Resolution}
\label{dependencies}

Another difficulty turning up in our framework is the dependency of certain flow onto others. In this case, a flow can not be executed untill its dependencies are resolved. An example is the published fact: we are not able to process this until the associated instance is integrated in the graph. To solve this we designed a stateful pipe that manages these dependencies. A dependent flow then waits in this pipe until all its dependencies are resolved.

\subsection{Overview}

The previous sections each described a component of our pipe network. All of this components and pipes need to word together. How this is done is illustrated in \autoref{completepipesmall}. We added a Message Distribution pipe with the responsibility of forwarding the messages with the right types to right pipes.

\begin{figure}[htb]
	\centering
		\includegraphics[width=1\textwidth]{fig/completepipesmall}
	\caption{Clustering flow}
	\label{fig:completepipesmall}
\end{figure}




%\chapter{Important components}

\section{Disambiguator}

\section{Disambiguation rules}

Per regel zouden we een soort van gewicht moeten meegeven die genormaliseerd is over alle regels heen. Hier moeten we nog over nadenken.

\subsection{Prequel - Doc split}

In order to enlarge our disambiguation possibilities, we want access to the actual text of the publications. Using a search engine, like Google Scholar, we can find most of the actual publications in pdf on the internet. By using the Ruby library Doc split \footnote{http://documentcloud.github.com/docsplit/}, we can easily parse the pdf into computer-understandable text. Most of the time it will be enough to just get the first page, giving us access to the authors, their affiliations and location. The first page also almost always has an abstract and a list of keywords.

In the following rules we will make a distinction between those who rely on this extra information and those who don't. The rule who rely on it will be followed by an asterisk *.

\subsection{Place of publication rule*}

This rule defines that is less likely to have multiple persons with the exact same name writing a publication in the same city.

\subsection{Co-author clustering}

Authors often work together with the same people, writing multiple publications together. If we find clusters \footnote{how do we formally define clusters and where do we explain this concept? Should this be in the sota related towards the publication 'Clustering using min cut tree'?} of co-authors which are completely indepent, it enlarges the chance that name we are investigating is related to multiple authors.

We can also use this rule to enlarge the possibility of author names being related by using this rule the other 

\chapter{Clustering Algorithm}

In this chapter we describe the clustering process, one of the core components of our framework. This process is responsible for deciding which nodes match to the same author, based on similarities between these nodes. These similarities are calculated by the different rules (see \autoref{rules}) that are implemented in our framework. Every time a new similarity is available, it is possible we have to recluster, meaning the grouping of the authors will be altered.

We need an efficient algorithm for clustering as it has to be able to handle a steady stream of new information while maintaining an optimal solution. The quality of this algorithm will make or break the quality of the output of our framework.

The algorithm we clarify in this chapter, is largely based on the algorithm described in \cite{saha2006dynamic}. It is an efficient, dynamic algorithm for clustering graphs handling insertion and deletion of edges while maintaining high quality clusters as defined by the quality requirement given in \cite{flake2004graph}. The algorithm makes heavy use of the minimum-cut tree. We implemented Gusfield's algorithm, based on \cite{rodrigues2011mpi}, and will explain this first.

The main feature of this dynamic algorithm is that it only builds part of the minimum-cut tree as and when necessary. The tree is computed over a subset of nodes, limited to a number of clusters. In our graph there might be an immense amount of authors, and thus clusters. However, the amount of authors involved with one change is limited. By just computing the clusters of this coarsened graph we can obtain the clusters of the original graph. These two properties are the reason the algorithm is efficient while maintaining an identical cluster quality as the static version described in \autoref{staticcutclustering}. Formal proof for the efficiency of this algorithm and the quality of the clusters can be found in \cite{saha2006dynamic}.

\section{Minimum-Cut Tree Algorithm}
\label{minimumcuttree}

\begin{algorithm}
\caption{Sequential Gusfield's Algorithm}
\label{mincutgusfield}
\begin{algorithmic}
\STATE \textbf{Input:} $G = (V,E,w)$ 
\STATE \textbf{Output:} $T = (V,E,f)$, where T is a cut tree of G
\STATE $V(T) \leftarrow V(G); E(T) \leftarrow \emptyset$
\FOR{$tree_i, flow_i, 1 \leq i \leq N$}
\STATE $tree_i \leftarrow 1; flow_i \leftarrow 0$
\ENDFOR
\STATE // $n - 1$ maximum flow iterations
\FOR{$s \leftarrow 2 $to$ N$}
\STATE $flow_s \leftarrow $MaxFlow$(s, tree_s)$
\STATE // adjust the $tree$ with Cut($s,tree_s$)
\STATE // c1 contains s and connected nodes, c2 contains $tree_s$ and connected nodes
	\FOR{$t \leftarrow 1 $ to $ N$}
		\IF{$t == s \vee t == tree_s$}
			\STATE next
		\ELSIF{$t \in c1 \wedge s \in c2$}
			\STATE $tree_t \leftarrow s$
		\ELSIF{$t \in c2 \wedge s \in c1$}
			\STATE $tree_t \leftarrow tree_s$
		\ENDIF
	\ENDFOR
\ENDFOR
\STATE // Generate T
\FOR{$s \leftarrow 1 $ to $ N$}
\STATE $E(T) \leftarrow E(T) \cup {s, tree_s}$
\STATE $f({s,tree_s}) \leftarrow flow_s$
\ENDFOR
\RETURN T
\end{algorithmic}
\end{algorithm}

% Perhaps this first paragraph should be moved to the SOTA

We briefly explain what is understood under a minimum cut tree, as clarified in \cite{saha2006dynamic}. 

Let $G = (V,E,w)$ denote a weighted undirected graph with $n = |V|$ nodes or vertices and $m = |E|$ links or edges. Each edge $e = (u, v), u,v \in V$ has an associated weight $w(u,v) > 0$. Let $s$ and $t$ be two nodes in $G(V,E)$, the source and destination. The minimum-cut of G with respect to $s$ and $t$ is a partition of $V$ which we will call $S$ and $V/S$. These partitions should be such that $s \in S, t \in V/S$ and the total weight of the edges linking nodes between the two partitions is minimum. The sum of these weights is called the cut-value and is denoted as $c(S,V/S)$. 

The minimum cut tree is a tree on $V$ such that inspecting the path between $s$ and $t$ in the tree, the minimum-cut of $G$ with respect of $s$ and $t$ can be obtained. Removal of the minimum weight edge in the path yields the two partitions and the weight of the corresponding edge gives the cut-value.

We implemented a sequential version of Gusfield's algorithm which calculates the minimum cut tree of any given graph. The pseudocode is given by \autoref{mincutgusfield}. In the pseudocode we use numbers to point to nodes or vertices. These numbers can be chosen randomly.

The algorithm consists of $n-1$ iterations of a Maximum Flow algorithm and for every iteration a different vertex is chosen as source. The destination vertex is determined by previous iterations and is saved in the tree. Initially all vertices of the output tree point to node 1, but this can be adjusted after each iteration. This adjustment depends on the minimum-cut between the current source and destination. We split all the nodes in two collections, using this minimum-cut. We adjust the parent of each node if it is on another side as its current parent, which is stored in the tree.

We choose an implementation of the Edmonds-Karp algorithm to find the maximum flow and the minimum-cut. This algorithm is an implementation of the Ford-Fulkerson method for computing the maximum flow and is provided to us through the Java library JUNG.

\section{Cut Clustering}

\cite{flake2004graph} defines a static algorithm for clustering based on minimum cut trees. \autoref{staticcutclustering} gives the pseudocode of the basic cut clustering algorithm. It adds an artificial sink $t$ to all the vertices of the graph with weight $\alpha > 0$. The minimum cut tree is computed using this new graph and the disjoint components obtained after removing the artificial vertex $t$, are the required clusters.

\begin{algorithm}
\caption{Static Cut Clustering Algorithm of \cite{flake2004graph}}
\label{staticcutclustering}
\begin{algorithmic}
\STATE \textbf{Input:} $G = (V,E,c), \alpha$ 
\STATE \textbf{Output:} Cluster of G
\STATE $V \leftarrow V \cup t$
\FOR{all vertices $v$ in G}
	\STATE Connect $t$ to $v$ with edge of weight $\alpha$
\ENDFOR
\STATE $G'(V',E') \leftarrow$ new graph after connecting t to V
\STATE Calculate the minimum-cut tree $T'$ of $G'$
\STATE Remove $t$ from T
\RETURN All connected components as clusters of G
\end{algorithmic}
\end{algorithm}

In \cite{saha2006dynamic}, they have extended this basic algorithm allowing it to work efficiently on dynamic graphs. They use some important new components which we also will use throughout the explanation of the algorithm. The adjacency matrix $A$ of $G$ is an $n \times n$ matrix in which $A(i,j) = w(i,j)$ if $(i,j) \in E$, else $A(i,j) = 0$. The algorithm also maintains two new variables for every vertex, the In Cluster Weight (ICW) and the Out Cluster Weigh (OCW). If $C_1,C_2,...C_s$ are the clusters of $G(V,E)$ then ICW and OCW are defined as below.

\begin{mydef}
\textbf{In Cluster Weight (ICW)} of a vertex $v \in V$ is defined as the total weight of the edges linking the vertex $v$ to all the vertices which belong to the same cluster as $v$. That is, if $v \in C_i$, $0 \leq i \leq s$ then $ICW(v) = \sum_{u \in C_i}{w(v,u)}$
\end{mydef}

\begin{mydef}
\textbf{Out Cluster Weight (OCW)} of a vertex $v \in V$ is defined as the total weight of the edges linking the vertex $v$ to all the vertices which do not belong to the same cluster as $v$. That is, if $v \in C_i$, $0 \leq i \leq s$ then $OCW(v) = \sum_{u \in C_j, j \neq i}{w(v,u)}$
\end{mydef}

A similarity between two instances is represented as a new edge between two nodes with a given weight. Using this weight, the probability that two instances belong to the same author can be derived. This weight is calculated by the rules in the disambiguator. There are two different possibilities that have to be treated separately: inter- and intra-cluster edge addition.

We assume we have $C = {C_1,C_2...C_s}$ as the clusters of the graph $G(V,E)$ which have been calculated in previous steps. We denote $A$ as the adjacency matrix of $G$.

\subsection{Intra-Cluster Edge Addition}

Intra-cluster edge addition means that both the nodes of the added edge belong to the same cluster. The result is that the cluster becomes stronger connected. We only have to update the $ICW$ and the adjacency matrix $A$ while the nodes remain unchanged. \autoref{intracluster} shows the pseudocode.

\begin{algorithm}
\caption{Intra-cluster edge addition between nodes $i$ and $j$ with weight $w(i,j)$}
\label{intracluster}
\begin{algorithmic}
\STATE \textbf{Input:} $G(V,E), (i,j), w(i,j)$ 
\STATE \textbf{Output:} Clusters of G
\STATE $A(i,j) \leftarrow A(i,j) + w(i,j)$
\STATE $ICW(i) \leftarrow ICW(i) + w(i,j)$
\STATE $ICW(j) \leftarrow ICW(j) + w(i,j)$
\RETURN $C$
\end{algorithmic}
\end{algorithm}

\subsection{Inter-Cluster Edge Addition}

Addition of an edge whose end nodes belong to different clusters is more challenging as it increases the connectivity across different clusters. This means the cluster quality of the clusters involved is lowered and as a result reclustering might be necessary when the quality is no longer maintained. 

There are three identifiable cases:

\begin{enumerate}
	\item \textbf{CASE 1} The addition of the edge does not break the clusters involved.
	\item \textbf{CASE 2} The addition of the edge causes the clusters to be so well connected that they are merged into one.
	\item \textbf{CASE 3} The new edge deteriorates the cluster quality and the nodes in both the clusters have to be reclustered.
\end{enumerate}

In each of these three cases, the addition of the new edge results in updating the adjacency matrix $A$ and the Outer Cluster Weight of the nodes involved. This is described in \autoref{updatevalues}. In order to understand the complete algorithm, we first need to explain two used processes: merging and contracting of clusters.

\begin{algorithm}
\caption{Updating the adjacency matrix $A$ and the Outer Cluster Weight: UPDATE($(i,j),w(i,j)$)}
\label{updatevalues}
\begin{algorithmic}
\STATE \textbf{Input:} Edge $(i,j)$ and weight $w(i,j)$
\STATE $A(i,j) \leftarrow A(i,j) + w(i,j)$
\STATE $OCW(i) \leftarrow OCW(i) + w(i,j)$
\STATE $OCW(j) \leftarrow OCW(j) + w(i,j)$
\end{algorithmic}
\end{algorithm}

\paragraph{Merging of Clusters} occurs in CASE 2 and is described in \autoref{merging}. Two clusters $C_u$ and $C_v$ are merged into one new cluster containing the nodes of the two original clusters. This causes the ICW of all the nodes involved to increase and the OCW to decrease as all the nodes are now more connected.

\begin{algorithm}
\caption{Merging of clusters $C_u$ and $C_v$: MERGE($C_u,C_v$)}
\label{merging}
\begin{algorithmic}
\STATE \textbf{Input:} $C_u$ and $C_v$ 
\STATE \textbf{Output:} Merged cluster
\STATE $D \leftarrow C_u \cup C_v$
\FOR{$\forall u \in C_u$}
	\STATE $ICW(u) \leftarrow ICW(u) + \sum_{v \in C_v}{w(u,v)}$
	\STATE $OCW(u) \leftarrow OCW(u) - \sum_{v \in C_v}{w(u,v)}$
\ENDFOR
\FOR{$\forall v \in C_v$}
	\STATE $ICW(v) \leftarrow ICW(v) + \sum_{u \in C_u}{w(v,u)}$
	\STATE $OCW(v) \leftarrow OCW(v) - \sum_{u \in C_u}{w(v,u)}$
\ENDFOR
\RETURN $D$
\end{algorithmic}
\end{algorithm}

\paragraph{Contraction of Clusters} occurs in CASE 3 as part of the reclustering process and is shown in \autoref{contracting}. All the nodes outside the set of clusters $S$ are replaced by a single new node $x$. Self loops that are created in this process are removed and parallel edges are replaced by a single edge with weight equal to the sum of the parallel edges. The reason we consider the clusters outside $S$ is because $S$ will generally be small while the other clusters will contain a lot of nodes.

\begin{algorithm}
\caption{Contraction of clusters outside the set of clusters $S$: CONTRACT($G(V,E),S)$)}
\label{contracting}
\begin{algorithmic}
\STATE \textbf{Input:} $G(V,E)$ and set of clusters $S$ 
\STATE \textbf{Output:} Contracted graph $G'(V',E')$
\STATE Add all vertices of $S$ to a new graph $G'$
\STATE $\forall i,j \in V' : A'(i,j) \leftarrow A(i,j)$
\STATE Add a new vertex $x$ to $G'$
\FOR{$\forall i \in \left\{V' - x\right\}$}
	\STATE $A'(i,x) = ICW(i) + OCW(i) - \sum_{j \in \left\{ V' - x \right\} }{A'(i,j)}$
\ENDFOR
\STATE Obtain $E'$ from $A'$
\RETURN $G'(V',E')$
\end{algorithmic}
\end{algorithm}

The complete algorithm for the addition of an edge between nodes that are contained in different clusters, is shown in \autoref{intercluster}. If the addition of the new edge does not deteriorate the clustering quality (CASE 1), the clusters are maintained and we only have to update the adjacency matrix $A$ and the Outer Cluster Weight of the nodes involved. 

If the hypothetical cluster quality of the cluster created by the combination of the two current clusters exceeds the threshold $\alpha$ (CASE 2), the clusters can be merged. 

Otherwise (CASE 3), we create a new, coarsened graph by contracting all the clusters except $C_u$ and $C_v$ to a node $x$. The resulting graph is significantly smaller than the original graph, resulting in lower execution times. Similarly as in \autoref{staticcutclustering}, we add an artificial sink $t$ to the coarsened graph. After adding edges between $t$ and the other nodes, we calculate the minimum-cut tree, as described in \autoref{minimumcuttree}. The connected components are computed from the resulting tree, after removing $t$. The components containing vertices of $C_u$ and $C_v$ along with the clusters $C-\left\{C_v,C_u\right\}$ are returned as the new clusters of the original graph. As a result of the reclustering, we have to recalculate the ICW and OCW of the nodes involved. This can easily be done by comparing the nodes in the old cluster and the nodes in the new cluster for each node.

\begin{algorithm}
\caption{Inter-cluster edge addition between nodes $i$ and $j$ with weight $w(i,j)$}
\label{intercluster}
\begin{algorithmic}
\STATE \textbf{Input:} $G(V,E), (i,j), w(i,j)$ and $\alpha$ 
\STATE \textbf{Output:} Clusters of $G$
\STATE $i \in C_u$ and $j \in C_v$
\IF{$\frac{\sum_{u \in C_u}{OCW(u) + w(i,j)}}{\left|V - C_u\right|} \leq \alpha \wedge \frac{\sum_{v \in C_v}{OCW(v) + w(i,j)}}{\left|V - C_v\right|} \leq \alpha$}
	\STATE // CASE 1
	\STATE $UPDATE((i,j),w(i,j))$
	\RETURN $C$
\ELSIF{$\frac{2 * c(C_u,C_v)}{V} \geq \alpha$}
	\STATE // CASE 2
	\STATE $UPDATE((i,j),w(i,j))$
	\STATE $D \leftarrow MERGE(C_u,C_v)$
	\RETURN $C + D - \left\{C_u,C_v\right\}$
\ELSE
	\STATE // CASE 3
	\STATE $UPDATE((i,j),w(i,j))$
	\STATE $G'(V',E') \leftarrow CONTRACT(G(V,E),\left\{C_u,C_v\right\} )$
	\STATE Connect $t$ to $v, \forall v \in C_u,C_v$ with edge of weight $\alpha$
	\STATE Connect $t$ to $V' - \left\{C_u, C_v\right\}$ with edge of weight $\alpha * \left| V - C_u - C_v \right|$
	\STATE G''(V'',E'') is the graph resulting in connecting $t$
	\STATE Calculate minimum-cut tree $T''$ of $G''(V'',E'')$
	\STATE Remove $t$
	\STATE // $\left\{D_1,D_2...D_k\right\}, k > 0$, are the connected components of $T''$ after removing $t$
	\STATE $C \leftarrow \left\{D_1,D_2...D_k,C_1,C_2...C_s\right\} - \left\{ C_u,C_v \right\}$
	\STATE // Update OCW and ICW using the new clusters
	\FOR{$\forall i \in C_u \cup C_v$}
		\STATE $sum \leftarrow \sum_{j \in old~neighbours}{A(i,j)}$
		\STATE $sum \leftarrow sum - \sum_{j \in new~neighbours}{A(i,j)}$
		\STATE $ICW(i) \leftarrow ICW(i) + sum$
		\STATE $OCW(i) \leftarrow OCW(i) - sum$
	\ENDFOR
	\RETURN $C$
\ENDIF
\end{algorithmic}
\end{algorithm}

We have a lot of rules resulting in small similarities, often following up on each other. This means we often will add edges will low weights. When testing the framework, we noticed that the clustering process often had to evaluate case 3 repeatedly, while the output did not result in reclustering. As case 3 is very time-intensive, especially when the clusters are getting bigger, we added an extra case in between case 1 and 2 based on the formula used to evaluate case 2. If $\frac{2 * c(C_u,C_v)}{V} \leq \alpha / 2$, we essentially fall back to case 1. This does not deteriorate the cluster quality.

% TODO moeten we hier nog vermelden hoe we dit precies geimplementeerd hebben ? Namelijk combinatie van Ruby, Java en Resque ? Misschien ook de voordelen aanhalen van deze aanpak tov andere mogelijke manieren of toch uitleggen waarom we hiervoor gekozen hebben ..


\chapter{Clustering Algorithm}

In this chapter we describe the clustering process, one of the core components of our framework. This process is responsible for deciding which nodes match to the same author, based on similarities between these nodes. These similarities are calculated by the different rules (see \autoref{rules}) that are implemented in our framework. Every time a new similarity is available, it is possible we have to recluster, meaning the grouping of the authors will be altered.

We need an efficient algorithm for clustering as it has to be able to handle a steady stream of new information while maintaining an optimal solution. The quality of this algorithm will make or break the quality of the output of our framework.

The algorithm we clarify in this chapter, is largely based on the algorithm described in \cite{saha2006dynamic}. It is an efficient, dynamic algorithm for clustering graphs handling insertion and deletion of edges while maintaining high quality clusters as defined by the quality requirement given in \cite{flake2004graph}. The algorithm makes heavy use of the minimum-cut tree. We implemented Gusfield's algorithm, based on \cite{rodrigues2011mpi}, and will explain this first.

The main feature of this dynamic algorithm is that it only builds part of the minimum-cut tree as and when necessary. The tree is computed over a subset of nodes, limited to a number of clusters. In our graph there might be an immense amount of authors, and thus clusters. However, the amount of authors involved with one change is limited. By just computing the clusters of this coarsened graph we can obtain the clusters of the original graph. These two properties are the reason the algorithm is efficient while maintaining an identical cluster quality as the static version described in \autoref{staticcutclustering}. Formal proof for the efficiency of this algorithm and the quality of the clusters can be found in \cite{saha2006dynamic}.

\section{Minimum-Cut Tree Algorithm}
\label{minimumcuttree}

\begin{algorithm}
\caption{Sequential Gusfield's Algorithm}
\label{mincutgusfield}
\begin{algorithmic}
\STATE \textbf{Input:} $G = (V,E,w)$ 
\STATE \textbf{Output:} $T = (V,E,f)$, where T is a cut tree of G
\STATE $V(T) \leftarrow V(G); E(T) \leftarrow \emptyset$
\FOR{$tree_i, flow_i, 1 \leq i \leq N$}
\STATE $tree_i \leftarrow 1; flow_i \leftarrow 0$
\ENDFOR
\STATE // $n - 1$ maximum flow iterations
\FOR{$s \leftarrow 2 $to$ N$}
\STATE $flow_s \leftarrow $MaxFlow$(s, tree_s)$
\STATE // adjust the $tree$ with Cut($s,tree_s$)
\STATE // c1 contains s and connected nodes, c2 contains $tree_s$ and connected nodes
	\FOR{$t \leftarrow 1 $ to $ N$}
		\IF{$t == s \vee t == tree_s$}
			\STATE next
		\ELSIF{$t \in c1 \wedge s \in c2$}
			\STATE $tree_t \leftarrow s$
		\ELSIF{$t \in c2 \wedge s \in c1$}
			\STATE $tree_t \leftarrow tree_s$
		\ENDIF
	\ENDFOR
\ENDFOR
\STATE // Generate T
\FOR{$s \leftarrow 1 $ to $ N$}
\STATE $E(T) \leftarrow E(T) \cup {s, tree_s}$
\STATE $f({s,tree_s}) \leftarrow flow_s$
\ENDFOR
\RETURN T
\end{algorithmic}
\end{algorithm}

% Perhaps this first paragraph should be moved to the SOTA

We briefly explain what is understood under a minimum cut tree, as clarified in \cite{saha2006dynamic}. 

Let $G = (V,E,w)$ denote a weighted undirected graph with $n = |V|$ nodes or vertices and $m = |E|$ links or edges. Each edge $e = (u, v), u,v \in V$ has an associated weight $w(u,v) > 0$. Let $s$ and $t$ be two nodes in $G(V,E)$, the source and destination. The minimum-cut of G with respect to $s$ and $t$ is a partition of $V$ which we will call $S$ and $V/S$. These partitions should be such that $s \in S, t \in V/S$ and the total weight of the edges linking nodes between the two partitions is minimum. The sum of these weights is called the cut-value and is denoted as $c(S,V/S)$. 

The minimum cut tree is a tree on $V$ such that inspecting the path between $s$ and $t$ in the tree, the minimum-cut of $G$ with respect of $s$ and $t$ can be obtained. Removal of the minimum weight edge in the path yields the two partitions and the weight of the corresponding edge gives the cut-value.

We implemented a sequential version of Gusfield's algorithm which calculates the minimum cut tree of any given graph. The pseudocode is given by \autoref{mincutgusfield}. In the pseudocode we use numbers to point to nodes or vertices. These numbers can be chosen randomly.

The algorithm consists of $n-1$ iterations of a Maximum Flow algorithm and for every iteration a different vertex is chosen as source. The destination vertex is determined by previous iterations and is saved in the tree. Initially all vertices of the output tree point to node 1, but this can be adjusted after each iteration. This adjustment depends on the minimum-cut between the current source and destination. We split all the nodes in two collections, using this minimum-cut. We adjust the parent of each node if it is on another side as its current parent, which is stored in the tree.

We choose an implementation of the Edmonds-Karp algorithm to find the maximum flow and the minimum-cut. This algorithm is an implementation of the Ford-Fulkerson method for computing the maximum flow and is provided to us through the Java library JUNG.

\section{Cut Clustering}

\cite{flake2004graph} defines a static algorithm for clustering based on minimum cut trees. \autoref{staticcutclustering} gives the pseudocode of the basic cut clustering algorithm. It adds an artificial sink $t$ to all the vertices of the graph with weight $\alpha > 0$. The minimum cut tree is computed using this new graph and the disjoint components obtained after removing the artificial vertex $t$, are the required clusters.

\begin{algorithm}
\caption{Static Cut Clustering Algorithm of \cite{flake2004graph}}
\label{staticcutclustering}
\begin{algorithmic}
\STATE \textbf{Input:} $G = (V,E,c), \alpha$ 
\STATE \textbf{Output:} Cluster of G
\STATE $V \leftarrow V \cup t$
\FOR{all vertices $v$ in G}
	\STATE Connect $t$ to $v$ with edge of weight $\alpha$
\ENDFOR
\STATE $G'(V',E') \leftarrow$ new graph after connecting t to V
\STATE Calculate the minimum-cut tree $T'$ of $G'$
\STATE Remove $t$ from T
\RETURN All connected components as clusters of G
\end{algorithmic}
\end{algorithm}

In \cite{saha2006dynamic}, they have extended this basic algorithm allowing it to work efficiently on dynamic graphs. They use some important new components which we also will use throughout the explanation of the algorithm. The adjacency matrix $A$ of $G$ is an $n \times n$ matrix in which $A(i,j) = w(i,j)$ if $(i,j) \in E$, else $A(i,j) = 0$. The algorithm also maintains two new variables for every vertex, the In Cluster Weight (ICW) and the Out Cluster Weigh (OCW). If $C_1,C_2,...C_s$ are the clusters of $G(V,E)$ then ICW and OCW are defined as below.

\begin{mydef}
\textbf{In Cluster Weight (ICW)} of a vertex $v \in V$ is defined as the total weight of the edges linking the vertex $v$ to all the vertices which belong to the same cluster as $v$. That is, if $v \in C_i$, $0 \leq i \leq s$ then $ICW(v) = \sum_{u \in C_i}{w(v,u)}$
\end{mydef}

\begin{mydef}
\textbf{Out Cluster Weight (OCW)} of a vertex $v \in V$ is defined as the total weight of the edges linking the vertex $v$ to all the vertices which do not belong to the same cluster as $v$. That is, if $v \in C_i$, $0 \leq i \leq s$ then $OCW(v) = \sum_{u \in C_j, j \neq i}{w(v,u)}$
\end{mydef}

A similarity between two instances is represented as a new edge between two nodes with a given weight. Using this weight, the probability that two instances belong to the same author can be derived. This weight is calculated by the rules in the disambiguator. There are two different possibilities that have to be treated separately: inter- and intra-cluster edge addition.

We assume we have $C = {C_1,C_2...C_s}$ as the clusters of the graph $G(V,E)$ which have been calculated in previous steps. We denote $A$ as the adjacency matrix of $G$.

\subsection{Intra-Cluster Edge Addition}

Intra-cluster edge addition means that both the nodes of the added edge belong to the same cluster. The result is that the cluster becomes stronger connected. We only have to update the $ICW$ and the adjacency matrix $A$ while the nodes remain unchanged. \autoref{intracluster} shows the pseudocode.

\begin{algorithm}
\caption{Intra-cluster edge addition between nodes $i$ and $j$ with weight $w(i,j)$}
\label{intracluster}
\begin{algorithmic}
\STATE \textbf{Input:} $G(V,E), (i,j), w(i,j)$ 
\STATE \textbf{Output:} Clusters of G
\STATE $A(i,j) \leftarrow A(i,j) + w(i,j)$
\STATE $ICW(i) \leftarrow ICW(i) + w(i,j)$
\STATE $ICW(j) \leftarrow ICW(j) + w(i,j)$
\RETURN $C$
\end{algorithmic}
\end{algorithm}

\subsection{Inter-Cluster Edge Addition}

Addition of an edge whose end nodes belong to different clusters is more challenging as it increases the connectivity across different clusters. This means the cluster quality of the clusters involved is lowered and as a result reclustering might be necessary when the quality is no longer maintained. 

There are three identifiable cases:

\begin{enumerate}
	\item \textbf{CASE 1} The addition of the edge does not break the clusters involved.
	\item \textbf{CASE 2} The addition of the edge causes the clusters to be so well connected that they are merged into one.
	\item \textbf{CASE 3} The new edge deteriorates the cluster quality and the nodes in both the clusters have to be reclustered.
\end{enumerate}

In each of these three cases, the addition of the new edge results in updating the adjacency matrix $A$ and the Outer Cluster Weight of the nodes involved. This is described in \autoref{updatevalues}. In order to understand the complete algorithm, we first need to explain two used processes: merging and contracting of clusters.

\begin{algorithm}
\caption{Updating the adjacency matrix $A$ and the Outer Cluster Weight: UPDATE($(i,j),w(i,j)$)}
\label{updatevalues}
\begin{algorithmic}
\STATE \textbf{Input:} Edge $(i,j)$ and weight $w(i,j)$
\STATE $A(i,j) \leftarrow A(i,j) + w(i,j)$
\STATE $OCW(i) \leftarrow OCW(i) + w(i,j)$
\STATE $OCW(j) \leftarrow OCW(j) + w(i,j)$
\end{algorithmic}
\end{algorithm}

\paragraph{Merging of Clusters} occurs in CASE 2 and is described in \autoref{merging}. Two clusters $C_u$ and $C_v$ are merged into one new cluster containing the nodes of the two original clusters. This causes the ICW of all the nodes involved to increase and the OCW to decrease as all the nodes are now more connected.

\begin{algorithm}
\caption{Merging of clusters $C_u$ and $C_v$: MERGE($C_u,C_v$)}
\label{merging}
\begin{algorithmic}
\STATE \textbf{Input:} $C_u$ and $C_v$ 
\STATE \textbf{Output:} Merged cluster
\STATE $D \leftarrow C_u \cup C_v$
\FOR{$\forall u \in C_u$}
	\STATE $ICW(u) \leftarrow ICW(u) + \sum_{v \in C_v}{w(u,v)}$
	\STATE $OCW(u) \leftarrow OCW(u) - \sum_{v \in C_v}{w(u,v)}$
\ENDFOR
\FOR{$\forall v \in C_v$}
	\STATE $ICW(v) \leftarrow ICW(v) + \sum_{u \in C_u}{w(v,u)}$
	\STATE $OCW(v) \leftarrow OCW(v) - \sum_{u \in C_u}{w(v,u)}$
\ENDFOR
\RETURN $D$
\end{algorithmic}
\end{algorithm}

\paragraph{Contraction of Clusters} occurs in CASE 3 as part of the reclustering process and is shown in \autoref{contracting}. All the nodes outside the set of clusters $S$ are replaced by a single new node $x$. Self loops that are created in this process are removed and parallel edges are replaced by a single edge with weight equal to the sum of the parallel edges. The reason we consider the clusters outside $S$ is because $S$ will generally be small while the other clusters will contain a lot of nodes.

\begin{algorithm}
\caption{Contraction of clusters outside the set of clusters $S$: CONTRACT($G(V,E),S)$)}
\label{contracting}
\begin{algorithmic}
\STATE \textbf{Input:} $G(V,E)$ and set of clusters $S$ 
\STATE \textbf{Output:} Contracted graph $G'(V',E')$
\STATE Add all vertices of $S$ to a new graph $G'$
\STATE $\forall i,j \in V' : A'(i,j) \leftarrow A(i,j)$
\STATE Add a new vertex $x$ to $G'$
\FOR{$\forall i \in \left\{V' - x\right\}$}
	\STATE $A'(i,x) = ICW(i) + OCW(i) - \sum_{j \in \left\{ V' - x \right\} }{A'(i,j)}$
\ENDFOR
\STATE Obtain $E'$ from $A'$
\RETURN $G'(V',E')$
\end{algorithmic}
\end{algorithm}

The complete algorithm for the addition of an edge between nodes that are contained in different clusters, is shown in \autoref{intercluster}. If the addition of the new edge does not deteriorate the clustering quality (CASE 1), the clusters are maintained and we only have to update the adjacency matrix $A$ and the Outer Cluster Weight of the nodes involved. 

If the hypothetical cluster quality of the cluster created by the combination of the two current clusters exceeds the threshold $\alpha$ (CASE 2), the clusters can be merged. 

Otherwise (CASE 3), we create a new, coarsened graph by contracting all the clusters except $C_u$ and $C_v$ to a node $x$. The resulting graph is significantly smaller than the original graph, resulting in lower execution times. Similarly as in \autoref{staticcutclustering}, we add an artificial sink $t$ to the coarsened graph. After adding edges between $t$ and the other nodes, we calculate the minimum-cut tree, as described in \autoref{minimumcuttree}. The connected components are computed from the resulting tree, after removing $t$. The components containing vertices of $C_u$ and $C_v$ along with the clusters $C-\left\{C_v,C_u\right\}$ are returned as the new clusters of the original graph. As a result of the reclustering, we have to recalculate the ICW and OCW of the nodes involved. This can easily be done by comparing the nodes in the old cluster and the nodes in the new cluster for each node.

\begin{algorithm}
\caption{Inter-cluster edge addition between nodes $i$ and $j$ with weight $w(i,j)$}
\label{intercluster}
\begin{algorithmic}
\STATE \textbf{Input:} $G(V,E), (i,j), w(i,j)$ and $\alpha$ 
\STATE \textbf{Output:} Clusters of $G$
\STATE $i \in C_u$ and $j \in C_v$
\IF{$\frac{\sum_{u \in C_u}{OCW(u) + w(i,j)}}{\left|V - C_u\right|} \leq \alpha \wedge \frac{\sum_{v \in C_v}{OCW(v) + w(i,j)}}{\left|V - C_v\right|} \leq \alpha$}
	\STATE // CASE 1
	\STATE $UPDATE((i,j),w(i,j))$
	\RETURN $C$
\ELSIF{$\frac{2 * c(C_u,C_v)}{V} \geq \alpha$}
	\STATE // CASE 2
	\STATE $UPDATE((i,j),w(i,j))$
	\STATE $D \leftarrow MERGE(C_u,C_v)$
	\RETURN $C + D - \left\{C_u,C_v\right\}$
\ELSE
	\STATE // CASE 3
	\STATE $UPDATE((i,j),w(i,j))$
	\STATE $G'(V',E') \leftarrow CONTRACT(G(V,E),\left\{C_u,C_v\right\} )$
	\STATE Connect $t$ to $v, \forall v \in C_u,C_v$ with edge of weight $\alpha$
	\STATE Connect $t$ to $V' - \left\{C_u, C_v\right\}$ with edge of weight $\alpha * \left| V - C_u - C_v \right|$
	\STATE G''(V'',E'') is the graph resulting in connecting $t$
	\STATE Calculate minimum-cut tree $T''$ of $G''(V'',E'')$
	\STATE Remove $t$
	\STATE // $\left\{D_1,D_2...D_k\right\}, k > 0$, are the connected components of $T''$ after removing $t$
	\STATE $C \leftarrow \left\{D_1,D_2...D_k,C_1,C_2...C_s\right\} - \left\{ C_u,C_v \right\}$
	\STATE // Update OCW and ICW using the new clusters
	\FOR{$\forall i \in C_u \cup C_v$}
		\STATE $sum \leftarrow \sum_{j \in old~neighbours}{A(i,j)}$
		\STATE $sum \leftarrow sum - \sum_{j \in new~neighbours}{A(i,j)}$
		\STATE $ICW(i) \leftarrow ICW(i) + sum$
		\STATE $OCW(i) \leftarrow OCW(i) - sum$
	\ENDFOR
	\RETURN $C$
\ENDIF
\end{algorithmic}
\end{algorithm}

We have a lot of rules resulting in small similarities, often following up on each other. This means we often will add edges will low weights. When testing the framework, we noticed that the clustering process often had to evaluate case 3 repeatedly, while the output did not result in reclustering. As case 3 is very time-intensive, especially when the clusters are getting bigger, we added an extra case in between case 1 and 2 based on the formula used to evaluate case 2. If $\frac{2 * c(C_u,C_v)}{V} \leq \alpha / 2$, we essentially fall back to case 1. This does not deteriorate the cluster quality.

% TODO moeten we hier nog vermelden hoe we dit precies geimplementeerd hebben ? Namelijk combinatie van Ruby, Java en Resque ? Misschien ook de voordelen aanhalen van deze aanpak tov andere mogelijke manieren of toch uitleggen waarom we hiervoor gekozen hebben ..


\include{expertfinding}

\chapter{Framework Evaluation and Results}



We need to evaluate the performance of the framework. We will mainly focus our tests on the clustering of authors and the influence of different combinations of rules and parameters on the results. Proper clustering results in a collection of publications related to one author, allowing to define specialties and the level of expertise. 

In order to accomplish realistic and useful results, it's important the tests resemble realistic use-cases. This means testing against both general and borderline queries. Parameters making it easier in disambiguation are rare and country specific author names, a multitude of publications which are written with the same co-authors or the inclusion of the same email address. A combination of these parameters allows for easier manual checking of results, but is consequently less challenging. 

Borderline cases make more defiant evaluations. Author names containing foreign characters or accents make it easier to be misspelled. In contrast, common last names also make it harder to differentiate between different authors. Examples are Anderson or Smith in United States, Chen or Lee in East Asian countries or Peters in Belgium.

% Wat moeten we bespreken in dit hoofdstuk:

% De testopstelling
	% Hoe zit deze in elkaar
	% Waarom hebben hiervoor gekozen
	% Vergelijking met anderen?
% De bekomen resultaten
% (Vergelijken met andere mensen)

% Moeten we dit doen voor de clustering (dus vooral kijken naar name disambiguation) en de expert finding (hiervoor hebben we voornamelijk de category builder nodig voor goede resultaten)

\section{Evaluation Setup}

\subsection{Comparative Research}

The authors of \cite{han2004two} focus on disambiguating between authors having the same name or names which are very closely related. They use a dataset consisting of nine different names, each having at least 10 different name variations. For each variation they have collected publications from DBLP. An overview of this dataset can be viewed on \autoref{tab:auth-dblp-dataset} together with the results they achieve using different rules for disambiguation. The mean accuracy they accomplish with their best approach is $73.3\%$ with a standard deviation of $5.4\%$.

\begin{table}
	\centering
		\begin{tabular}[ht]{|c||c|c|c|c|c|c|}
			\hline
			\bfseries{Name} & \bfseries{Variations} & \bfseries{Size} & \bfseries{Coauthor} & \bfseries{Paper} & \bfseries{Journal} & \bfseries{Combination} \\
			\hline
			S Lee & 35 & 244 & 61.3\% & 14.3\% & 43.8\% & 65.4\% \\
			\hline
			J Lee & 33 & 172 & 70.9\% & 17.7\% & 39.9\% & 75.9\% \\
			\hline
			J Kim & 25 & 127 & 57.1\% & 18.8\% & 40.2\% & 66.1\% \\
			\hline
			Y Chen & 24 & 108 & 78.5\% & 14.0\% & 26.9\% & 81.7\% \\
			\hline
			S Kim & 20 & 94 & 69.0\% & 13.8\% & 27.6\% & 70.1\% \\
			\hline
			C Lee & 18 & 80 & 72.2\% & 13.9\% & 43.1\% & 75.0\% \\
			\hline
			A Gupta & 16 & 172 & 75.0\% & 25.6\% & 50.6\% & 78.1\% \\
			\hline
			J Chen & 13 & 91 & 66.3\% & 31.3\% & 44.6\% & 72.3\% \\
			\hline
			H Kim & 11 & 63 & 73.7\% & 21.1\% & 43.9\% & 75.4\% \\
			\hline
			\hline
			\bfseries{Mean} & & & 69.3\% & 18.9\% & 40.0\% & \bfseries{73.3\%} \\
			\hline
			\bfseries{StdDev} & & & 6.8\% & 6.1\% & 7.9\% & 5.4\% \\
			\hline
		\end{tabular}
	\caption{The nine DBLP datasets used in \cite{han2004two}. Each row contains the base name, the number of recorded variations and the total amount of examined publications of this name. The last four columns show the accuracy measured using the different rules. The two bottom rows give the mean accuracy for the rules and the standard deviation.}
	\label{tab:auth-dblp-dataset}
\end{table}

It is important to note that the division of the clusters is solely based on the division of the authors in DBLP. As this division on DBLP is often wrong, we find that this makes a poor benchmark. Getting very high accuracy when comparing to DBLP does not mean the result is good. In order to get a proper insight into the disambiguation quality, we would still have to check the results manually. This is the main reason we choose to compose our own dataset.

\subsection{Test set}
\label{sec:testset}

As stated before, we want to resemble a realistic combination of easier and harder use-cases for our own dataset. As \autoref{tab:testset} shows, we have chosen five base names, each with a number of variations and have manually disambiguated these variations and combined the authors into clusters using the information on DBLP and the actual papers, if they were available. In total our test set contains just over a 1000 publications. An overview of the names that correspond to each of these base names, can be found in \autoref{appendix:testset}, together with links to DBLP. In order to make it easier, we will use the following definition in the rest of the text:

\begin{mydef}
	\bfseries{Family} With the term family, we refer to one of the five base names.
\end{mydef}

Nevertheless we composed the clusters manually, there are no guarantees that this is completely accurate. Especially for the families "Chen" and "Johnson", there might be small mistakes as the amount of different authors is overwhelming. However, the dataset we composed is a lot more accurate than DBLP. The comparison between the number of authors represented by DBLP and the number of authors we disambiguated, is also shown on \autoref{tab:testset}.

\begin{table}
	\centering
		\begin{tabular}[ht]{|c|c|c|c|}
			\hline
			\bfseries{Name Set} & \bfseries{Authors} & \bfseries{Publications} & \bfseries{DBLP} \\
			\hline
			Turck & 4 & 172 & 4 \\
			\hline
			Chen & 70 & 221 & 1 \\
			\hline
			Woo & 1 & 9 & 3 \\
			\hline
			Mens & 2 & 153 & 2 \\
			\hline
			Johnson & 107 & 460 & 64 \\
			\hline
		\end{tabular}
	\caption{The classification of our manually composed dataset. The first column contains the family. This is the name we used to find the authors on DBLP. The second column gives the number of clusters we defined manually for this family. The third column gives the number of publications we found on DBLP and the last column gives the number of different authors DBLP gives for this base name.}
	\label{tab:testset}
\end{table}

In order to be able to test the affiliation and the email rule, we had to get affiliations and email addresses for each of the authors of each publication. We started with writing a new pipe that would be able to search the publication on Bing \cite{bing}. The amount of papers that are actually available on Bing is abysmal, rendering this pipe useless. As the focus of our thesis is information processing, rather than retrieval, we manually added the affiliation and the email address of the author we are examining to each of the publications, if we could find them. This allows us to still test the rules.

\subsection{Setup}

A simplified overview of the test setup is shown in \autoref{fig:testsetup}. We start with the links of the author of a family we want to add. This is the input for the framework. The framework will search publications on DBLP for each of this link and combine them with the email and affiliation information that have been composed manually. The result is a graph containing clusters with the different authors. We extract the clusters from the graph and pass them to the result calculator. This is a script that calculates the precision, recall and $F_{1}$ measure by comparing the calculated clusters with the manually composed dataset, as explained in \autoref{sec:testset}, which is called the "ground truth" in \autoref{fig:testsetup}.

\begin{figure}[htb]
	\centering
		\includegraphics{./fig/testsetup.pdf}
	\label{fig:testsetup}
	\caption{A simplified overview of the test setup starting with the author name and ending with the accuracy results.}
\end{figure}

The result calculator calculates the precision, recall and F measure as defined in \autoref{eq:prf}. Unless explicitly stated otherwise, when we refer to F measure, we mean the $F_{1}$ measure where recall and precision are equally important. These statistical values are calculated for each of the clusters from the ground truth. 

\begin{equation}
	\label{eq:prf}
	\begin{array}{r c l}
		precision & = & \frac{\left|\left\{relevant~documents\right\} \cap \left\{retrieved~documents\right\}\right|}{\left|\left\{retrieved~documents\right\}\right|} \\
		recall & = & \frac{\left|\left\{relevant~documents\right\} \cap \left\{retrieved~documents\right\}\right|}{\left|\left\{relevant~documents\right\}\right|} \\
		F_{\beta} & = & ( 1 + \beta^{2} ) * \frac{precision * recall}{\beta^{2} * precision + recall} \\
	\end{array}
\end{equation}

If we denote the cluster of the ground truth as $C_g$, then the calculated cluster from the graph the result calculator will $C_g$ compare with is given by 

\begin{equation}
	\label{eq:calccluster}
	\min_{\forall c \in calculated~clusters}{\left| C_g \setminus c \right|}
\end{equation}

In \autoref{eq:prf}, we use the terms relevant and retrieved documents. The explanations of these terms are given by the following definitions.

\begin{mydef}
	\textbf{Relevant documents} The publications from the cluster in the ground truth.
\end{mydef}

\begin{mydef}
	\textbf{Retrieved documents} The publications from the cluster calculated by the framework belonging to the cluster in the ground truth, calculated in \autoref{eq:calccluster}.
\end{mydef}

After calculating these statistical values for all the clusters in the ground truth, the mean F measure is calculated to give us an idea of the accuracy for a given test.

\section{Results}

\subsection{Synchronous versus Asynchronous Execution}

Any connection in a pipe network can be configured to be local or asynchronous. Asynchronous connections push the flow in a shared queue of flows that need to be processed. A pool of workers on different machines will poll this queue and resume the flow on their own machines, making the pipe network distributed. We expect this scaling to entail an increase in performance.

We notice that a pipe network that is executed synchronously spends more than 50\% of its time waiting for input. This is because accessing websites, distributed memory or the graph is associated with a certain latency causing pipes to wait for the service to respond. The amount of time spent waiting heavily depends on the characteristics of the input and the hardware. Some inputs require more computation and others more communication. When we run three workers on the same machine, we can already perceive a speedup of 200-300\%.

We expect that distributing the pipes over multiple servers would have a enormous impact on performance. The pipes try to minimize the amount of communication with the graph database as much as possible, making the system more scalable. As we do not have decent server hardware at our disposal, we were not able to test this.

\subsection{Rules}

We have tested the influence of the different rules on the accuracy. For each of the families, we tested the same combination of rules in succession. The F measure for each of these combinations can be seen on \autoref{fig:test-rules}. The combination of all four rules, renders the best result, although sometimes the increase in accuracy from an additional rule is minimal. For "Chen", adding the affiliation rule to the community and email rule even results in a small decrease in accuracy. This is because certain authors are clustered together wrongly. The exact F measures for this combination can be found in \autoref{table:comparison-dblp}. 

\begin{figure}[htb]
	\centering
		\includegraphics[width=0.80\textwidth]{./fig/test-rules.pdf}
	\caption{A comparison of the accuracy, measured as F score, for the different rules and combinations. The first column shows the use of just the community rule (community rule). The second column shows it in combination with the email rule (email rule). The third rule is a combination of the previous two with the affiliation rule (affiliation rule), while the fourth is a combination of all four rules (keyword rule). The fifth column shows the combination of the email, affilation and keyword rule. The last column gives the so-called baseline, this is the F measure we get when all authors are considered different authors.}
	\label{fig:test-rules}
\end{figure}

The weights we used for the different rules are the same in each of the tests and is the result of \autoref{sub:weights}. We used the weight distribution called "highkey", as denoted in \autoref{table:highkey}, as this distribution yielded the highest average accuracy.


% bespreek figuur

\subsection{Weights}
\label{sub:weights}

We have tested different values for the weights allocated to each of the rules and also the value of $\alpha$ which is used to determine if reclustering should occur. The results are shown on \autoref{fig:test-weights}, while the different distributions are depicted in \autoref{table:distributions}. The average F measure for each distribution over the five families is given by:

\begin{table}[ht]
	\center
	\begin{tabular}{|c|c|}
		\hline
		\bfseries{Distribution} & \bfseries{Average F measure} \\
		\hline
		basic & 81,5\% \\
		\hline
		lowkey & 72,7\% \\
		\hline
		highco & 79,0\% \\
		\hline
		highkey & 84,5\% \\
		\hline
	\end{tabular}
	\caption{The average F measure over the five families for each of the weight distributions.}
	\label{table:avg-f-distr}
\end{table}

The distribution "highkey" gets the best average accuracy. This distribution gives high values to all properties, favoring a lot of clustering, while the higher alpha makes sure there is still a threshold. Only for family "Johnson", it doesn't get the highest F measure. This can be attributed to too much clustering, resulting in a higher recall, but a lower precision. The actual F measures for this distribution can be seen in \autoref{comparison-dblp}.

\begin{table}[ht]
	\begin{minipage}[b]{0.5\linewidth}\centering
		\begin{tabular}{|c|c|}
			\hline
			\bfseries{Property} & \bfseries{Weight} \\
			\hline
			$\alpha$ & 25 \\
			\hline
			keyword rule & 4\\
			\hline
			community rule & 8\\
			\hline
			affiliation rule & 10\\
			\hline
			email rule & 1000\\
			\hline
		\end{tabular}
		\caption{Basic}
	\end{minipage}
	\begin{minipage}[b]{0.5\linewidth}
		\centering
		\begin{tabular}{|c|c|}
			\hline
			\bfseries{Property} & \bfseries{Weight} \\
			\hline
			$\alpha$ & 25 \\
			\hline
			keyword rule & 1\\
			\hline
			community rule & 8\\
			\hline
			affiliation rule & 10\\
			\hline
			email rule & 1000\\
			\hline
		\end{tabular}
		\caption{Lowkey}
	\end{minipage}
	\begin{minipage}[b]{0.5\linewidth}\centering
		\begin{tabular}{|c|c|}
			\hline
			\bfseries{Property} & \bfseries{Weight} \\
			\hline
			$\alpha$ & 25 \\
			\hline
			keyword rule & 1\\
			\hline
			community rule & 50\\
			\hline
			affiliation rule & 10\\
			\hline
			email rule & 1000\\
			\hline
		\end{tabular}
		\caption{Highco}
	\end{minipage}
	\begin{minipage}[b]{0.5\linewidth}
		\centering
		\begin{tabular}{|c|c|}
			\hline
			\bfseries{Property} & \bfseries{Weight} \\
			\hline
			$\alpha$ & 25 \\
			\hline
			keyword rule & 10\\
			\hline
			community rule & 50\\
			\hline
			affiliation rule & 10\\
			\hline
			email rule & 1000\\
			\hline
		\end{tabular}
		\caption{Highkey}
		\label{table:highkey}
	\end{minipage}
	\caption{Weight distributions with the name of each distribution as used in \autoref{fig:test-weight}}
	\label{table:distributions}
\end{table}

\begin{figure}[htb]
	\centering
		\includegraphics[width=0.80\textwidth]{./fig/test-weights.pdf}
	\caption{The F measure for different weight distributions for each family. The weight distributions, from left to right, are "basic", "lowkey", "highco" and "highkey". The actual value for the distributions is shown on \autoref{table:distributions}.}
	\label{fig:test-weights}
\end{figure}

\subsection{Comparing to DBLP}

We have calculated the F measure for each of the families as they are divided on DBLP, to make a comparison with our own results. The values are shown on \autoref{table:f-dblp}. The divisions for "Turck" and "Mens" are completely correct, the other three are far less accurate with "Chen" having an astonishingly low F measure of $2.73\%$. We compare the F measure to our highest average scoring distribution, "highkey". We also compare the mean F measures and a weighted variation of the mean measure which is calculated using the following formula ($F_X(f)$ is the F measure for family name $f$ for $X$):

	\[
	\sum_{f \in families}{\frac{F_{X}(f)}{|f_{publications}} * N},~X \in \left\{DBLP,highkey\right\},~N = total~number~of~publications
\]

Our framework scores $14\%$ to $17\%$ better than DBLP, depending on how the mean value is measured.

\begin{table}
	\centering
		\begin{tabular}{|c|c|c|c|c|c|c|c|}
			\hline
			& \bfseries{Turck} & \bfseries{Chen} & \bfseries{Woo} & \bfseries{Mens} & \bfseries{Johnson} & \bfseries{Mean} & \bfseries{Weighted} \\
			\hline
			\bfseries{DBLP} & 100.0\% & 87.5\% & 100.0\% & 2.7\% & 62.8\% & 70.6\% & 61.8\% \\
			\hline
			\bfseries{Highkey} & 100.0\% & 94,1\% & 89,8\% & 63,7\% & 74,7\% & 84,5\% & 79,0\% \\
			\hline
		\end{tabular}
	\caption{Comparison of the F measures for the different families as they are divided on DBLP and as calculated by our framework. The last columns give the mean F measure and a weighted distribution based on the number of papers in each family.}
	\label{tab:comparison-dblp}
\end{table}

%\include{gerelateerdwerk}
%\include{gelijkheid}
%\include{aanbevelingen}

\chapter{Conclusion and Future Work}

%Our system only provides a snapshot of the expertise on a given time, it should be better if we could provide an overview in time.

\section{Conclusion}

In this master's thesis we examined the opportunities using the semantic web and data processing. We found out that the possibilities within expert finding and author disambiguation are challenging and can contribute in solving a real-life problem. At the end of this informative investigation, the objective became clear: creating a framework that is able to disambiguate authors and allows users to find experts for a certain subject.

% Wat is juist zo naief aan de eerste aanpak en waarop is deze precies gebaseerd, in 1 zin

We started with a naive approach for the framework. We focused on what we knew and were comfortable with rather than the problem we had to solve. It lacked both in performance and scalability, while the usage of a relational database would render it unrealizable. Profound inspection of this first version of the framework, however, learned us that in order to create a good one, it would have to amplify the following five foundations:

\begin{enumerate}
	\item All instances are considered different authors until proven otherwise.
	\item No decision is made permanent.
	\item Any information is considered partial information.
	\item A constantly changing input asks for a constantly changing output.
	\item The stream of informaton is endless.
\end{enumerate}

% Iets over graaf en regels

With these foundations in mind, we first created a theoretical model. This model consists of three layers that integrate structural, informational and algorithmic aspects. It deals with the main difficulties of author disambiguation, while minimizing the problem domain and we explained how to process the (partial) information in order for similarities to be found. 

Rules drive the entire flow in our framework by converting novel facts into similarities that group instances with authors through clustering. The four rules we defined for our framework are:

\begin{enumerate}
	\item \textit{Community Rule} Exploiting the fact that authors often work together with the same co-author.
	\item \textit{Interest Rule} The subjects of publications of the same author are usually located within the same field of research.
	\item \textit{Email Rule} Authors with the same email address, are most likely the same person.
	\item \textit{Affiliation Rule} Authors are more likely to work at one affiliation at a given time.
\end{enumerate}

We implemented this theoretical model with pipes and filters using Ruby, Java, a key-value store Redis, Resque for scaling and the Tinkerpop stack for the graph representation. The usage of pipes and filters allows for modifiability. New pipes can easily be plugged into the flow which enables the convenient addition of extra information sources or new rules while the performance and scalability remain intact.

Clustering is the process that is responsible for grouping of authors based on the calculated similarities, a key component of our framework. We implemented the dynamic clustering algorithm proposed in \cite{saha2006dynamic} which is based on minimum-cut trees. The quality of the clusters is guaranteed to be the same as its static counterpart, while it is a lot faster as it considers much smaller subgraphs. We added an extra case to the proposed algorithm, exploiting the fact that our framework often has a succession of small similarities connecting the same clusters. This extra case permits the most resource-intensive case to be postponed and executed less frequently.

% Expert finding ?

% TODO precieze resultaat waarden invullen, gemiddelde waarde en vergelijking met dblp

We tested the framework thoroughly using a manually annotated testset of just over 1000 publications. We showed the impact of each of the rules on the accuracy, clarifying that the combination of all four of them render the best results, although the increased accuracy from the addition of a rule is minimal in certain cases. We also compared the accuracy of our framework with the accuracy of DBLP for the tested authors and concluded that our framework transcends DBLP.

\section{Future Work}

% Rekening houden met negatieve gewichten
% Absolute vaststellingen kunnen maken : deze 2 auteurs zijn ZEKER verschillend en moeten altijd zo blijven
% Opstellen van categorie boom om betere keyword results te krijgen
% Grotere gewichten toekennen voor woorden die meer subject specifiek zijn
% Scalability fixen :o
% Meer bronnen opnemen
% Category bepaling niet enkel op basis van titel, maar ook abstract of zelfs volledige tekst => plugins schrijven voor verschillende paper aanbied sites (zijn er slechts een paar)
% 

% appendices
\appendix
\chapter{Dataset}
\label{appendix:dataset}
The testset we composed is subdivided in five base names. For each of these names we searched for publications on DBLP. In order to be able to recalculate the results or make a comparison with our results, we will give an overview of the links we used to find the publications for each of these names.

The testset with the clusters we defined manually can be found on the DVD in the map $truth$. For each of the five base names, the clusters can be found in the file \textit{$<$base\_name$>$\_parsed.json} in JSON format.

\section*{Turck}

The base name "Turck" is a combination of the authors "Filip De Turck", "Natacha Turck", "Clemens T\"urck" and "Koen De Turck". The publications can be found with the following links:

\begin{verbatim}
http://dblp.uni-trier.de/db/indices/a-tree/t/Turck:Filip_De.html
http://dblp.uni-trier.de/db/indices/a-tree/t/Turck:Natacha.html
http://dblp.uni-trier.de/db/indices/a-tree/t/T=uuml=rck:Clemens.html
http://dblp.uni-trier.de/db/indices/a-tree/t/Turck:Koen_De.html
\end{verbatim}

\section*{Woo}

The base name "Woo" is one author "Chong Woo". On DBLP he has three pages which we combined to create one cluster:

\begin{verbatim}
http://dblp.uni-trier.de/db/indices/a-tree/w/Woo:Chong.html
http://dblp.uni-trier.de/db/indices/a-tree/w/Woo:Chong=Woo.html
http://dblp.uni-trier.de/db/indices/a-tree/w/Woo:Chongwoo.html
\end{verbatim}

\section*{Chen}

The base name "Chen" corresponds with "Yu Chen" on DBLP. There he is depicted as one author with 221 publications, while manual checking showed us that this corresponds with approximatly 70 different authors.

\begin{verbatim}
http://dblp.uni-trier.de/db/indices/a-tree/c/Chen:Yu.html
\end{verbatim}

\section*{Mens}

The base name "Mens" is a combination of the authors "Tom Mens" and "Kim Mens", a brother and sister working at the same affiliation. The links from DBLP are:

\begin{verbatim}
http://dblp.uni-trier.de/db/indices/a-tree/m/Mens:Kim.html
http://dblp.uni-trier.de/db/indices/a-tree/m/Mens:Tom.html
\end{verbatim}

\section*{Johnson}

The base name "Johnson" corresponds with a combination of authors which all map to "D. Johnson". We made a combination of 64 different authors from DBLP, which can all be found with the following links:

\begin{verbatim}
http://www.informatik.uni-trier.de/~ley/db/indices/a-tree/j/Johnson:D=.html
http://www.informatik.uni-trier.de/~ley/db/indices/a-tree/j/Johnson:D=_A=_H=.html
http://www.informatik.uni-trier.de/~ley/db/indices/a-tree/j/Johnson:D=_Aaron.html
http://www.informatik.uni-trier.de/~ley/db/indices/a-tree/j/Johnson:D=_B=.html
http://www.informatik.uni-trier.de/~ley/db/indices/a-tree/j/Johnson:D=_D=.html
http://www.informatik.uni-trier.de/~ley/db/indices/a-tree/j/Johnson:D=_E=.html
http://www.informatik.uni-trier.de/~ley/db/indices/a-tree/j/Johnson:D=_G=.html
http://www.informatik.uni-trier.de/~ley/db/indices/a-tree/j/Johnson:D=_H=.html
http://www.informatik.uni-trier.de/~ley/db/indices/a-tree/j/Johnson:D=_R=.html
http://www.informatik.uni-trier.de/~ley/db/indices/a-tree/j/Johnson:D=_Randolph.html
http://www.informatik.uni-trier.de/~ley/db/indices/a-tree/j/Johnson:D=_T=.html
http://www.informatik.uni-trier.de/~ley/db/indices/a-tree/j/Johnson:Dabney.html
http://www.informatik.uni-trier.de/~ley/db/indices/a-tree/j/Johnson:Dale_A=.html
http://www.informatik.uni-trier.de/~ley/db/indices/a-tree/j/Johnson:Dale_M=.html
http://www.informatik.uni-trier.de/~ley/db/indices/a-tree/j/Johnson:Damian.html
http://www.informatik.uni-trier.de/~ley/db/indices/a-tree/j/Johnson:Dan.html
http://www.informatik.uni-trier.de/~ley/db/indices/a-tree/j/Johnson:Dana.html
http://www.informatik.uni-trier.de/~ley/db/indices/a-tree/j/Johnson:Dana_M=.html
http://www.informatik.uni-trier.de/~ley/db/indices/a-tree/j/Johnson:Daniel.html
http://www.informatik.uni-trier.de/~ley/db/indices/a-tree/j/Johnson:Daniel_Ezra.html
http://www.informatik.uni-trier.de/~ley/db/indices/a-tree/j/Johnson:Daniel_P=.html
http://www.informatik.uni-trier.de/~ley/db/indices/a-tree/j/Johnson:Daniel_R=.html
http://www.informatik.uni-trier.de/~ley/db/indices/a-tree/j/Johnson:Daphne.html
http://www.informatik.uni-trier.de/~ley/db/indices/a-tree/j/Johnson:Darin.html
http://www.informatik.uni-trier.de/~ley/db/indices/a-tree/j/Johnson:Darin_B=.html
http://www.informatik.uni-trier.de/~ley/db/indices/a-tree/j/Johnson:Darrel_Eric.html
http://www.informatik.uni-trier.de/~ley/db/indices/a-tree/j/Johnson:Daryl.html
http://www.informatik.uni-trier.de/~ley/db/indices/a-tree/j/Johnson:Dave.html
http://www.informatik.uni-trier.de/~ley/db/indices/a-tree/j/Johnson:David.html
http://www.informatik.uni-trier.de/~ley/db/indices/a-tree/j/Johnson:David_A=.html
http://www.informatik.uni-trier.de/~ley/db/indices/a-tree/j/Johnson:David_B=.html
http://www.informatik.uni-trier.de/~ley/db/indices/a-tree/j/Johnson:David_C=.html
http://www.informatik.uni-trier.de/~ley/db/indices/a-tree/j/Johnson:David_E=.html
http://www.informatik.uni-trier.de/~ley/db/indices/a-tree/j/Johnson:David_H=.html
http://www.informatik.uni-trier.de/~ley/db/indices/a-tree/j/Johnson:David_K=.html
http://www.informatik.uni-trier.de/~ley/db/indices/a-tree/j/Johnson:David_L=.html
http://www.informatik.uni-trier.de/~ley/db/indices/a-tree/j/Johnson:David_Lloyd.html
http://www.informatik.uni-trier.de/~ley/db/indices/a-tree/j/Johnson:David_O=.html
http://www.informatik.uni-trier.de/~ley/db/indices/a-tree/j/Johnson:David_R=.html
http://www.informatik.uni-trier.de/~ley/db/indices/a-tree/j/Johnson:David_S=.html
http://www.informatik.uni-trier.de/~ley/db/indices/a-tree/j/Johnson:David_W=.html
http://www.informatik.uni-trier.de/~ley/db/indices/a-tree/j/Johnson:Dean.html
http://www.informatik.uni-trier.de/~ley/db/indices/a-tree/j/Johnson:Deborah_G=.html
http://www.informatik.uni-trier.de/~ley/db/indices/a-tree/j/Johnson:Delia.html
http://www.informatik.uni-trier.de/~ley/db/indices/a-tree/j/Johnson:Derek.html
http://www.informatik.uni-trier.de/~ley/db/indices/a-tree/j/Johnson:Derek_M=.html
http://www.informatik.uni-trier.de/~ley/db/indices/a-tree/j/Johnson:Desiree_C=.html
http://www.informatik.uni-trier.de/~ley/db/indices/a-tree/j/Johnson:Diane_Tobin.html
http://www.informatik.uni-trier.de/~ley/db/indices/a-tree/j/Johnson:Dianne.html
http://www.informatik.uni-trier.de/~ley/db/indices/a-tree/j/Johnson:Dion_L=.html
http://www.informatik.uni-trier.de/~ley/db/indices/a-tree/j/Johnson:Don.html
http://www.informatik.uni-trier.de/~ley/db/indices/a-tree/j/Johnson:Don_H=.html
http://www.informatik.uni-trier.de/~ley/db/indices/a-tree/j/Johnson:Don_W=.html
http://www.informatik.uni-trier.de/~ley/db/indices/a-tree/j/Johnson:Donald.html
http://www.informatik.uni-trier.de/~ley/db/indices/a-tree/j/Johnson:Donald_B=.html
http://www.informatik.uni-trier.de/~ley/db/indices/a-tree/j/Johnson:Donald_Byron.html
http://www.informatik.uni-trier.de/~ley/db/indices/a-tree/j/Johnson:Donald_W=.html
http://www.informatik.uni-trier.de/~ley/db/indices/a-tree/j/Johnson:Donna_L=.html
http://www.informatik.uni-trier.de/~ley/db/indices/a-tree/j/Johnson:Doug.html
http://www.informatik.uni-trier.de/~ley/db/indices/a-tree/j/Johnson:Doug_V=.html
http://www.informatik.uni-trier.de/~ley/db/indices/a-tree/j/Johnson:Douglas.html
http://www.informatik.uni-trier.de/~ley/db/indices/a-tree/j/Johnson:Douglas_A=.html
http://www.informatik.uni-trier.de/~ley/db/indices/a-tree/j/Johnson:Duane_D=.html
http://www.informatik.uni-trier.de/~ley/db/indices/a-tree/j/Johnson:Dustin.html
\end{verbatim}
\chapter{DVD Content}

In the listing below, we summarize what can be found in each folder on the included DVD.

\paragraph{boek} The book in \LaTeX with all figures in pdf and vsd.

\paragraph{clustering} The clusteringalgorithm java project, all dependencies are included in the lib folder.

\paragraph{config} Configuration of the parameters of the disambiguation algorithm in config.rb, the entire configuration of the graph network in pipes.rb and the tinkerpop configuration file graph.xml.

\paragraph{exports} A collection of exports of the state of the graph after clustering. Exports are for all data used in the discussion of the results.

\paragraph{lib} The framework itself.

\paragraph{spec} Spec-tests for the application, currently only some teste for instance integration in the graph.

\paragraph{truth} The manually annotated ground truth used for comparison with our results.

\paragraph{vendor} Directory for plugins, currently holding the Rexster library.
% hier worden de appendices ingevoegd (\includes)


%\include{referenties}

% De bibliografie en de index
\bibliography{collection}
\backmatter

% eventueel: lijst van figuren en tabellen
\listoffigures
\listoftables

% lege pagina (!!)

% kaft

\end{document}
