%
%  THESISBOEK
%
%  Dit bestand zorgt voor algemene (layout)definities, en groepeert de
%  afzonderlijke LaTeX-files tot een geheel.
%
%  @author Erwin Six, David De Reu, Brecht Vermeulen
%

\documentclass[11pt,a4paper,oneside,notitlepage]{book}
\usepackage[english]{babel}
\usepackage{algorithmic}
\usepackage{algorithm}
\usepackage{amsthm}
\usepackage{hyperref}
\usepackage{array}
%\usepackage[nottoc]{tocbibind} % Bibliografie in ToC; zie tocbibind.dvi

% marges aanpassen
% (opmerking: moet *voor* inclusie van fancyhdr package komen)
\setlength{\hoffset}{-1in}
\setlength{\voffset}{-1in}
\setlength{\topmargin}{2cm}
\setlength{\headheight}{0.5cm}
\setlength{\headsep}{1cm}
\setlength{\oddsidemargin}{3.5cm}
\setlength{\evensidemargin}{3.5cm}
\setlength{\textwidth}{16cm}
\setlength{\textheight}{23.3cm}
\setlength{\footskip}{1.5cm}

\usepackage{fancyhdr}
\usepackage{graphicx}
% \usepackage[colorlinks]{hyperref}
% Het bibliografisch opmaak bestand.
\bibliographystyle{unsrt}
%\bibliographystyle{bibliodutch}
%\bibpunct{[}{]}{,}{n}{,}{,}

\newtheorem{mydef}{Definition}
\newtheorem{foundation}{Foundation}

\pagestyle{fancy}

\renewcommand{\chaptermark}[1]{\markright{\MakeUppercase{#1}}}
\renewcommand{\sectionmark}[1]{\markright{\thesection~#1}}

\newcommand{\headerfmt}[1]{\textsl{\textsf{#1}}}
\newcommand{\headerfmtpage}[1]{\textsf{#1}}

\fancyhf{}
\fancyhead[LE,RO]{\headerfmtpage{\thepage}}
\fancyhead[LO]{\headerfmt{\rightmark}}
\fancyhead[RE]{\headerfmt{\leftmark}}
\renewcommand{\headrulewidth}{0.5pt}
\renewcommand{\footrulewidth}{0pt}

\fancypagestyle{plain}{ % eerste bladzijde van een hoofdstuk
  \fancyhf{}
  \fancyhead[LE,RO]{\headerfmtpage{\thepage}}
  \fancyhead[LO]{\headerfmt{\rightmark}}
  \fancyhead[RE]{\headerfmt{\leftmark}}
  \renewcommand{\headrulewidth}{0.5pt}
  \renewcommand{\footrulewidth}{0pt}
}

% anderhalve interlinie (opm: titelblad gaat uit van 1.5)
\renewcommand{\baselinestretch}{1.5}

% indien LaTeX niet goed splitst, neem je woord hierin op, of evt om splitsen 
% te voorkomen
\hyphenation{ditmagnooitgesplitstworden dit-woord-splitst-hier}

\begin{document}

%!!!!!!!!!!!!!!!!!!!!!!!!!!!!!!!!!!!!!!!!!!!!!!!!!!!!!!!!!!!!!!!!!!!!!!!!!!!!!!!!!!!!!!!!!!!!!!!!!
%!!!!!!!!!!!              onderaan/bovenaan elk blad thesistitel zetten                !!!!!!!!!!!
%!!!!!!!!!!!!!!!!!!!!!!!!!!!!!!!!!!!!!!!!!!!!!!!!!!!!!!!!!!!!!!!!!!!!!!!!!!!!!!!!!!!!!!!!!!!!!!!!!

% overzicht/samenvatting
%%  Overzichtsbladzijde met samenvatting

\newpage

{
\setlength{\baselineskip}{32pt}
\setlength{\parindent}{0pt}
\setlength{\parskip}{18pt}

\begin{center}

\noindent \textbf{\huge
Identifying experts through }
\textbf{\huge a framework for knowledge extraction}
\textbf{\huge from public online sources}

\setlength{\baselineskip}{12pt}
\setlength{\parindent}{0pt}
\setlength{\parskip}{12pt}

door 

Simon Buelens, Mattias Putman

Promotors: Prof.~Dr.~Ir.~Filip~De~Turck,~Elena~Tsiporkova~(Sirris),~Tom~Tourw\'{e}~(Sirris)\\
Scriptiebegeleiders: Anna~Hristoskova,~Tim~Wauters\\

Masterproef ingediend tot het behalen van de academische graad van\\
Master in de ingenieurswetenschappen: computerwetenschappen

Academiejaar 2010-2011\\
Faculteit Ingenieurswetenschappen\\
Voorzitter: Prof. Dr. Ir. Dani\"{e}l De Zutter\\
Vakgroep Informatietechnologie\\

\end{center}

\setlength{\baselineskip}{10pt}
\setlength{\parindent}{0pt}
\setlength{\parskip}{10pt}

\renewcommand{\baselinestretch}{1.1} 	% De interlinie afstand wat vergroten.
\small\normalsize                       % Nodig om de baselinestretch goed te krijgen.

\section*{Samenvatting}

Onderzoekers verliezen veel tijd met de zoektocht naar informatie gerelateerd aan hun onderzoeksdomein. Er bestaan bijna geen diensten die toelaten om aan de hand van trefwoorden een overzicht te verkrijgen met experts voor de opgegeven domeinen. Er is onderzoek naar disambiguatie van auteurs, maar deze worden meestal niet in combinatie gebracht met het opzoeken van experten, maar het indelen van publicaties (alhoewel de twee gerelateerd zijn).

In deze masterproef onderzoeken we de mogelijkheid om een framework op te stellen dat dit toelaat door online informatie op te zoeken, deze informatie in relatie te brengen met de correcte auteurs en gebruikers toe te laten dit framework te gebruiken om hierin te zoeken. De nadruk van het framework ligt op de disambiguatie van auteurs (aan de hand van de aanwezige informatie alle namen zo goed mogelijk connecteren met de juist auteur) aan de hand van een regelgebaseerde aanpak en de uitbreidbaarheid van het framework.

We maken gebruik van een graafgebaseerde representatie van de data en de architectuur is gebaseerd op pipes en filters. Dit laat toe dat het framework uitbreidbaar, schaalbaar en eenvoudig aanpasbaar is. Op het einde volgen de resultaten gebaseerd op een manueel geannoteerde testset. Uiteindelijk gaan we ook de vergelijking aan met de verdeling van auteurs door DBLP.

\section*{Trefwoorden}

auteur disambiguatie, data verwerking, clustering, pipes en filters

}

\newpage % strikt noodzakelijk om een header op deze blz. te vermijden


\pagestyle{fancy}
\frontmatter

\setlength{\parindent}{0pt}
\setlength{\parskip}{0.5\baselineskip plus 0.5ex minus 0.2ex}
\setlength{\parskip}{1ex plus 0.5ex minus 0.2ex}

% hoofdstukken
\mainmatter

\chapter{Problem Insights and Foundation}

In the first framework, many properties of our environment were not taken into account. We did not think enough about the consequences of the characteristics of the disambiguation problem. Therefore, we went back on our steps and formulated five basic foundation acknowledgement. This new foundation has a lot of contradictions with the design of our first framework. We decided to abandon the framework. In the next chapter we build an entire theoretical model based on this foundation. The chapter after that discusses the practical realisation of this model.

\section{Foundation Acknowledgements}
\label{foundation}

Below, we list the five foundation acknowledgement that drive the development of a theoretical model.

\begin{foundation}
\label{foundation:different}
All instances are considered different authors until proven otherwise.
\end{foundation}

First, we want a solution that provides precision. The most existing services do simple grouping on equal names. This causes the service to return all the publications for the person that we are searching, but also a lot of papers that belong elsewhere. In other words, they have a high recall, but a low precision. Reaching a high recall is much less difficult than obtaining a high precision while still having a relatively high recall. To obtain this high precision, we only decide that two authors are actually the same if enough evidence is available. When a piece of information on an author with a certain name is collected (we call this an instance of an author), we always consider this information to be an author on its own. This way, we do not make unsupported decisions.

\begin{foundation}
\label{foundation:decision}
No decision is made permanent.
\end{foundation}

We operate in a highly dynamic environment, subject to constant change. This change is controlled from two sides. The first is the one of information itself. As time goes on, more and more information is produced (authors write new publications) and more of this information becomes available on the web. The second is the one of our framework. Its capabilities will constantly improve, making it possible to collect information that was not collectable before. The decision to which physical author information belongs to can change as new information becomes available. Therefore it is wise to not base a decision only on past information. Future information can provide new motives to make another decision. It has to be possible to change our minds though time.

\begin{foundation}
\label{foundation:partial}
Any information is considered partial information.
\end{foundation}

No information can be considered complete and correct. We will substantiate this rule with an example. Suppose that a publication document is scanned for author names. Two things can go wrong here. Fist, the information in the paper can either be incomplete or plain wrong. The publication could accidentally not mention a writer or his name could be misspelled. Secondly, a case that is not recognized by the framework could result is an incomplete extraction of the information. Assuming that the information would be complete would not leave room to add the missing author later or straighten the spelling mistake.

\begin{foundation}
\label{foundation:incremental}
A constantly changing input asks for a constantly changing output.
\end{foundation}

This foundation is based on the same observations as Foundation \autoref{foundation:decision}. In this highly dynamic environment, there is constant change. We want our output, disambiguated authors, to adapt to this constant change. This means that, ideally, for every change of information in the input, a changed output would be precieved almost instantaneous. To achieve this, changes in decisions must be processed efficiently.

\begin{foundation}
\label{foundation:endless}
The stream of informaton is endless.
\end{foundation}

Last but not least, do not forget that the stream is almost endless. Information that would be relevant to the process of disambiguating authors is immense. Processing information of some sources can also be very intensive (ex. OCR scanning PDF documents). If the framework plans to cope with this endless growing stream, it needs to scale.

\section{Why the first approach needed a complete overhaul}

Now we can check if the measures taken in the first framework suffice to cope with the challenges posed in the foundation acknowledgements. Different aspects of the framework will be treated seperately to determine to what degree they meet the requirements.

\paragraph{The flow of data} A logical flow of data in a system with this much data to process is of primary importance. In the framework, several components do not comunicate directly with each other. All communication is "relayed" through the database. This puts enormous unnecessary load on our database. When we would scale the framework out to several servers, all communication would still pass the database nodes (over the network).

\paragraph{Scalability} We have more than one concern about the scalability of the first framework. First, the more instances in the cluster, the more queries to the database to retreive the information about a task. Secondly, the data merger would put an enormous load on the database from time to time (see next paragraph). Thirdly, a task dependency control component would also relay all its control messages through the database, increasing load in this point.

\paragraph{Performance} At certain intervals, the data merger would take a snapshot of the entire system in order to execute a cluster algorithm on it (grouping publications in physical author clusters). This is a very intensive process. As we deal with a very dynamic environment, it is preferrable that this algorithm would be run timely to keep the output up to date. The drawback with this approach is that "snapshot"-clustering is a very time consuming operation every time it is executed. If we had an incremental clustering algorithm that modifies the clusters when new information is found, it would only be intensive the first time. Afterwards it would just have to make a small modification to the clusters. This kind of approach would be more suitable in this dynamic situation.

\paragraph{The right tool for the job} In the first framework, we chose MySql as our database technology. We used it because it is familiar and widely supported. In practice, the complex queries that needed to be made were very impractical and slow due to the many joins. It occured to us that a graph database would be a better tool for the job. In \cite{vicknair2010comparison}, we find a comparison between Mysql and Neo4j, a leading graph database technology. When querying with a large depth, Neo4j is the clear winner.

\paragraph{Conclusion} Out of these experiences we learn that our new model and framework must meet some new requirements and avoid some pitfalls. In short, we need to improve the communication between the components of the framework, use a graph-database instead of a relational database and replace the data merger with a more dynamic solution. It would also be preferrable if the new framework was more scalable than the first version. We elaborate on this better framework in \autoref{newframework}.
\backmatter
\end{document}
